\documentclass[12pt]{article}

% --------------------
% PAQUETES BÁSICOS
% --------------------
\usepackage[spanish,es-noshorthands]{babel}
\usepackage[utf8]{inputenc}      % Si usas pdflatex
\usepackage[T1]{fontenc}
\usepackage{lmodern}

\usepackage{amsmath, amssymb, amsthm}
\usepackage{geometry}
\usepackage{setspace}
\usepackage[
  colorlinks=true,
  linkcolor=black,
  citecolor=blue,
  urlcolor=blue
]{hyperref}


\DeclareMathOperator{\Frm}{Frm}
\DeclareMathOperator{\Top}{Top}
\DeclareMathOperator{\pt}{pt}
% --------------------
% CONFIGURACIÓN
% --------------------
\geometry{margin=2.5cm}
\onehalfspacing

% --------------------
% DATOS DEL DOCUMENTO
% --------------------
\title{Título del documento}
\author{Nombre del autor}
\date{\today}

% --------------------
% INICIO DEL DOCUMENTO
% --------------------
\begin{document}

%\maketitle

\section{Cosas para trabajar durante el periodo 2026 A}
Para la mejor comprensión del presente documento, incluimos la notación básica que será manejada en los distintos puntos que 
más adelante se enunciaran.\\

Sea $A\in \Frm$ un marco, entonces $PA\subseteq NA$ denota el marco de parches, donde $NA$ denota el conjunto de todos los núcleos sobre $A$.\\

Consideremos $\nabla(j)=\{a\in A\mid j(a)=1\}$. Se puede verificar que el anterior es un filtro y es conocido como filtro de admisibilidad. Usando los filtros de admisibilidad definimos la siguiente relación de equivalencia:
\[
j\sim k \iff \nabla(j)=\nabla(k).
\] 
donde $j, k\in NA$.\\

Cada clase de equivalencia proporciona un bloque de elementos en $NA$. Cuando $F=\nabla(j)$ es un filtro abierto, este bloque tiene asociado un mayor y un menor elemento. Dicho bloque tiene la forma $[v_F, w_F]$ y lo llamamos intervalo de admisibilidad.\\

Si $p\in \pt A$, entonces podemos asociarle un elemento $w_p\in \pt NA$ dado por 
\[
w_p(x)=\left\{
	\begin{array}{lcc}
1 & \mbox{ si } & x\nleq p\\
\\
p & \mbox{ si } & x\leq p
\end{array}\right.
\]
Sabemos que $PA\subseteq NA$ y aplicando el funtor $\pt$ obtenemos $\pt NA\subseteq \pt PA$. Por lo tanto $w_p\in \pt PA$. De esta manera, para $p\in \pt A$ decimos que $p$ es un punto ajustado si $w_p$ es un núcleo ajustado.\\

Por último, incluimos un breve resumen de algunas propiedades de separación en $\Frm$. Para cualesquiera $a\nleq b\in A$ tenemos que $A$ es:
\begin{itemize}
    \item \textbf{(H)} si $\exists\, c\in A$ tal que $c\nleq a$ y $\neg c\leq b$, con $a\neq 1$.
    \item \textbf{(aju)} si $\exists\, x,y\in A$ tales que $x\vee a=1, y\nleq b$ y $x\wedge y\leq b$.
    \item \textbf{(saju)} si $\exists\, c\in A$ tal que $c\vee a=1\neq c\vee b$.
\end{itemize}

\subsection{Lista de actividades}
En la siguiente lista se presentan las principales actividades que realizaré durante el semestre 2026 A, periodo que abarca mi estancia del 6to semestre dentro del
doctorado en ciencias en matemáticas.

\begin{enumerate}

\item Leer el artículo \cite{Arrieta2024LocalicSeparation} para caracterizar los $p\in \pt A$ que producen $w_p$ ajustados.
\item Relacionar el ejemplo 12.1 de \cite{sexton2006point} y ver la relación que pueda existir con los marcos eficientes y los marcos $KC$.
\item Para $A=[0,1]$ describir quienes son $^pS$ y $PA$, donde $^pS$ es el espacio de parches.
\begin{itemize}
\item En el ejemplo anterior, verificar que propiedades de separación cumple $PA$.
\item Comprobar si $P[0,1]$ es eficiente pero no $KC$.
\end{itemize}
\item Si existe un ejemplo de marcos eficientes que no son $KC$, verificar que la clase de marcos $KC$ es cerrada bajo coproductos.
\item Caracterizar los marcos $KC$ por medio de los intervalos $[v_F, w_F]$.
\item Ver la relación que existe entre la propiedad $KC$ y $\mathbf{(H)}$.
\item Definir las nociones libres de puntos de apilado y fuertemente apilado.
\item Verificar si las nociones anteriores son conservativas o no.
\item En caso de que las nociones anteriores no sean conservativas, ver la relación que existe entre ellas y la propiedad $\mathbf{(H)}$.
\item Teniendo las nociones libres de puntos de apilado y fuertemente apilado, analizar el comportamiento de marcos que cumplan las propiedades:
\begin{itemize}
    \item eficiente $+$ fuertemente apilado.
    \item eficiente $+$ apilado.
    \item $\mathbf{(saju)} +$ fuertemente apilado.
    \item $\mathbf{(saju)} +$ apilado.
    \item $\mathbf{(aju)} +$ fuertemente apilado.
    \item $\mathbf{(aju)} +$ apilado.
\end{itemize}
\item Leer el artículo \cite{BalmerGallauer2025PatchDensity} y dar versiones del Lema 2.6 y Teorema 2.8 para $^pS$ y $PA$.
\end{enumerate}


\bibliographystyle{amsalpha}
\bibliography{research2}
% --------------------
% BIBLIOGRAFÍA (OPCIONAL)
% --------------------
% \begin{thebibliography}{9}
% \bibitem{ref1}
% Autor, \textit{Título del libro}, Editorial, Año.
% \end{thebibliography}

\end{document}
