\chapter{Marcos arreglados}\label{Parches}

Este capítulo muestra una recopilación del trabajo hecho por Sexton en su tesis doctoral (ver \cite{R.S.}). 
Algunas consecuencias espaciales también pueden ser encontradas en \cite{R.S.3}.\\

Parte de los objetivos de nuestra investigación es comprender la teoría que muestra Sexton sobre construcciones de parches 
y constrastarla con los axiomas de separación libres de puntos que se presentaran en el Capítulo \ref{Axiomas de separacion}.

\section{Filtros admisibles y núcleos ajustados}
Este será nuestro primer contacto con propiedades de separación traducidas al lenguaje de la teoría de marcos. El siguiente capítulo abardora de manera 
más profunda este tema.

\begin{dfn}
Consideremos $A$ un marco. Decimos que 
\begin{enumerate}
\item $A$ es \emph{regular} si para cada $a,b\in A$ con $a\nleq b$, existen $x,y\in A$ tales que 
\[
a\vee x=1, \quad y\nleq b, \quad x\wedge y=0
\]
se cumplen.
\item $A$ es \emph{ajustado} si para cada $a, b\in A$, con $a\nleq b$ existen $x, y\in A$ tales que
\[
a\vee x=1,\quad y\nleq b, \quad x\wedge y=0
\]
se cumplen.
\end{enumerate}
\end{dfn}

Como lo mencionamos antes, estas nociones vuelven a ser abordadas en el Capítulo \ref{Axiomas de separacion}. Aunque el enfoque proviene de diferentes autores, estas definiciones terminan siendo equivalentes a las que ahí se presentan. Por el momento, solo enunciaremos aquellas propiedades que no serán presentadas en el siguiente 
capítulo. Las pruebas pueden consultarse en \cite{R.S.}.

\begin{lem}
    Sea $A$ un marco ajustado. Los puntos (vistos como elementos $\wedge-$irreducibles) de $A$ son elementos máximos.
\end{lem}

\begin{lem}
    Si el marco $A$ es ajustado, entonces $\pt A$ es $T_1$ y sobrio.
\end{lem}

\begin{lem}
    En un espacio con topología ajustada, las tres condiciones son equivalentes, $T_0$, $T_1$ y sobrio.
\end{lem}

\begin{cor}
    Un espacio $T_0$ con topología ajustada es $T_1$ y sobrio.
\end{cor}

En el Capítulo \ref{Preliminares}, para un marco $A$, definimos diferentes tipos de filtros. De manera particular, existe un filtro especial que está en relación con los filtros abiertos. 

\begin{dfn}\label{Filtroadmisible}
    \begin{enumerate}
        \item Sea $A$ un marco. Para un elemento $a\in A$ y núcleo $j\in NA$ decimos que $j$ \emph{admite} al elemento $a$ si $j(a)=1$.

        \item Sea $\nabla (j)$ el conjunto de elementos admitidos por el núcleo $j$. $\nabla(j)$ es un filtro en $A$.

        \item Para un marco $A$, un filtro en $A$ es \emph{de admisibilidad} si tiene la forma $\nabla(j)$ para algún $j\in NA$.

        \item La relación $j\sim k$ si y solo si $\nabla(j)=\nabla(k)$ es una relación de equivalencia. A las clases de equivalencia las llamamos \emph{bloques}.

        \item Un núcleo es \emph{ajustado} si es el menor elemento de su bloque.
    \end{enumerate}
\end{dfn}

El Teorema \ref{TeoremaHM} proporciona una correspondencia 
biyectiva entre filtros abiertos y conjuntos compactos saturados. 
Los filtros de admisibilidad permiten extender esta correspondencia por medio de los núcleos ajustados. 
Para ver esto necesitamos primero unos cuantos resultados más.

\begin{lem}\label{Menorelemento}
    Sea $A$ un marco. Cada bloque de un núcleo tiene un menor elemento.
\end{lem}

\begin{proof}
    Sea $F$ un filtro admisible en $A$ y consideremos $B=\{j\in NA\mid \nabla(j)=F\}$. De esta manera $B$ es la colección de todos los núcleos que admiten exactamente al conjunto $F$. Recordemos que los ínfimos en $NA$ se calculan puntualmente. Así, sea $k=\bigwedge B$ y $k$ es el menor elemento de $B$.\\

    Sea $a\in F$, entonces por definición $j(a)=1$ para todo $j\in B$, en particular, $k(a)=1$. De modo que $a\in \nabla(k)$. Por lo tanto $F=\nabla(j)\subseteq \nabla(k)$. La otra inclusión se cumple debido a que $k\leq j$. Así $\nabla(k)=F$ y $k\in B$.
\end{proof}

\begin{lem}
    Cada filtro principal es admisible.
\end{lem}

\begin{proof}
    Consideremos el filtro principal $F=\{x\in A\mid x\geq a\}$ para algún $a\in A$. Notemos que para $j=v_a$, $\nabla(j)=F$, pues si $x\geq a$, $(a\succ x)=1$. 
\end{proof}

No todos los filtros son admisibles. Por ejemplo, supongamos que $A$ es booleano. Entonces cada núcleo $j$ tiene la forma $u_a$ para algún $a\in A$. De igual manera podría tener la forma $v_{a'}$ para algún $a'\in A$. Entonces cada filtro admisible $\nabla(j)$ es principal, pero cuando $A$ es infinito no hay filtros principales.\\

Aunque no todos los filtros son admisibles, existe una manera de obtener filtros de admisibilidad.
\begin{lem}\label{Lema5.5.4}
    Sea $A$ un marco. Para todo $F\in A^\wedge$, $F=\nabla(j)$ para algún $j\in NA$.
\end{lem}

\begin{proof}
    Sean $F\in A^{\wedge}$ y $f=\dot{\bigvee}\{v_a\mid a\in F\}$ de modo que para algún ordinal $\infty$ tenemos $Vv_F=f^\infty\in NA$, y este es el menor núcleo que admite a $F$. Así, $F\subseteq \nabla(f^\infty)$. Debemos probar que $\nabla(f^\infty)\subseteq F$. Comencemos por mostrar que si 
    \begin{equation}\label{H.induccion}
        f(x)\in F \Rightarrow x\in F,
    \end{equation}
    para cada $x\in A$.\\

    El supremo $f(x)=\bigvee\{v_a(x)\mid a\in F\}$ es dirigido y como $F\in A^\wedge$, si $f(x)\in F$, entonces se cumple que $v_a(x)\in F$ para algún $a\in F$, es decir, para $a\in F$, $v_a(x)\in F$. Luego
    \[
    x\geq a\wedge x=a\wedge (a\succ x)\in F.
    \]
    Por lo tanto $x\in F$ como queriamos. Ahora probamos por inducción sobre los ordinales que si $f^\alpha(x)\in F$, entonces $x\in F$ se cumple para cada ordinal $\alpha$.\\

    El caso $\alpha =0$ es trivial. El paso de inducción de $\alpha$ a $\alpha +1$ se sigue de \ref{H.induccion}, pues si suponemos que $f^\alpha (x)\in F$, entonces $x\in F$. De aquí que 
    \[
    f^{\alpha +1}(x)=f(f^\alpha(x))\in F\Rightarrow f^\alpha(x)\in F\Rightarrow x\in F.
    \]
    Resta el caso $\lambda$ un ordinal limite. Por definición, $f^\lambda (x)=\bigvee\{f^\alpha(x)\mid \alpha\leq \lambda\}$, el cual es un supremo dirigido y así
    \[
    f^\lambda(x)\in F\Rightarrow (\exists \alpha \leq \lambda )[f^\alpha (x)\in F]
    \]
    pues $F$ es abierto. Luego la hipótesis de inducción implica que $x\in F$. Por lo tanto $f^\infty (x)\in F$ si y solo si $x\in F$ para todo $x\in A$. En particular 
    \[
    f^\infty (x)=1\Rightarrow f^\infty (x)\in F\Rightarrow x\in F,
    \]
    es decir, $\nabla(f^\infty )\subseteq F$. 
\end{proof}

\begin{ej}
    En el marco $(\mathbb{N}, \leq)\cup \{\infty\}$ consideremos el filtro generado por el conjunto de los números pares. Notemos que este es un filtro principal y por lo tanto es admisible, pero no es un filtro abierto. Ya que 
    \[
    \infty=\bigvee \{\mbox{impares}\}\in\{\mbox{pares}\}
    \]
    pero $\{\mbox{impares}\}\cap \{\mbox{pares}\}=\emptyset$, es decir, no existe $y\in \{\mbox{impares}\}$ tal que $y\in \{\mbox{pares}\}$. Por lo tanto $\{\mbox{pares}\}$ no es filtro abierto.
\end{ej}

Sabemos que no todos los filtros son admisibles, pero cada filtro genera un menor filtro admisible por arriba de el.

\begin{dfn}\label{Definicion5.5.6}
    Sean $A$ un marco y $F$ un filtro en $A$. Definimos 
    \begin{equation}\label{Nucleoajustado}
    v_F=\bigvee\{v_a\mid a\in F\},
    \end{equation}
    donde el supremo es calculado en $NA$.
\end{dfn}

Por como construimos a $v_F$, éste admite cada $a\in F$, y así $F\subseteq \nabla(v_F)$. Se puede verificar que $\nabla(v_F)$ es el menor filtro admisible por encima de $F$. Además, $v_F$ es ajustado. De hecho, un núcleo es ajustado si tiene la forma de \ref{Nucleoajustado}.\\

Los núcleos ajustados se comportan de manera similar a los $v-$núcleos. El siguiente resultado es consecuencia de las propiedades de los $v-$núcleos.

\begin{lem}\label{Lema5.5.7}
    Sea $A$ un marco. Los siguientes resultados se cumplen para todos los filtros $F, G$ y familias dirigidas de filtros $\mathcal{F}$ en $A$.
    \begin{enumerate}[$i) $]
        \item $v_F\wedge v_G=v_{F\cap G}$.
        \item $v_F\vee v_G=v_{F\cup G}$.
        \item $\bigvee\{v_F\mid F\in \mathcal{F}\}=v_{\bigcup \mathcal{F}}$.
    \end{enumerate}
\end{lem}

Además de un menor elemento mínimo, algunos bloques también tienen un mayor elemento.

\begin{lem}\label{Lema5.5.8}
    Para cada $a\in A$ el núcleo $w_a$ es el mayor elemento de su bloque.
\end{lem}

\begin{proof}
    Supongamos que $j$ es un compañero de $w_a$. Basta con demostrar que $j(a)=a$, pues esto es equivalente a que $j\leq w_a$. Sean $x=j(a)$ y $y=(x\succ a)$, de aquí que 
    \[
    w_a(y)=((y\succ a)\succ a)=(((x\succ a)\succ a)\succ a)=(x\succ a)=y.
    \]
    Además
    \[
    ((y\vee x)\succ a)=(y\succ a)\wedge (x\succ a)=(y\succ a)\wedge y=y\wedge a=a
    \]
    Por lo tanto $((y\vee x)\succ a)=a$ y $1=((y\vee x)\succ a)\succ a)=w_A(y\vee x)$. Así $y\vee x\in \nabla(w_a)$ y por hipótesis $y\vee x\in \nabla (j)$, es decir, $j(y\vee x)=1$.\\
    Luego 
    \[
    j(y\vee a)=j(y\vee j(a))=j(y\vee x)=1,
    \]
    de aquí que $w_a(y\vee a)=1$. Pero $(x\succ a)=y=w_a(y)=w_a(y\vee a)=1$. Por lo tanto $j(a)=x\leq a$, es decir, $j(a)=a$.
\end{proof}

Cuando $j\in NA$ es ajustado, entonces el filtro de admisibilidad proporciona un poco más de información.

\begin{lem}\label{Lema5.5.9}
    Sea $A$ un marco. Supongamos que $j\in NA$ es ajustado. Entonces 
    \[
    j\leq k\Leftrightarrow \nabla(j)\subseteq \nabla(k)
    \]
    se cumple para todo $k\in NA$.
\end{lem}

\begin{proof}
    Consideremos $a\in \nabla(j)\subseteq \nabla(k)$ y $x\in A$. Sea $y=v_a(x)$, entonces $a\wedge y\leq x$. Así, 
    \[
    y\leq k(y)=k(a)\wedge k(y)=k(a\wedge y)\leq k(x).
    \]
    Lo cual muestra que $v_a\leq k$ y como $j$ es ajustado $j=\bigvee\{v_a\mid a\in \nabla(j)\}\leq k$.
\end{proof}

Una de las últimás propiedades que abordaremos aquí sobre los núcleos ajustados está en relación con los marcos que son ajustados.
\begin{thm}\label{Teorema5.5.10}
    Para cada marco $A$ las siguientes condiciones son equivalentes.
    \begin{enumerate}[$i) $]
        \item $A$ es ajustado.
        \item Cada núcleo en $A$ es ajustado.
        \item Cada $u-$núcleo en $A$ está solo en su bloque.
        \item Cada $u-$núcleo en $A$ es mínimo en su bloque.
    \end{enumerate}
\end{thm}

\begin{proof}
    \begin{description}
        \item[$i)\Rightarrow ii) $] Supongamos que $A$ es ajustado y supongamos que existen núcleos no ajustados, es decir, existen $j, k\in NA$ tales que $j\nleq k$. Entonces $j(c)\nleq k(c)$ para algún $c\in A$. Sean $a=j(c)$, $b=k(c)$ y al ser $A$ ajustado, podemos encontrar $x, y\in A$ tales que 
        \[
        a\vee x=1,\quad x\wedge y\leq b, \quad y\nleq b.
        \]
        Definimos $z=(y\succ b)$, de modo que $x\leq z$ y $c\leq b\leq z$. De aquí que $a=j(c)\leq j(z)$ y $x\leq z\leq j(z)$. Por lo tanto $1=a\vee x\leq j(z)$, lo cual implica que $k(z)=1$, pues $j$ y $k$ son compañeros.\\

        Como $y\wedge z\leq b$ tenemos que $k(y)\leq k(b)=k(k(c))=k(c)=b$, es decir, $y\leq b$ lo cual es una contradicción. Por lo tanto cada núcleo en $A$ es ajustado.

        \item[$ii)\Rightarrow iii)\Rightarrow iv) $] Si consideramos un $u-$núcleo, por $ii)$ este es ajustado. De aquí que $u_\bullet$ es el menor elemento de su bloque, es decir, para $j\in NA$, no se cumple que $j\leq u_\bullet$, pero para todo $j\in NA$
        \[
        j=\bigvee\{u_\bullet\wedge v_{j(\bullet)}\mid \bullet \in A\},
        \]
        de aquí que $j=u_\bullet$, es decir, $u_\bullet$ no tiene compañeros en su bloque. Al no tener compañeros en su bloque, $u_\bullet$ es el menor elemento del bloque.

        \item[$iv)\Rightarrow i) $] Supongamos $iv)$ y sean $A$ un marco y $a\nleq b\in A$, de aquí que $u_a\nleq w_b$, pues para $0\in A$, $u_a(0)=a$ y $w_b(0)=b$. Por hipótesis, $u_a$ es ajustado y por el Lema \ref{Lema5.5.9} se cumple que $\nabla(u_a)\nsubseteq \nabla(w_b)$. Entonces existe $x\in A$ tal que $a\vee x=1$ y $w_b(x)\neq 1$. Consideremos $y=(x\succ b)$, así $w_b(x)=(y\succ b)\neq 1$ lo que implica que $y\nleq b$ y $x\wedge y=x\wedge (x\succ b)=x\wedge b\leq b$.
    \end{description}
\end{proof}

En otras palabras, el resultado anterior nos dice que si $A$ es ajustado, se simplifica la estructura del bloque de admisibilidad. En el Capítulo \ref{Relación MA y AH} profundisaremos un poco más sobre esta cuestión.

\section{La extensión de Teorema de Hoffman-Mislove}

Una de las aportaciones más importantes de Sexton es que da el analogo a la construcción del espacio de parches. Para hacer esto, 
ella proporcina los objetos necesarios en marcos para construir una $pbase$. Por el Teorema \ref{TeoremaHM} tenemos la correspondencia 
entre filtros abiertos y conjuntos compactos saturados. Por otro lado, todo filtro abierto está asociado a un núcleo ajustado  
\[
v_F=\bigvee\{v_a\mid a\in F\}.
\]
y viceversa. Por lo tanto, podemos extender el Teorema de Hoffman-Mislove de la siguiente manera.
\begin{thm}\label{HM extendido}
Para $A\in \Frm$ y $S=\pt A$ existe una correspondencia biyectiva entre: 
\begin{enumerate}[$i) $]
    \item Filtros abiertos en $A$.
    \item Conjuntos compactos saturados en $S$.
    \item Núcleos ajustados en $NA$.
\end{enumerate}
\end{thm}

En esta sección exploramos las propiedades que cumplen los núcleos asociados a los filtros abiertos. Tembién, veremos la relación que 
existe entre todos estos objetos cuando interactuan entre si.

\begin{lem}
    Sea $A$ un marco. Entonces para todos los filtros abiertos $F, G$ y familias dirigidas de filtros abiertos $\mathcal{F}$ tenemos
    \begin{enumerate}[$i) $]
        \item $v_F\wedge v_G=v_{F\cap G}$,
        \item $\bigvee\{v_F\mid F\in \mathcal{F}\}=v_{\bigcup \mathcal{F}}$
    \end{enumerate}
    y $F\cap G$, $\bigcup \mathcal{F}$ son filtros abiertos
\end{lem}

\begin{proof}
    La prueba se sigue del Lema \ref{Lema5.5.7} y la Proposición \ref{CaracterizacionFabiertos}.
\end{proof}

Cada uno de los núcleos ajustados $v_F$ es el supremo sobre un conjunto dirigido. Tomando el supremo puntual
\[
f_F=\dot{\bigvee}\{v_a\mid a\in F\}
\]
donde podremos omitir el subíndice $F$ cuando el filtro en cuestión éste claro, e iterando a través de los ordinales obtenemos una sucesión 
\[
f^0=\id,\quad f^{\alpha+1}=f(f^\alpha), \quad f^\lambda=\bigvee\{f^\alpha\mid \alpha \leq \lambda\}
\]
para cada ordinal $\alpha$ y ordinal limite $\lambda$. Esta sucesión eventualmente se estabiliza en $f^\infty$ para algún ordinal $\infty$.\\

Nos concentraremos en la sucesión obtenida al aplicar cada derivada $f^\alpha$ al menor elemento de nuestro marco. Definimos
\[
d(0)=0,\quad d(\alpha +1)=f(d(\alpha)), \quad d(\lambda)=\bigvee\{f(\alpha)\mid \alpha\leq \lambda\}
\]
para cada ordinal $\alpha$ y ordinal limite $\lambda$.\\

Podemos hacer lo mismo en un contexto sensible a puntos. Sea $S$ un espacio topológico. Para un filtro abierto $F$ en $\mathcal{O}S$ tenemos 
\[
v_F=\bigvee\{v_a\mid a\in F\}=\bigvee\{[U']\mid Q\subseteq U\}
\]
donde $Q=\cap F$ es el conjunto compacto saturado correspondiente a $F$.\\

En lugar de considerar la sucesión de abiertos, resulta más sencillo el hacerlo en sus complementos. Para $Q\in \mathcal{Q}S$ usamos la operación $\hat{Q}$ en $\mathcal{C}S$ dada por 
\[
\hat{Q}(X)=\bigcap\{(X\cap U)^-\mid Q\subseteq U\}
\]
para cada $X\in \mathcal{C}S$. Establecemos
\[
Q(0)=S,\quad Q(\alpha +1)=\hat{Q}(Q(\alpha)), \quad Q(\lambda)=\bigcap\{Q(\alpha)\mid \alpha\leq \lambda\}
\]
para obtener una sucesión descendente de conjuntos cerrados. Por razones de cardinalidad, esta sucesión eventualmente se estabiliza en algún conjunto cerrado $Q(\alpha)$. Sabemos que $Q\subseteq Q(\infty)$ ya que cada conjunto cerrado $Q(\alpha)$ contiene a $Q$. 
Después se darán condiciones que hagan notar la diferencia entre $Q^-$ y $Q(\infty)$ y las consecuencias que esto tiene para un marco y lo que en la siguiente sección denominaremos como su \emph{marco de parches}.

\subsection{Estructura de bloques}\label{Estructura de bloques}

Si $A$ es un marco, podemos construir un nuevo marco formado por todos sus núcleos ($NA$). Por si mismo, $NA$ puede ser un marco difícil de estudiar, incluso si solo nos restringimos a los núcleos que producen el mismo filtro de admisibilidad.\\

En esta subsección veremos una primera construcción de lo que más adelante llamaremos \emph{el $Q$-cuadrado}.\\

Consideremos $A\in \Frm$, $S=\pt(A)$, $F\in A^\wedge$ y $Q$ el compacto saturado correspondiente para $F$. Así, por el Teorema \ref{HM extendido}
\[
F=\nabla(v_F)\Leftrightarrow a\in F\Leftrightarrow Q\subseteq U(a)
\]
donde $v_F\in NA$. Consideremos el cociente de $A$ dado por $v_F$, de decir, $A_F=A_{v_F}$. El marco $A_F$ tiene un espacio de puntos fácil de localizar.

\begin{lem}\label{Lema5.7.1}
    Sean $A$ un marco y $F$, $Q$ los conjuntos antes considerados. Entonces $Q=\pt(A_F)$.
\end{lem}

\begin{proof}
\begin{description}
    \item[$\Rightarrow) $] Recordemos que los puntos de $A_F$ son aquellos $p\in S$ tales que $v_F(p)=p$. Además, si $p\in F$ entonces $v_F(p)=1$ y por lo tanto $p\notin \pt(A_F)$. De esta manera si $p\in \pt(A_F)$, entonces $p\in S\setminus F=Q$, es decir, $\pt(A_F)\subseteq Q$.
    
    \item[$\Leftarrow) $] Consideremos cualquier $p\in Q$. Para cada $x\in F$ sea $y=(x\succ p)$ y así $y\wedge x\leq p$. Notemos que si $p\notin F$, entonces
    \[
    (x\succ p)\neq 1\Rightarrow x\nleq p
    \]
    y como $p\in S$ se debe cumplir que $y\leq p$. Como $y=(x\succ p)$ es arbitrario, se debe cumplir que
    \[
    f(p)=\bigvee\{v_x(p)\mid x\in F\}\leq p,
    \]
    y además $p\leq f(p)$. De aquí que $f_F(p)=p$, es decir, $V_F(p)=p$. Por lo tanto $p \in \pt(A_F)$. 
    
\end{description}
\end{proof}

Lo anterior nos proporciona el siguiente diagrama
\begin{equation}
    \begin{tikzcd}
	A & {A_F} \\
	{\mathcal{O}S} & {\mathcal{O}\pt(A_F)=\mathcal{O}Q}
	\arrow[from=1-1, to=1-2]
	\arrow[from=1-1, to=2-1]
	\arrow[from=1-2, to=2-2]
	\arrow[from=2-1, to=2-2]
\end{tikzcd}
\end{equation}
el cual será extendido y estudiado en el Capítulo \ref{Relación MA y AH}.\\

Para un subespacio $T\subseteq S$, tenemos el siguiente cociente
\[
A\to \mathcal{O}S\to \mathcal{O}T
\]
donde $a\mapsto \bigwedge \{p\in T\mid a\leq p\}$ es el kernel de dicho cociente. En particular, podemos hacer lo anterior para $Q\in \mathcal{Q}S$.\\

Por el Lema \ref{Lema3.4.1}, el conjunto $Q$ tiene un conjunto de generadores mínimos $M\subseteq Q$, es decir, el conjunto de elementos máximos de $A\setminus F$. Ahora, si extendemos el mismo razonamiento para $M$ visto como un subespacio de $Q$ tenemos
\begin{equation}\label{Cociente a M}
A\to A_F \to \mathcal{O}Q \to \mathcal{O}M
\end{equation}
con kernel dado por $w_F(a)=\bigwedge \{p\in m\mid a\leq o\}$, donde $a\in A$. 

\begin{lem}
    El núcleo $w_F$ es el núcleo máximo que admite a $F$. 
\end{lem}

\begin{proof}
    Sea $j\in NA$ tal que $\nabla (j)=F$. Cada punto $m\in M$ es fijado por $j$ ya que $m$ es un punto máximo y no está en $F$. Sabemos que $j(a)=1\Leftrightarrow a\in F$, en otras palabras 
    \[
    j(a)=1\Leftrightarrow (\forall \, m\in M)[j(a)\nleq m] 
    \]
    (ver Lema \ref{Lema3.4.1}). Como $j\sim w_F$, entonces $j(a)=(w_F)(a)=1$ para $a\in F$. Supongamos que $a\notin F$, entonces $a\leq m$ para algún $m\in M$ y $j(a)\leq j(m)=m$, de modo que 
    \[
    j(a)\leq \bigwedge \{p\in M\mid a\leq p\}=w_F(a).
    \]
    Por lo tanto $j\leq w_F$.
\end{proof}

De esta manera, si $F\in A^\wedge$, obtenemos bloques en $NA$ de la forma $[v_F, w_F]$ (los cuales serán llamados \emph{intervalos de admisibilidad} más adelante).\\

Notemos que $I_F=[v_F(0), w_F(0)]$ es un intervalo de $A$. Para cada $a\in I_F$ consideremos $j_a=(v_F\vee u_a)$ para producir un núcleo $v_F\leq j_a\leq w_F$. Además, $a\leq b$ si y solo si $j_a\leq j_b$ para cada $a, b\in I_F$. Esto nos da un encaje de marcos 
\[
\begin{split}
    I_F &\to [v_F, w_F]\\
    a & \mapsto j_a
\end{split}
\]
y por lo tanto, el intervalo $I_F$ da una idea de lo complejo que puede ser el intervalo $[v_F, w_F]$. Existen formas de asegurar que $v_F=w_F$, (algunas de ellas serán vista en el Capítulo \ref{Relación MA y AH}), en cuyo caso $I_F$ es solo un punto.\\

Supongamos que $A=\mathcal{O}S$ para algún espacio sobrio $S$. Si $Q\in \mathcal{Q}S$ tenemos el siguiente caso del cociente \ref{Cociente a M} 
\begin{equation}\label{Cociente espacial a M}
\mathcal{O}S\to (\mathcal{O}S)_F\to \mathcal{O}Q\to \mathcal{O}M
\end{equation}
que determinan al menor y al mayor elemento del intervalo ($v_F$ y $w_F$, respectivamente) y un elemento intermedio. En este caso tenemos que $w_F=[M']$ y, de manera similar, $[Q']$ es el elemento intermedio del intervalo. Así tenemos un intervalo 
\[
v_F\leq [Q']\leq w_F\in N\mathcal{O}S
\]
En general, $[Q']$ puede estar en cualquier extremo o en algún punto intermedio. La observación de que $v_F\leq [Q']$. El estudio de este intervalo espacial también se verá más adelante.\\

Si $S$ es $T_1$, entonces $Q=M$, pero esto no asegura que el intervalo sea simple. De esta manera, tenemos condiciones que rigen al cociente \ref{Cociente espacial a M}. La idea es hacer algo similar, pero para 
el caso general \ref{Cociente a M}.

\section{El marco de parches}\label{Marco de parche}

En esta sección veremos la construcción libre de puntos del espacio de parches $^pS$. Daremos el análogo de la pbase dada en la Sección \ref{Parche puntos}, pero para $A\in \Frm$. A esta construcción la llamaremos \emph{el marco de parches}. Dicho marco resultará ser un submarco del ensamble $NA$.\\

Recordemos que para un espacio $S$, el espacio de parches se construye a través de sus abiertos y sus conjuntos compactos saturados. Así 
\[
\mbox{pbase}=\{U\cap Q'\mid U\in \mathcal{O}S, Q\in \mathcal{Q}S\}
\]
Esta es la base para una nueva topología en $S$ y $\mathcal{O}^pS$ es el conjunto de uniones de todas las subfamilias de la pbase. Una construcción similar es la que nos permite obtener el marco de parches. \\

Consideremos $A\in \Frm$ un marco arbitrario y $NA$ su ensamble. Sabemos que dentro de $NA$ podemos considerar las familias
\[
\{u_a\mid a\in A\}\quad\mbox{ y }\quad\{v_F\mid F\in A^\wedge\}.
\]
La primera es una copia isomorfa de $A$ en $NA$, la cual corresponderia a $\mathcal{O}S$. Por el Teorema \ref{HM extendido}, la segunda es un análogo de $\mathcal{Q}S$.

\begin{dfn}\label{Definición7.1.1}
    Para $A\in\Frm$ definimos
    \[
    \mbox{Pbase}=\{u_a\wedge v_F\mid a\in A, F\in A^\wedge\}
    \]
    la cual es una familia $\wedge-$cerrada de elementos en $NA$.
\end{dfn}

Notemos que si consideramos $F=A$ y $a=1$, podemos probar que la Pbase contiene cada $u_a$, para cada $a\in A$, y cada $v_F$ para $F\in A^\wedge$.

\begin{dfn}\label{Definicion7.1.2}
    Para cada $A\in \Frm$, definimos el \emph{marco de parches}, denotado por $PA$ por el conjunto supremos de todas las subfamilias de la Pbase donde estos supremos son calculados en $NA$.
\end{dfn}

No es complicado verificar que, efectivamente, $PA$ es un marco. De esta manera obtenemos lo siguiente.

\begin{thm}\label{Teorema7.1.3}
    Para $A\in \Frm$, $PA$ es un submarco de $NA$, el cual incluye la imagen canónica de $A$.
\end{thm}

El resultado anterior nos proporciona el siguiente diagrama
\[
\begin{tikzcd}
A \arrow[r, "\iota"'] \arrow[rr, "\eta_A", bend left, shift left] & PA \arrow[r, "i"'] & NA
\end{tikzcd}
\]
donde $\iota$ es un encaje e $i$ es una inclusión. Está construcción nos lleva a cuestionarnos lo siguiente:
\begin{enumerate}[P1)]
    \item Dentro de la relación $A\to NA$, ¿dónde puede ocurrir $PA?$
    \item ¿Puede $A\to PA$ ser un isomorfismo de manera no trivial?
    \item ¿Puede ocurrir $PA=NA$ de una manera no trivial?
    \item ¿Cuál es la relación, si la hay, entre la construcción sin puntos y la construcción sensible a puntos?
    \item ¿Qué es $\pt (PA)$? ¿Coinciden con $^p\pt A$?
\end{enumerate}

El siguiente resultado da una manera de responder P2). Como veremos más adelante, existen otras manera de obtener $A\simeq PA$.

\begin{thm}\label{Teorema7.1.4}
    Para $A$ un marco regular y $j\in NA$ un núcleo tal que $\nabla(j)\in A^\wedge$. Se cumple que $j=u_d$, donde $d=j(0)$. 
\end{thm}

\begin{proof}
    Por hipótesis, $A$ es un marco regular, en consecuencia, $A$ es ajustado. De esta manera, cada bloque de admisibilidad está compuesto por un único elemento. Así, basta con probar que $j$ y $u_d$ tienen el mismo filtro de admisibilidad, y por lo tanto, concluir que $j=u_d$.\\

    Como $d=j(0)$, se cumple que $u_d\leq j$ y así $\nabla(u_d)\subseteq \nabla(j)$.\\
    
    Para la otra contención debemos probar que para $x\in A$ y $j(x)=1$, entonces $u_d(x)=d\vee x=1$. Por la regularidad se cumple que 
    \[
    x=\bigvee\{y\in A\mid (\exists z)[z\wedge y=0 \mbox{ y } z\vee x=1]\},
    \]
    además, este es un supremo dirigido. Al ser $\nabla(j)$ un filtro abierto se debe cumplir que $y'\in\nabla(j)$ para algún $y'\in \{y\in A\mid (\exists z)[z\wedge y=0 \mbox{ y } z\vee x=1]\}$., es decir,
    \[
    j(y')=1, \quad z\wedge y'=0, \quad z\vee x=1
    \]
    para algunos $y', z\in A$. De esta manera $d\vee x=j(z)\vee z\geq z\vee x=1$, es decir $\nabla(j)\subseteq\nabla(u_d)$.\\

    Por lo tanto $\nabla(j)=\nabla(u_d)$.
\end{proof}

Sabemos que $A\simeq NA$ ocurre cuando $A$ es booleano (lo cual provocaría que $A\simeq PA$, lamentablemente que $A$ sea booleano es una condición bastante fuerte). El Teorema \ref{Teorema7.1.4}, de manera indirecta nos dice que, bajo las hipótesis convenientes, la regularidad implica que $A\simeq PA$, pues solo nos restringimos a algunos $j\in NA$.

\subsection{$P( \_ )$ como funtor}

En la subsección \ref{P. funtoriales spuntos} se discuten las propiedades funtoriales de la construcción del espacio de parches. Ahí se menciona que una función continua $\phi\colon T\to S$ es parche continua si la imagen inversa $\phi^{-1}$ envía conjuntos compactos saturados $Q\in \mathcal{Q}S$ a conjuntos compactos saturados $\phi^{-1}(Q)\in \mathcal{Q}T$ (ver Definición \ref{Parchecontinua}). La restricción de estás imágenes inversas producen un morfismo de marcos $\phi^*$ entre las topologías y estos tienen adjunto derecho $\phi_*$, es decir,
\[\begin{tikzcd}
	{\mathcal{O}S} & {\mathcal{O}T}
	\arrow["{\phi^*}", shift left=2, from=1-1, to=1-2]
	\arrow["{\phi_*}", shift left=2, from=1-2, to=1-1]
\end{tikzcd}\]

Por el Lema \ref{Pcontinua y Scontinua}, al ser $\phi_*$ una función Scott-continua, tenemos que $\phi^*$ es parche continua. También, por la funtorialidad de $N$, tenemos que para cada morfismo de marcos $f\colon A\to B$, $Nf$ resulta ser un morfismo entre los ensambles. De esta manera obtenemos el siguiente diagrama.

\[\begin{tikzcd}
	A & PA & NA \\
	B & PB & NB
	\arrow[from=1-1, to=1-2]
	\arrow["f"', from=1-1, to=2-1]
	\arrow[from=1-2, to=1-3]
	\arrow["Nf", from=1-3, to=2-3]
	\arrow[from=2-1, to=2-2]
	\arrow[from=2-2, to=2-3]
\end{tikzcd}\]

El objetivo de esta subsección es obtener un morfismo $Pf\colon PA\to PB$ entre los marcos de parches.\\

Para un filtro $F$ en $A$, la imagen $f(F)$ no necesariamente es un filtro en $B$, sino es la clausura de la sección superior $f(F)=\uparrow f(F)$. Sin embargo, cunado $F\in A^\wedge$ no necesita serlo.\\

Recordemos que en espacios, una función continua no necesariamente respeta conjuntos compactos saturados y de manera, se pide una condición para lograr la funtorialidad. Algo similar ocurre para los morfismos de marcos, no necesariamente deben respetar filtros abiertos.

\begin{dfn}\label{Definicion7.2.2}
    Un morfismo de marcos $f\colon A\to B$ decimos que \emph{convierte filtros abiertos} si para cada $F\in A^\wedge$, la imagen $f(F)\in B^\wedge$.
\end{dfn}

La definición anterior proporciona la condición que queremos, pues recordemos que 
\[
N(u_a)=u_{f(a)},\quad N(v_a)=v_{f(a)},\quad N(v_F)=N(v_{f(F)})
\]
para cada $a\in A$ y $F\in A^\wedge$. En particular, si $f$ convierte filtros abiertos se cumple que $F\in A^\wedge$ implica $f(F)\in B^\wedge$, por definición. En consecuencia, si $j\in \mbox{Pbase}(A)$ implica que $Nf(j)\in \mbox{Pbase}B$.

\begin{lem}\label{Lema7.2.3}
    Sea $f\colon a\to B$ tal que convierte filtros abiertos. Entonces $j\in PA$ implica que $Nf(j)\in PB$ y por lo tanto, $P$ actúa funtorialmente sobre esta clase de flechas.
\end{lem}

En otras palabras, cuando $f$ convierte filtros abierto, podemos definir $Pf$ como la restricción de $Nf$ a $PA$. Esto nos da el diagrama conmutativo
\[\begin{tikzcd}
	A & PA & NA \\
	B & PB & NB
	\arrow[from=1-1, to=1-2]
	\arrow["f"', from=1-1, to=2-1]
	\arrow[from=1-2, to=1-3]
	\arrow["{Nf_{\mid PA}}", from=1-2, to=2-2]
	\arrow["Nf", from=1-3, to=2-3]
	\arrow[from=2-1, to=2-2]
	\arrow[from=2-2, to=2-3]
\end{tikzcd}\]
y así $P$ pasa a través de las composiciones.\\

Requerimos descubrir cual es la relación que existe entre ambas construcciones de parches. En el siguiente resultado se considera un morfismo de marcos $f\colon A\to B$ y su adjunto derecho $f_*$. También, consideramos una función continua $\phi\colon T\to S$ y el morfismo de marcos inducido por sus topologías $\phi^*\colon\mathcal{O}S\to \mathcal{O}T$ junto con su adjunto derecho.\\

\begin{thm}\label{Teorema7.2.4}
    \begin{enumerate}
        \item Para un morfimos de marcos como el de antes, si el adjunto derecho $f_*$ es Scott-continuo, entonces $f^*$ convierte filtros abiertos.
        \item Para una función continua $\phi$ como la de antes, el adjunto derecho $\phi_*$ es Scott-continuo si y solo si $\phi^*$ convierte filtros abiertos.
    \end{enumerate}
\end{thm}

\begin{proof}
    \begin{enumerate}
        \item Sabemos que $f^*(a)\leq b\Leftrightarrow a\leq f_*(b)$ para $a\in A$ y $b\in B$. Consideremos un filtro abierto $F\in A^\wedge$ y sea $Y\subseteq B$ un subconjunto dirigido con $\bigvee Y\in f^*(F)$. De esta manera $f^*(a)\leq \bigvee Y$ para algún $a\in F$. Luego $a\leq f_*(\bigvee Y)=\bigvee f_*(\bigvee Y)$, lo anterior se debe a que $f_*$ es Scott-continua.\\

        Por lo tanto, como $f_*(Y)$ es dirigido en $A$ se cumple que $a\leq f_*(y)$ para algún $y\in Y$. De esta manera $f^*(a)\leq y$ y $Y\cap f(F)\neq \emptyset$.

        \item Supongamos primero que $\phi_*$ es Scott-continua, de esta manera, por $a)$ se cumple que $\phi^*$ convierte filtros abiertos.\\

        Recíprocamente, supongamos que $\phi^*$ convierte filtros abiertos. Consideremos conjunto dirigido $\mathcal{W}\subseteq \mathcal{O}T$ y sea $V=\phi_*(\bigcup  \mathcal{W})$. Debemos mostrar $V\subseteq \bigcup \phi_*(\mathcal{W})$ para obtener la Scott-continuidad.\\

        Consideremos cualquier $s\in V$. El filtros de vecindades $F$ de $s$ está dado por $U\in F\Leftrightarrow s\in U$, donde $U\in \mathcal{O}S$. Este filtro es abierto y así $\phi(F)$ también lo es, pero en $\mathcal{O}T$. Luego
        \[
        W\in \phi(F)\Leftrightarrow \phi_*(W)\in F\Leftrightarrow s\in \phi_*(W)
        \]
        para cada $W\in \mathcal{O}T$. En particular, tenemos $\bigcup \mathcal{W}\in \phi(F)$ y por lo tanto $\exists W\in \mathcal{W}$ con $W\in \phi F$. Así $s\in \phi_*(W)\subseteq \bigcup \phi_(W)$ como requeríamos. 
    \end{enumerate}
\end{proof}

Consideremos una función continua como antes y supongamos que los espacios $S$ y $T$ son sobrios. Tenemos un morfismo de marcos asociado $\phi^*\mapsto \phi_*$ 
entre las topologías. Supongamos que el adjunto derecho $\phi_*$ es Scott-continuo. De esta manera $\phi$ convierte conjuntos compactos saturados, y por lo tanto 
es parche continuo. También, por el Teorema \ref{Teorema7.2.4}, el morfismo $\phi^*$ convierte filtros abiertos. Con esto se obtiene un par de morfismos de marcos 
\[
P(\phi^*)\colon P\mathcal{O}S\to P\mathcal{O}T\quad \mbox{ y } \quad \phi^*\colon \mathcal{O}^pS\to \mathcal{O}^pT
\]
que relacionan a ambas construcciones de parches.

\begin{thm}\label{Teorema7.2.5}
    Sea $A\in \Frm$ y $S=\pt A$. La reflexión espacial $U_A\colon A\to \mathcal{O}S$ convierte filtros abiertos, pero su adjunto derecho $U_A)_*$ no necesariamente es un morfismo continuo.
\end{thm}

\begin{proof}
    Sea $F\in A^\wedge$ y $\nabla U_A(F)$. Mostraremos que $\nabla\in \mathcal{O}S^\wedge$. Consideremos cualquier familia dirigida $\mathcal{U}$ en $\mathcal{O}S$ tal que $\bigcup\mathcal{U}\in \nabla$. Debemos verificar que $\mathcal{U}\cap \nabla\neq \emptyset$.\\

    Consideremos $X\subseteq A$ dado por $x\in X\Leftrightarrow U(x)\in \mathcal{U}$ y así, al ser $U( \_ )$ suprayectivo obtenemos $\mathcal{U}=\{U(x)\mid x\in X\}$ y $X$ indexa a $\mathcal{U}$ (posiblemente con alguna repetición). Vemos que $X\cap F\neq \emptyset$ y por lo tanto $\mathcal{U}\cap \nabla\neq \emptyset$.\\

    Sea $sp$ el kernel de $U( \_ )$, entonces
    \[
    y\leq sp(x)\Leftrightarrow U(y)\subseteq U(x)\quad \mbox{ y }\quad U(x)=U(sp(x))
    \]
    para todo $x, y \in A$. En particular $x\in X\Leftrightarrow sp(x)\in X$ para $x\in A$. Usando esto, verificamos primero que $X$ es dirigido. Sean $x, y\in A$, entonces $U(x), U(y)\in \mathcal{U}$ y por lo tanto, al ser $\mathcal{U}$ dirigida, tenemos que $U(x), U(y)\subseteq U(z)=U(sp(z)$ para algún $z\in X$. Así $sp(z)\in X$ y por la definición de $sp$ tenemos que si $x, y\leq sp(z)$, entonces $x\vee y\leq sp(z)$ produciendo la cota superior en $X$ requerida para concluir que $X$ es dirigido.\\

    Luego, sea $a=\bigvee X$, entonces 
    \[
    U(a)=\bigcup\{U(x)\mid x\in X\}=\bigcup \mathcal{U},
    \]
    de modo que $U(a)\in \nabla$, así $U(a)=U(b)$ para algún $b\in  F$.\\
    
    Ahora $sp(a)=s(b)\in F$ y como $U(sp(x))=U(x)$ tenemos que $sp(x)\in X\Leftrightarrow x\in X$. Así $\bigvee X=a\in F$ y al ser $F$ un filtro abierto se cumple que $x\in X\cap F$. Por lo tanto $U(x)\in \mathcal{U}\cap \nabla$.\\

   En \cite{R.S.} se puede consultar un ejemplo donde el adjunto derecho de $U_A$ no es continuo (Ejemplo 7.2.7).
\end{proof}

El Teorema \ref{Teorema7.2.5} tiene un lado positivo: nos permite llegar a $U_A$ con el funtor $P$ y así obtener un morfismo 
\[\begin{tikzcd}
	PA && {P\mathcal{O}S}
	\arrow["{P(U_A)}", from=1-1, to=1-3]
\end{tikzcd}\]
entre el marco de parches asociado.

\subsection{El diagrama completo del marco de parches}

La información recopilada hasta este momento nos permite construir el siguiente diagrama.
\[\begin{tikzcd}
	A & PA & NA \\
	{\mathcal{O}S} & {P\mathcal{O}S} & {N\mathcal{O}S} \\
	& {\mathcal{O}^PS} & {\mathcal{O}^fS}
	\arrow[from=1-1, to=1-2]
	\arrow["{U_A}"', from=1-1, to=2-1]
	\arrow[hook, from=1-2, to=1-3]
	\arrow["{P(U_A)}", from=1-2, to=2-2]
	\arrow["{N(U_A)}", from=1-3, to=2-3]
	\arrow[from=2-1, to=2-2]
	\arrow[hook, from=2-1, to=3-2]
	\arrow[hook, from=2-2, to=2-3]
	\arrow["{\sigma_S}", from=2-3, to=3-3]
	\arrow[hook, from=3-2, to=3-3]
\end{tikzcd}\]

El rectángulo superior de este diagrama se presenta en el Lema \ref{Lema7.2.3}. La sección inferior
\[
\mathcal{O}S\hookrightarrow \mathcal{O}^PS\hookrightarrow \mathcal{O}^fS
\]
es la relación que tiene una topología con las topologías de parches y Skulla. El morfismo $\sigma_S\colon N\mathcal{O}S\to \mathcal{O}^fS$ es el morfismo del espacio de puntos del ensamble a la topología de su espacio de puntos ($\pt N\mathcal{O}S=\mathcal{O}^fS$.\\

Para completar el diagrama y conseguir que todas las partes conmuten necesitamos los siguientes resultados.
\begin{lem}\label{Lema7.3.1}
    Sea $S$ un espacio con topología $\mathcal{O}S$. Para cada filtro $F$ en $\mathcal{O}S$ tenemos $Q$ con $F=\nabla(Q)$ y $\sigma(v_F)=Q'$ donde $Q$ es el correspondiente conjunto compacto saturado.
\end{lem}

\begin{proof}
    Sabemos que $v_F$ y $[Q']$ son compañeros bajo la relación de admisibilidad, es decir, admiten los mismos elementos. Al ser $v_F$ el mínimo elemento del bloque, se cumple que $v_F\leq [Q']$ y así $\sigma(v_F)\subseteq \sigma([Q'])=Q'$. pues $Q\in \mathcal{O}^fS$.\\

    Para verificar la otra contención consideremos $p\in Q'$. Notemos que $\overline{p}\subseteq Q'$ se cumple al ser $Q$ saturado, entonces $Q\subseteq \overline{p}'$ y $\overline{p}'\in F$. Por lo tanto $p\in v_F(\overline{p}')=1$ y así, por el Lema \ref{Lema6.3.3}, $p\in \sigma(v_F)$.
\end{proof}

\begin{lem}\label{Lema7.3.2}
    Sea $S$ un espacio con topología $\mathcal{O}S$. La restricción del morfismo de marcos $\sigma$ a $P\mathcal{O}S$ proporciona un morfismo de marcos $\pi\colon PA\to \mathcal{P}^PS$. Además, el morfismo es suprayectivo.
\end{lem}

\begin{proof}
    El marco $P\mathcal{O}S$ es generado por núcleos de la forma $[U]\wedge v_F$ donde $U\in \mathcal{O}S$ y $F\in \mathcal{O}S^\wedge$. Tenemos que $\sigma([U])=U$, pues $U$ es abierto, y $F=\nabla(Q)$ para algún $Q\in \mathcal{Q}S$ de modo que $\sigma(v_F)=Q'$ por el Lema \ref{Lema7.3.1}. Por lo tanto $\sigma([U]\wedge v_F)=U\cap Q'$ y estos conjuntos forman una base para $\mathcal{O}^PS$.
\end{proof}

Lo anterior nos proporciona el diagrama completo del marco de parches.
\[\begin{tikzcd}
	A & PA & NA \\
	{\mathcal{O}S} & {P\mathcal{O}S} & {N\mathcal{O}S} \\
	& {\mathcal{O}^PS} & {\mathcal{O}^fS}
	\arrow[from=1-1, to=1-2]
	\arrow["{U_A}"', from=1-1, to=2-1]
	\arrow[hook, from=1-2, to=1-3]
	\arrow["{P(U_A)}", from=1-2, to=2-2]
	\arrow["{N(U_A)}", from=1-3, to=2-3]
	\arrow[from=2-1, to=2-2]
	\arrow[hook, from=2-1, to=3-2]
	\arrow[hook, from=2-2, to=2-3]
	\arrow["{\pi_S}", from=2-2, to=3-2]
	\arrow["{\sigma_S}", from=2-3, to=3-3]
	\arrow[hook, from=3-2, to=3-3]
\end{tikzcd}\]

El morfismo de marco $\pi$ no necesariamente debe ser un isomorfismo. Notemos que si $S$ es empaquetado, entonces $^pS=P$ así las composición
\[\begin{tikzcd}
	{\mathcal{O}S} & {P\mathcal{O}S}
	\arrow[shift left=2, from=1-1, to=1-2]
	\arrow[shift left=2, from=1-2, to=1-1]
\end{tikzcd}\]
dan la identidad en $\mathcal{O}S$, pero $\pi$ no necesita ser un isomorfismo.

\section{Jerarquía de propiedades de separación}\label{Marcos arreglados}

Como hemos visto en este capítulo, existen dos construcciones que tratan de imitar una propiedad similar a la que cumplen los espacios $T_2$, la construcción del espacio de parches y la del marco de parches (una sensible a puntos y la otra libre de puntos). La primera sirve para caracterizar a los espacios empaquetados. La segunda, como mostraremos más adelante, caracteriza a los espacios cuya topología cumple ser ``arreglada''. De manera similar a los espacios empaquetados, los marcos arreglados produce una propiedad de separación entre los axiomas $T_1$ y $T_2$.

\subsection{Marco parche trivial}
Cuando se traslada una noción sensible a puntos a su variante libre de puntos, lo que se busca es que la nueva nueva 
noción tenga un comportamiento parecido al de su variante espacial. En este caso, las propiedades que se busca imitar son las que 
satisface el espacio de parches, de manera especifica 
\[
\mbox{Empaquetado} \Leftrightarrow S\,=\,^pS.
\]

\begin{dfn}\label{Parche trivial}
    Para $A\in \Frm$ decimos que este es \emph{parche trivial} si el encaje $\iota\colon A\to PA$ es un isomorfismo.
\end{dfn}
De esta manera la propiedad libre de puntos sería 
\[
\mbox{Parche trivial}\Leftrightarrow A\simeq PA.
\]
La prueba del Teorema \ref{Teorema7.1.4} da una idea de como, con algunas condiciones particulares, algunos $j\in NA$ cumplen que $j=u_d$ donde $d=j(0)$. 
Para restringirnos únicamente al marco $PA$ debemos observar que para algún $u\in A$ y cualquier $F\in A^\wedge$ se cumple que $u_d=v_F$. Lo anterior daría 
una condición necesaria y suficiente para obtener la trivialidad del parche.\\

Veamos como se comporta la trivialidad del parche con algunas propiedades espaciales.
\begin{lem}\label{Lema8.1.4}
    Consideremos $S\in \Top$ si:
    \begin{enumerate}[a)]
        \item $S$ es $T_2$ o
        \item $S$ es $T_0$, ajustado y empaquetado (en otras palabras, $T_1$ y sobrio),
    \end{enumerate}
    entonces $\mathcal{O}S$ es parche trivial.
\end{lem}

\begin{proof}
    \begin{enumerate}[a)]
        \item Se verá más adelante (Teorema \ref{Teorema8.4.4}).
        \item Consideremos cualquier $v_F$ para $F\in A^\wedge$. Por el Teorema \ref{TeoremaHM}, $F$ está determinado por algún $Q\in \mathcal{Q}S$. Además, también sabemos que los núcleos $v_F$ y $[Q']$ producen el mismo filtro de admisibilidad y al ser ajustado, se cumple que $v_F=[Q']$. Luego, al ser $T_1$, se cumple que $Q=M$, es decir $[Q']=[M']=w_F$, por lo tanto, el intervalo de admisibilidad colapsa.
    \end{enumerate}
\end{proof}

La parte $b)$ de la prueba anterior nos da un criterio un poco distinto para verificar la trivialidad del parche (cuando el marco en cuestión es la topología de un espacio). En este caso, basto verificar que los intervalos de admisibilidad son solo un punto, siempre que $F\in \mathcal{O}S^\wedge$. \\

Ambos ejemplos proporcionan una condición necesaria, pero no suficiente. Existen espacios $T_2$ y espacios $T_1+$sobrios que no son parche trivial.

\subsection{Marcos arreglados}

Si $F\in A^\wedge$, podemos asignarle un núcleo $v_F=\bigvee \{v_a\mid a\in F\}$ y, a manera de notación, consideramos sel supremo puntual $f=\dot{\bigvee}\{v_a\mid a\in F\}$ el cual nos permite construir una sucesión 
\[
d(0)=\id, \quad d(\alpha +1)=f(d(\alpha)),\quad d(\lambda)=\bigvee\{d(\alpha)\mid \alpha <\lambda\}
\]
para cada ordinal $\alpha$ y ordinal límite $\lambda$. Además, verificamos que esta se estabiliza en algún elemento $d=d(\infty)=v_F(0)$ para algún ordinal $\infty$.\\

Al principio de esta sección mencionamos que una condición que asegura la trivialidad del parche es que $u_d=v_F$ para algún $d\in A$ y $F\in A^\wedge$.\\

En general, se cumple que $u_d\leq v_F$. Por lo tanto, solo ocupamos ver la otra desigualdad y esta ocurre siempre que para $x\in F$, $u_d(x)=1$. 

\begin{dfn}\label{Definición8.2.1}
    Sean $A\in \Frm$ y $\alpha$ un ordinal. 
    \begin{enumerate}
        \item Un filtro abierto $F\in A^\wedge$ es \emph{$\alpha$-arreglado} si 
    \[
    x\in F\Rightarrow u_{d(\alpha)}(x)=d(\alpha)\vee x=1
    \]
    donde $d(\alpha)=f^\alpha(0)$.
    
    \item El marco $A$ es $\alpha$-arreglado si para todo $F\in A^\wedge$, $F$ es $\alpha-$arreglado.

    \item El marco $A$ es arreglado si es $\alpha$-arreglado para algún ordinal $\alpha$.
    \end{enumerate}
\end{dfn}

Notemos que si el marco $A$ es arreglado, su grado de arreglo es el menor ordinal $\alpha$ para el cual se cumple que $A$ es $\alpha$-arreglado.

\begin{lem}\label{Lema8.2.2}
    Un marco es arreglado si y solo si es parche trivial.
\end{lem}

\begin{proof}
    \begin{description}
        \item[$\Rightarrow )$] Consideremos $A\in \Frm$ y supongamos que $A$ es arreglado. De esta manera para $F\in A^\wedge$ se cumple que si $x\in F\Rightarrow d(\alpha)\vee x=1$, es decir, $F\subseteq \nabla(u_{d(\alpha)}$, en particular, para el núcleo $v_F$ asociado se cumple que $v_F\leq u_d$. Por lo tanto $v_F=u_d$, es decir, $A$ es parche trivial.

        \item[$\Leftarrow )$] Supongamos que $A\cong PA$ y consideremos $F\in A^\wedge$ arbitrario. Por la trivialidad del parche se cumple que $v_F=u_d$ para algún $d\in A$. Notemos que lo anterior obliga que para $x\in F$, siempre se debe de cumplir que $d\vee x=1$, pero esto solo ocurre cuando 
        \[
        \begin{split}
        d=\bigvee \{\neg x\mid x\in F\}&=\bigvee\{(x\succ 0)\mid x\in F\}\\
        &=\dot{\bigvee}\{v_x(0)\mid x\in F\}\\
        &=v_F(0)=d(\infty)
        \end{split}
        \]
        para algún ordinal $\infty$. Por lo tanto como $F=\nabla(v_F)=\nabla(u_d)$ se debe cumplir que si $x\in F$, entonces $d\vee x=1$, es decir, $A$ es arreglado (pues $F$ es arbitrario). 
    \end{description}
\end{proof}

Notemos que si $F\in A^\wedge$ es $\alpha$-arreglado, entonces $F$ es $\beta$-arreglado para ordinales $\beta\leq \alpha$, pues $\nabla(f(d(\beta)))\subseteq \nabla(f(d(\alpha)))$. De esta manera se obtiene una jerarquía de propiedades
\[
\cdots \Rightarrow \alpha\mbox{-arreglado}\Rightarrow (\alpha-1)\mbox{-arreglado}\Rightarrow \cdots 1\mbox{-arreglado}\Rightarrow 0\mbox{-arreglado}
\]
En el Teorema \textbf{Referenciar el Teorema 11.3.7 una vez que ya este escrito} se mostrará que cada una de estas es distinta.\\

De manera sencilla podemos observar que
\[
A \mbox{ es } 0\mbox{-arreglado }\Leftrightarrow A=\{*\}.
\]
Daremos un poco más de información para cuando $A$ es $1-$arreglado.\\

Queremos usar la condición de arreglo para el caso en que $A=\mathcal{O}S$.

\begin{lem}\label{Lema8.2.3}
    Consideremos $S$ un espacio sobrio, $Q\in \mathcal{Q}S$ y $F=\nabla(Q)$ el filtro abierto correspondiente a $Q$ en $\mathcal{O}S$. Para cada ordinal $\alpha$, el filtro $F$ es $\alpha-$arreglado si y solo si $Q(\alpha)=Q$.
\end{lem}

\begin{proof}
    Por hipótesis $F=\nabla (Q)$ es $\alpha$-arreglado y por definición esto ocurre si 
    \[
    U\in F\Rightarrow (Q(\alpha))'\cup U=S
    \]
    pues $d(\alpha)=(Q(\alpha))'$. Así, si $Q\subseteq U$, entonces $Q(\alpha)\subseteq U$. Por lo tanto $Q(\alpha)\subseteq Q$ y $Q\subseteq Q(\alpha)$, es decir, $Q(\alpha)=Q$ como queríamos.
\end{proof}

\subsection{La jerarquía de la regularidad}

La subsección anterior proporciona una jerarquía de propiedades por medio del grado de arreglo. Como veremos ahora, podemos establecer una jerarquía similar, pero en este caso, relacionadas con la regularidad.

\begin{dfn}\label{Definición8.3.1}
    Consideremos $A\in \Frm$ y $\alpha$ un ordinal. Decimos que $A$ es:
    \begin{enumerate}[a)]
        \item débilmente $\alpha$-regular si para cada $a, b\in A$ y $F\in A^\wedge$ con $a\nleq b$ y $a\in F$ existen $x, y\in A$ tales que 
        \[
        a\vee x=1,\quad y\leq a,\quad y\nleq b\quad \mbox{ y }\quad x\wedge y\leq d(\alpha)
        \]
        se cumple.
        \item $\alpha$-regular si para cada$a,b\in A$ existe $y\in A$ tal que para cada $F\in A^\wedge$ con $a\in F$ existe un elemento $x\in A$ tal que 
        \[
        a\vee x=1,\quad y\leq a,\quad y\nleq b\quad \mbox{ y }\quad x\wedge y\leq d(\alpha)
        \]
        se cumple.
    \end{enumerate}
\end{dfn}

Observemos que $a)$ y $b)$ de la definición anterior está en el orden de los cuantificadores. Para el caso de $a)$ tenemos 
\[
(\forall a)\; (\forall F)\; (\exists y)\; (\exists x)
\]
y para $b)$ se cumple que 
\[
(\forall a)\; (\exists y)\; (\forall F)\; (\exists x).
\]
En otras palabras, débilmente $\alpha$-regular requiere cierta uniformidad en la elección de $y$.\\

Al principio, débilmente $\alpha$-regular parece la más obvia y, de hecho, fue la primera que demostraron. Sin embargo, la segunda se relaciona mejor con algunas de las otras propiedades que implican la regularidad. En particular, podemos dar una versión $\alpha$-indexada para decir que un elemento $a$ está bastante por debajo de $b$. Recordemos que podemos caracterizar a los marcos regulares por medio de sus elementos bastante por debajo.

\begin{dfn}\label{Definicion8.3.2}
    Para $a, y\in A$, decimos que $y$ está \emph{bastante $\alpha$-por debajo de} $a$ (denotado por ``$y\eqslantless_\alpha a$''), si para cada $F\in A^\wedge$ tal que $a\in F$ existe $x\in A$ tal que 
    \[
    a\vee x=1,\quad y\leq a,\quad  \mbox{ y }\quad x\wedge y\leq d(\alpha)
    \]
    se cumple.
\end{dfn}

Observemos que $\eqslantless_0$ es simplemente la definición de $\eqslantless$. De las definiciones de $\alpha$-regular y bastante $\alpha$-por debajo vemos que $A$ es $\alpha$-regular exactamente cuando para cada $a\nleq b$ existe algún $y\eqslantless_\alpha a$ y $y\nleq b$ se cumple.

\begin{lem}\label{Lema8.3.3}
    Un marco $A$ es $\alpha$-regular si y solo si para todo $a\in A$
    \[
    a=\bigvee\{b\in A\mid b\eqslantless_\alpha a\}.
    \]
\end{lem}

\begin{proof}
    Supongamos que $A$ es $\alpha$-regular. Consideremos $a\in A$ y $b=\bigvee\{y\in A\mid y\eqslantless_\alpha a\}$, entonces $b\leq a$. Si $a\nleq b$, por la definición de la $\alpha$-regularidad $y\eqslantless_\alpha a$ y $y\nleq b$ para algún $y\in A$, lo cual es una contradicción (pues $y\leq b)$. Por lo tanto $a\leq b$.\\

    Recíprocamente, supongamos que $a=\bigvee\{y\in A\mid y\eqslantless_\alpha a\}$ para todo $a\in A$. Así, si $a\nleq b$, entonces existe $y\in A$ tal que $y\eqslantless_\alpha a$ y $y\nleq b$ lo cual implica la $\alpha$-regularidad.
\end{proof}

La siguiente propiedad de la $\alpha$-regularidad surgen directamente de la definición.

\begin{lem}\label{Lema8.3.4}
    Para cada $A\in \Frm$ y ordinales $\alpha\leq \beta$, las siguiente implicaciones se cumplen:
    \begin{enumerate}
        \item $\alpha$-regular $\Rightarrow$ débilmente $\alpha$-regular.
        \item $\alpha$-regular $\Rightarrow$ $\beta$-regular.
        \item débilmente $\alpha$-regular $\Rightarrow$ débilmente $\beta$-regular.
        \item $0$-regular $\Leftrightarrow$ regular.
    \end{enumerate}
\end{lem}

De esta manera tenemos una jerarquía en tres propiedades. Además, podemos relacionarlas.

\begin{thm}\label{Teorema8.3.5}
    Para cada $A\in \Frm$ y ordinal $\alpha$ las siguientes implicaciones se cumplen
    \[
    \alpha\mbox{-arreglado}\Rightarrow \alpha\mbox{-regular}\Rightarrow \mbox{débilmente }\alpha\mbox{-regular}\Rightarrow (\alpha-1)\mbox{-regular}.
    \]
\end{thm}

\begin{proof}
    Supongamos que $A$ es $\alpha$-arreglado y consideremos $a, b\in A$ con $a\nleq b$. Sea $y=a$ y supongamos que para $F\in A^\wedge$, $a\in F$. Al ser $A$ $\alpha$-arreglado, se cumple que $d(\alpha)\vee a=1$. Tomemos $x=(a\succ d(\alpha))$, entonces 
    \[
    x\wedge y=x\wedge a\leq d(\alpha), \quad a\vee x\geq a\vee d(\alpha)=1\quad \mbox{ y }\quad y\leq a
    \]
    con $y\nleq b$ y así obtener que $A$ es $\alpha$-regular.\\

    La segunda implicación se cumple por la definición de ambas propiedades.\\

    Por último,  supongamos que $A$ es débilmente $\alpha$-regular. Sean $F\in A^\wedge$ y $a\in F$. Por hipótesis
    \[
    a=\bigvee\{y\in A\mid y\leq a \mbox{ y }a\vee (y\succ d(\alpha))=1\}
    \]
    donde este supremo resulta ser dirigido. Como $a\in F$, se debe cumplir que existe $y\in F$ tal que $y\leq a$ y $a\vee (y\succ d(\alpha))=1$. Luego 
    \[
    (y\succ d(\alpha))=v_y(d(\alpha))\leq f(d(\alpha))=d(\alpha+1).
    \]
    De esta manera $1=a\vee  (y\succ d(\alpha))\leq a\vee  d(\alpha+1)$. Lo cual implica que $a\vee d(\alpha+1)=1$ y por lo tanto $A$ es $(\alpha+1)$-arreglado.
\end{proof}

La parte $c)$ del Lema \ref{Lema8.3.4} nos proporciona un caso particular del Teorema \ref{Teorema8.3.5}, pues regular$=0$-regular, y este implica $1-$arreglado. Notemos que lo anterior es el Teorema \ref{Teorema7.1.4}.

\subsection{Topologías arregladas}

Teniendo en cuenta que la topología de un espacio es un marco, resulta natural el preguntarnos, ¿cuál es el comportamiento del grado de arreglo con respecto a las propiedades clásicas de separación?

\begin{lem}\label{Lema8.4.1}
    Si el marco $A$ es arreglado, entonces cada punto de $A$ es máximo. Además, $\pt A$ es un espacio $T_1$.
\end{lem}

\begin{proof}
    Consideremos $S=\pt A$. Sea $p\in S$ y $P$ el filtro completamente primo correspondiente a $p$, es decir, 
    \[
    y\in P\Leftrightarrow y\nleq p
    \]
    para $y\in A$. Recordemos que si $P$ es completamente primo, entonces este es abierto y primo, de esta manera consideramos $v_P$ el núcleo asociado a $P$. Luego $d=v_P(0)\leq w_p(0)=p$ y, sin perdida de generalidad, tomemos $a\in A$ tal que $p< a$. Como $a\nleq p$ entonces $a\in P$ y al ser $A$ arreglado, se cumple que 
    \[
    a=a\vee p\leq a\vee d=1,
    \]
    es decir, $a\vee p=1$ y al ser $a$ arbitrario, se debe cumplir que $p$ es máximo.
\end{proof}

El resultado anterior se puede extender a espacios $T_0$ generales. Cada espacio $T_0$ es un subespacio de su reflexión sobria $^+S$ y ambos espacios tienen topologías isomorfas. Si la topología $\mathcal{O}S$ es arreglada, entonces, por el Lema anterior, su espacio de puntos de $^+S$ es $T_1$. Sabemos que si $S$ es un espacio con reflexión sobria que es $T_1$, entonces el espacio $S$ es $T_1$ y sobrio.

\begin{lem}\label{Lema8.4.2}
    Si un espacio $T_0$ tiene una topología arreglada, entonces el espacio original es $T_1$ y sobrio.
\end{lem}

Un espacio $T_0$ es $T_3$ precisamente cuando este es $0$-regular. ¿Qué pasa con el siguiente nivel de la jerarquía que $0$-regular implica? Los siguientes resultados responden lo anterior. 

\begin{lem}\label{Lema8.4.3}
    Si un marco $A$ es $1$-arreglado entonces su espacio de puntos $S$ es $T_2$.
\end{lem}

\begin{proof}
    Consideremos $p\in S$ y su correspondiente filtro completamente primo $P$. Notemos que 
    \[
    d(1)=\bigvee\{v_x(0)\mid x\in P\}=\bigvee\{\neg x\mid x\nleq p\}
    \]
    y al ser $A$ $1$-arreglado, si $a\nleq p$, entonces $a\vee d(1)=1$ para $a\in A$. Consideremos cualquier punto $q\neq p$, necesitamos encontrar vecindades abiertas disjuntas de $p$ y $q$. Por el Lema \ref{Lema8.4.1} $p$ y $q$ son máximos, entonces se debe cumplir que $q\nleq p$ en $A$ y así $q\in P$. Si $q\in P$ entonces $q\vee d(1)=1$ y por la maximalidad de $q$ se debe cumplir que $d(1)\nleq q$. De esta manera, existe $x\nleq p$ con $y=\neg x\nleq q$ y por lo tanto tenemos que 
    \[
    p\in U_x,\quad q\in U_y,\quad U_x\cap U_y=U_0=\emptyset,
    \]
    es decir, $S$ es un espacio $T_2$, pues obtuvimos una separación de abiertos para $p$ y $q$.
\end{proof}

\begin{thm}\label{Teorema8.4.4}
    Un espacio $S$ que es $T_0$ tiene topología $1$-arreglada si y solo si $S$ es $T_2$.
\end{thm}

\begin{proof}
    Cada espacio $T_0$ es un subespacio de su reflexión sobria. Si tal espacio tiene topología $1$-arreglada, entonces por el Lema \ref{Lema8.4.3} es un subespacio de un espacio $T_2$ y por lo tanto, $S$ es $T_2$ en si mismo.\\

    Recíprocamente, supongamos que $S$ es $T_2$ y consideremos $F\in \mathcal{O}S^\wedge$. Al ser $S$ $T_2$ este es un espacio sobrio. Sea $F=\nabla(Q)$ para el respectivo $Q\in \mathcal{Q}S$ y así $U\in F\Leftrightarrow Q\subseteq U$ para $U\mathcal{O}S$. Notemos que 
    \[
    Q(1)=\hat{Q}(Q(0))=\hat{Q}(S)=\bigcap\{\overline{(S\cap U)}\mid Q\subseteq U\}=\bigcap\{\overline{U}\mid Q\subseteq U\}
    \]
    y $Q\subseteq Q(1)$. Por el Lema \ref{Lema8.2.3},  $F$ es $1$-arreglado si $Q(1)=Q$, y al ser $F$ arbitrario, tendríamos que $\mathcal{O}S$ es$1$-arreglado. Por lo tanto, debemos verificar que $Q(1)\subseteq Q$.\\

    Supongamos que $p\notin Q$, entonces existen $U, V\in \mathcal{O}S$ tales que $p\in U$, $Q\subseteq V$ y $U\cap V=\emptyset$. De esta manera $p\notin \overline{V}$, y además, por la forma de $Q(1)$ se cumple que $p\notin Q(1)$, es decir, $Q(1)\subseteq Q$.
\end{proof}

Tenemos dos resultados que relacionan los espacios $T_2$ con la condición de arreglo. Uno de los objetivos que buscamos con estas notas es explorar de mejor manera la relación que existe entre los marcos arreglados y los diferentes axiomas de separación libre de puntos.\\

A manera de resumen, si $S$ es al menos un espacio $T_0$, se tienen las siguientes caracterizaciones.

\begin{enumerate}
    \item $\mathcal{O}S$ es $0$-arreglado $\Leftrightarrow S=\emptyset$.
    \item $\mathcal{O}S$ es $0$-regular $\Leftrightarrow S$ es $T_3$.
    \item $\mathcal{O}S$ es $4$-arreglado $\Leftrightarrow S$ es $T_2$.
    \item $\mathcal{O}S$ es $1$-regular $\Leftrightarrow$ ??
    \item $\mathcal{O}S$ es arreglado $\Leftrightarrow S$ es empaquetado y \textbf{\emph{apilado}}. 
\end{enumerate}

\subsection{Espacios apilados}

Las nociones empaquetado y arreglado, hasta cierto punto, podrían parecer similares. Sin embargo, la caracterización $5)$ mencionada antes nos da la sospecha de que no es así. Recordemos parte de la información que tenemos sobre estas.

\begin{itemize}
    \item Un marco $A$ es parche trivial si y solo si este es arreglado. De esta manera, podemos asociar a un marco un grado de arreglo.

    \item Un espacio $S$ es empaqueta justamente cuando $S=\,^pS$.
\end{itemize}

\begin{lem}\label{Lema8.5.1}
    Sean $A\in \Frm$ y $S=\pt A$. Si $A$ es arreglado, entonces $S$ es empaquetado.
\end{lem}

\begin{proof}
    Consideremos $Q\in\mathcal{Q}S$ y sea $F=\nabla(Q)$ el filtro abierto correspondiente. Por el Teorema \ref{Teorema6.4.1} tenemos que $\Sigma_A=\sigma_{\mathcal{O}S}\circ NU_A$, donde 
    \[
    \Sigma_A\colon NA\to \mathcal{O}^fS,\quad\sigma_{\mathcal{O}S}\colon N\mathcal{O}S\to \mathcal{O}^fS\quad\mbox{ y }\quad NU_A\colon NA\to N\mathcal{O}S.
    \]
    Luego,
    \[
    \Sigma(v_F)=(\sigma_{\mathcal{O}S}\circ NU_A)(v_F)=\sigma_{\mathcal{O}S}(NU_A(v_F))=\sigma_{\mathcal{O}S}(v_{U(F)})=\sigma_{\mathcal{O}S}(v_F).
    \]
    Por el Lema \ref{Lema7.3.1} tenemos que $\sigma_{\mathcal{O}S}(v_F)=Q'$. Por hipótesis, $A$ es arreglado, así este es parche trivial, es decir, $v_F=u_d$ para algún $d\in A$. De aquí que
    \[
    Q'=\Sigma(v_F)=\Sigma(u_d)=U(d),
    \]
    pues $\Sigma$ es la reflexión espacial de $\mathcal{O}S$, es decir, $Q'\in \mathcal{O}S$. Por lo tanto, $Q\in \mathcal{C}S$ y al ser $Q$ arbitrario, se cumple que todo conjunto compacto saturado es cerrado, es decir, $S$ es empaquetado.
\end{proof}

Como un caso particular de lo anterior, si un espacio sobrio tiene una topología arreglada.

\begin{lem}\label{Lema8.5.2}
    Si el espacio $S$ es sobrio y empaquetado, entonces el encaje canónico del marco de parches
    \[\begin{tikzcd}
	{\mathcal{O}S} & {P\mathcal{O}S}
	\arrow[shift left=2, from=1-1, to=1-2]
	\arrow[shift left=2, from=1-2, to=1-1]
\end{tikzcd}\]
se divide, es decir, tiene un inverso unilateral donde la composición en $\mathcal{O}S$ es la identidad.
\end{lem}

El Lema \ref{Lema8.5.1} muestra que si la topología de un espacio sobrio $S$ es arreglada, entonces $S$ es empaquetado. Sin embargo, existen ejemplos que muestran que sobriedad y empaquetado no son suficientes para obtener la condición de arreglo. Para que esto suceda necesitamos algo más de información.

\begin{dfn}\label{Definicion8.5.3}
    Sean $S$ un espacio y $Q\in \mathcal{Q}S$. Decimos que un conjunto cerrado $X\in \mathcal{C}S$ es \emph{$Q$-irreducible} (denotado por $Q\ltimes X$), si 
    \[
    Q\subseteq U\Rightarrow X\subseteq \overline{(X\cap U}
    \]
    se cumple para cada $U\in \mathcal{O}S$.
\end{dfn}

¿Qué tiene que ver esta noción de $Q$-irreductibilidad con la noción estándar de irreductibilidad? Para cada punto $x$ de un espacio, la saturación $\uparrow x$ es compacto. Nos fijamos en $(\uparrow x)$-irreductibilidad.

\begin{lem}\label{Lema8.5.4}
    Sean $S$ un espacio y $X\in \mathcal{C}S$ con $X\neq \emptyset$. Entonces $X$ es irreducible exactamente cuando $x\in X$  implica $(\uparrow x)\ltimes X$ para cada $x\in S$.
\end{lem}

\begin{proof}
    Supongamos primero que $X$ es irreducible y consideremos cualesquiera $x\in X$ y $U\in \mathcal{O}S$ tal que $x\in U$. Sea $V=\overline{(X\cap U)}'$, entonces debemos probar que $X\subseteq V'$, es decir, $X\cap V=\emptyset$.\\

    Por contradicción, supongamos que $X\cap V\neq \emptyset$. Sabemos que $X\cap U\neq \emptyset$ y por la irreductibilidad se cumpliría que $X\cap U\cap V\neq \emptyset$, pero $X\cap U\cap V\subseteq V\cap V'=\emptyset$ lo cual es una contradicción.\\

    Recíprocamente, supongamos $x\in X\Rightarrow (\uparrow x)\ltimes X$ para cada $x\in S$. Consideremos que $U, V\in \mathcal{O}S$, $x\in X\cap U$ y $y\in X\cap V$. Debemos probar que $X\cap U\cap V\neq \emptyset$. Por hipótesis, $(\uparrow x)\ltimes X$, es decir, $X=\overline{(X\cap U)}$ para cada $U\in \mathcal{O}S$. De esta manera 
    \[
    \overline{(X\cap U)}=X=\overline{(X\cap V)},
    \]
    y en particular $y\in \overline{(X\cap U)}$. Pero $y\in V\in \mathcal{O}S$ y por lo tanto $X\cap U\cap V\neq \emptyset$ como requeríamos. 
\end{proof}

Por medio de la relación $\ltimes$ podemos dar las siguientes definiciones.

\begin{dfn}\label{Definicion8.5.5}
    \begin{enumerate}[a)]
        \item Un espacio $S$ es \emph{apilado} si $Q\ltimes X \Rightarrow X\subseteq \overline{Q}$ se cumple para cada $Q\in \mathcal{Q}S$ y $X\in \mathcal{C}S$.
        
        \item Un espacio $S$ es fuertemente apilado si $Q\ltimes X \Rightarrow X\subseteq \overline{(X\cap Q)}$ se cumple para cada $Q\in \mathcal{Q}S$ y $X\in \mathcal{C}S$. 
    \end{enumerate}
\end{dfn}

Como mencionamos antes, la noción de apilamiento se puede relacionar con el grado de arreglo. Antes de ver eso, observemos la relación con las propiedades espaciales.

\begin{lem}\label{Lema8.5.6}
    \begin{enumerate}[a)]
        \item Cada espacio $T_2$ es fuertemente apilado.
        \item Cada espacio fuertemente apilado es apilado.
        \item Cada espacio $T_1$ y sobrio es apilado.
    \end{enumerate}
\end{lem}

\begin{proof}
    \begin{enumerate}[a)]
        \item Consideremos un espacio $S$ que es $T_2$ y supongamos que $Q\ltimes X$ para $Q\in \mathcal{Q}S$ y $X\in \mathcal{C}S$. Es suficiente mostrar que $X\subseteq Q$. \\
        
        Por contradicción, supongamos que $X\nsubseteq Q$. De esta manera existe $p\in X\setminus Q$ y al ser $S$ un espacio $T_2$, existen $U, V\in \mathcal{O}S$ tales que $Q\subseteq U$, $p\in V$ y $U\cap V=\emptyset$. Al cumplirse $Q\ltimes X$, entonces $X\subseteq \overline{(X\cap U)}\subseteq \overline{U}\subseteq V'$ y por lo tanto $p\in V\cap X\subseteq V\cap V'=\emptyset$, lo cual es una contradicción.

        \item Se da por definición.

        \item Consideremos un espacio $S$ tal que es $T_1$ y apilado. Consideremos cualquier $X\in \mathcal{C}S$ irreducible y $x\in X$. Mostraremos que $X=\{x\}$.\\

        Por el Lema \ref{Lema8.5.4} tenemos que $(\uparrow x)\ltimes X$.  Como $S$ es $T_1$, entonces $(\uparrow x)=\{x\}=\overline{x}$ y al ser $S$ apilado se cumple $x\in X\subseteq \overline{(\uparrow x)}=\{x\}$. Por lo tanto, $\{x\}=X$.
    \end{enumerate}
\end{proof}

El parte $a)$ del Lema \ref{Lema8.5.6} nos abre el panorama con una de las propiedades de separación más importantes, pero la mayoría de los espacios que nos interesan son fuertemente apilados, pero no $T_2$. La ventaja es que existen muchos otros espacios fuertemente apilados que cumplen otras condiciones.

\begin{lem}\label{Lema8.5.7}
    Cada topología de Alexandroff es fuertemente apilada.
\end{lem}

\begin{proof}
    Sean $Q\in \mathcal{Q}S$ y $X\in \mathcal{C}S$ tales que $Q\ltimes X$. Por definición tenemos que $Q\subseteq U\Rightarrow X\subseteq \overline{(X\cap U)}$, donde $U\in \mathcal{O}S$. Si $S$ es un espacio de Alexandroff, entonces todo conjunto saturado es abierto, es decir, $Q\in \mathcal{O}S$. Por lo tanto si $Q\ltimes X \Rightarrow X\subseteq \overline{(X\cap Q)}$ y es lo que queríamos probar.
\end{proof}

Podríamos no tener claro cual es la función de la relación $\ltimes$. El Lema \ref{Lema8.5.6} únicamente menciona un comportamiento sensible a puntos. Su verdadero propósito se vuelve claro cuando vemos la situación en un enfoque sin puntos.\\

Cada $Q\in\mathcal{Q}S$ produce un $F\in \mathcal{O}S^\wedge$ y a su vez, este produce una derivada $f$ y un núcleo $v_F=f^\infty$ en $\mathcal{O}S$. De igual manera tenemos un núcleo espacialmente inducido $[Q']$ en $\mathcal{O}S$. Además, $v_F\leq [Q']$ (pues $F$ es admitido por ambos núcleos). \\

Sabemos que $f(W)=\bigcup\{(u\succ W)^\circ\mid q\subseteq U\}$ para cada $W\in \mathcal{O}S$. Así, para cada $X\in\mathcal{C}S$ tenemos
\[
\begin{split}
f(X')&=\bigcup\{(U'\cup X')^\circ\mid Q\subseteq U\}\\
&=\bigcup\{\overline{(X\cap U)}'\mid Q\subseteq U\}\\
& =\left(\bigcap\{\overline{(X\cap U)}\mid Q\subseteq U\}\right)'\\
&=(\hat{Q}(X))'
\end{split}
\]

\begin{lem}\label{Lema8.5.8}
    Para cada espacio $S$ y $Q\in \mathcal{Q}S$ tenemos
    \[
    Q\ltimes X\Leftrightarrow \hat{Q}(X)=X\Leftrightarrow v_F(X')=X'
    \]
    para cada $X\in \mathcal{C}S$.
\end{lem}

\begin{lem}\label{Lema8.5.9}
    Para cada espacio $S$ y $Q\in \mathcal{Q}S$ tenemos
    \begin{enumerate}[a)]
    \begin{multicols}{3}
        \item $\overline{Q}\subseteq Q(\infty)$,
        \item $Q\ltimes Q(\infty)$,
        \item  $Q\ltimes X\Rightarrow X\subseteq Q(\infty)$
    \end{multicols}
    \end{enumerate}
    para cada $X\in \mathcal{C}S$.
\end{lem}

\begin{proof}
    \begin{enumerate}[a)]
        \item Tenemos que por la construcción de $\hat{Q}(\alpha)$, $Q\subseteq Q(\infty)$. Luego, como $Q(\infty)$ es cerrado, por lo tanto $\overline{Q}\subseteq Q(\infty)$.

        \item Por definición $Q(\infty)=\hat{Q}(Q(\infty))=\bigcap\{\overline{(Q(\infty)\cap U)}\mid Q\subseteq U\}$. de esta manera, si $Q\subseteq U$, entonces $Q(\infty)\subseteq \overline{(Q(\infty)\cap U)}$, es decir, $Q\ltimes Q(\infty)$.

        \item Por construcción $Q(\infty)$ es el mayor conjunto $Y$ con $\hat{Q}(Y)=Y$
        \[
        \hat{Q}(Y)=\bigcap\{\overline{(Y\cap U)}\mid Q\subseteq U\}=Y,
        \]
        si $Q\ltimes X$, entonces por definición, $Q\subseteq U$ implica que $X\subseteq \overline{(X\cap U)}\subseteq \hat{Q}(\infty)$.
    \end{enumerate}
\end{proof}

De esta manera $Q\subseteq \overline{Q}\subseteq Q(\infty)$ para cada $Q\in \mathcal{Q}S$ Por definición, la inclusión de la izquierda es una igualdad precisamente cuando el espacio $S$ es empaquetado, ¿cuándo es una igualdad la contención de la derecha?

\begin{cor}\label{Corolario8.5.10}
    Un espacio $S$ es apilado precisamente cuando $\overline{Q}=Q(\infty)$ para cada $Q\in \mathcal{Q}S$.
\end{cor}

\begin{proof}
    Supongamos que el espacio $S$ es apilado. Por $b)$ del Lema \ref{Lema8.5.9}, tenemos que para cada espacio $S$ y $Q\in \mathcal{Q}S$ se cumple que $Q\ltimes Q(\infty)$ y al ser $S$ apilado implica que $Q(\infty)\subseteq \overline{Q}$. La otra contención es la parte $a)$ del Lema \ref{Lema8.5.9}, Por lo tanto $\overline{Q}=(\infty)$.\\

    Recíprocamente, si $\overline{Q}=Q(\infty)$, entonces por $c)$ del Lema \ref{Lema8.5.9}, se cumple que $Q\ltimes X\Rightarrow X\subseteq Q(\infty)=\overline{Q}$ y esta es la definición de apilado.
\end{proof}

Los espacios apilados van un paso más allá.

\begin{lem}\label{Lema8.5.11} 
    Para cada espacio $S$ las siguientes afirmaciones son equivalentes.
    \begin{enumerate}
        \item $S$ es fuertemente apilado.
        \item Para cada $F\in \mathcal{O}S^\wedge$ tenemos que $v_F=[Q']$ donde $F$ es el filtro de vecindades de $Q\in \mathcal{Q}S$.
        \item Para cada $F\in \mathcal{O}S^\wedge$ el núcleo $v_F$ es espacialmente inducido.
    \end{enumerate}
\end{lem}

\begin{proof}
\begin{description}
    \item[$1)\Rightarrow 2)$] Supongamos que $S$ es fuertemente apilado. Consideremos $Q\in \mathcal{Q}S$ y $F\in\mathcal{Q}S$ el respectivo filtro abierto asociado a $Q$. Sabemos que $v_F\leq [Q']$, entonces debemos verificar la otra desigualdad. Por el Lema \ref{Lema8.5.8} se cumple que para cada $X\in \mathcal{C}S$
    \[
    v_F(X')=X'\Rightarrow Q\ltimes X\Rightarrow X\subseteq \overline{(X\cap Q)}\Rightarrow \overline{(X\cap Q)}'\subseteq X'\Rightarrow [Q'](X')=X',
    \]  
    es decir, cualquier abierto fijado por $v_F$ es fijado por $[Q']$ y así $[Q']\leq v_F$.

    \item[$2)\Rightarrow 3)$] Si $v_F=[Q']$ en consecuencia $v_F$ es espacialmente inducido.

    \item[$3)\Rightarrow 1)$] Supongamos que $v_F=[E']$ para algún $E\subseteq S$. Sabemos que 
    \[
    v_F=\bigcup\{[U']\mid Q\subset U\},
    \]
    donde $Q$ es el compacto saturado correspondiente a $F$ y $Q=\bigcap F$. De aquí que  $Q\subseteq U \Leftrightarrow E\subseteq U$ para $U\in \mathcal{O}S$. Al ser $Q$ saturado
    \[
    Q=\bigcap \{U\mid E\subseteq U\}\quad\mbox{ y }\quad E\subseteq Q\Leftrightarrow Q'\subseteq E'\Leftrightarrow [Q']\leq [E']=v_F.
    \]

    Supongamos ahora que $Q\ltimes X$ para $X\in \mathcal{C}S$, entonces $X\subseteq \overline{(X\cap U)}$, pues $Q\subseteq U$. Luego, 
    \[
    \begin{split}
    [Q'](X')\leq [E'](X')&\Leftrightarrow (Q'\cup X')^\circ \subseteq (E'\cup X')^\circ\\
    &\Leftrightarrow \overline{(Q\cap X)}'\subseteq \overline{(E\cap X)}'\subseteq X',
    \end{split}
    \]
    es decir, $X\subseteq \overline{(Q\cap X}$. Por lo tanto $S$ de fuertemente apilado.
\end{description}
\end{proof}

Con esto podemos juntar varios resultados para obtener una caracterización de arreglo que es sensible a puntos.

\begin{thm}\label{Teorema8.5.12}
    Un espacio $S$ que es $T_0$ tiene topología arreglada si y solo si $S$ es empaquetado y apilado.
\end{thm}

\begin{proof}
    \begin{description}
        \item[$\Rightarrow )$] Supongamos que $\mathcal{O}S$ es arreglado y consideremos $Q\in \mathcal{Q}S$. Sabemos que $Q\subseteq \overline{Q}\subseteq Q(\infty)$ y así, es suficiente demostrar que $Q\supseteq Q(\infty)$, de esta manera obtendríamos que $Q=\overline{Q}$ (todo conjunto compacto saturado es cerrado) y $\overline{Q}=Q(\infty)$ (por el Corolario \ref{Corolario8.5.10}).\\

        Sea $F$ el filtro en $\mathcal{O}S$ inducido por $Q$. Como $\mathcal{O}S$ es arreglado, se cumple que $v_F=[D]$ para algún $D\in \mathcal{O}S$. Supongamos que para $X\in \mathcal{C}S$, $Q\ltimes X$. De aquí que $Q(\infty)=\hat{Q}(Q(\infty))$ y por el Lema \ref{Lema8.5.8} $Q(\infty)=D'$, o equivalentemente $D=Q(\infty)'$. Luego
        \[
        U\in F\Leftrightarrow Q\subseteq U\Rightarrow D\cup U=S 
        \]
        (por definición de arreglado). Además, si $\mathcal{O}S$ es arreglado, también es parche trivial, es decir, $v_F=u_D$. Entonces 
        $v_F(U)=u_D(U)=D\cup U=S$, y como $D=Q(\infty)'$ tenemos que $Q(\infty)\subseteq U$, es decir, se cumplen las siguientes equivalencias
        \[
        Q\subseteq U\Leftrightarrow v_F(U)=S\Leftrightarrow D\cup U=S \Leftrightarrow Q(\infty)\subseteq U,
        \]
        como $Q$ es saturado, $Q=\bigcap\{U\mid Q(\infty)\subseteq U\}$, es decir, $Q(\infty)\subseteq Q$, que es lo que queríamos.

        \item[$\Leftarrow )$]  Recíprocamente, supongamos que $S$ es empaquetado y apilado y consideremos cualquier $F\in \mathcal{O}S^\wedge$. Si $S$ es empaquetado, entonces $S$ es $T_1$ y así, por el Lema \ref{Lema8.5.6}, $S$ es sobrio.\\

        Sabemos que $F$ es inducido por algún $Q\in \mathcal{Q}S$. En general, entonces $[Q(\infty)']\leq v_F\leq [Q']$, pero en un espacio empaquetado y apilado se cumple que $Q(\infty)=Q\in \mathcal{C}S$, por el Lema \ref{Lema8.2.3}. Luego $v_F=[D]$ para algún $D\in \mathcal{O}S$ y así $\mathcal{O}S$ es arreglado.
    \end{description}
\end{proof}

\section{El espacio de puntos del marco de parches}

Para un marco $A$ con espacio de puntos $S$ podemos construir dos marcos de parches diferentes. Estos son la topología del espacio de puntos del marco de parches ($\mathcal{O}\pt(PA)$) y el marco de parches de la topología del espacio de puntos ($P\mathcal{O}S=P(\mathcal{O}\pt A)$). Es momento de ver si existe relación entre estos.\\

Sabemos que la relación que existe entre el marco $A$, $PA$ y $NA$ nos proporciona el siguiente diagrama conmutativo
\[\begin{tikzcd}
	A & PA & NA \\
	{\mathcal{O}S} & {P\mathcal{O}S} & {N\mathcal{O}S} \\
	& {\mathcal{O}^pS} & {\mathcal{O}^fS}
	\arrow[from=1-1, to=1-2]
	\arrow["{U_A}"', from=1-1, to=2-1]
	\arrow[from=1-2, to=1-3]
	\arrow["{PU_A}", from=1-2, to=2-2]
	\arrow["{NU_A}", from=1-3, to=2-3]
	\arrow[from=2-1, to=2-2]
	\arrow[from=2-1, to=3-2]
	\arrow[from=2-2, to=2-3]
	\arrow["{\pi_S}", from=2-2, to=3-2]
	\arrow["{\sigma_S}", from=2-3, to=3-3]
	\arrow[from=3-2, to=3-3]
\end{tikzcd}\]

para el cual sabemos que
\begin{itemize}
    \item Cada flecha horizontal es un encaje y tres de estas son inclusiones.
    \item La flecha reflexión espacial ($U_A$) es suprayectiva.
    \item La propiedades funtoriales de $N$ aseguran que tanto $NU_A$ como $PU_A$ son suprayectivas.
    \item La flecha $\sigma_S$ es suprayectiva. Además, el espacio $^fS$ es el espacio de puntos tanto de $N\mathcal{O}S$ y $NA$, donde $\sigma_S$ y la composición $\Sigma_S=\sigma_S\circ NU_A$ son las respectivas reflexiones espaciales.
    \item La flecha $\pi_S$ es suprayectiva, pues para cada $Q\in \mathcal{Q}S$ tenemos que $\pi(v_F)=Q'$ donde $F$ es el filtro abierto en $\mathcal{O}S$ generado por $Q$.
\end{itemize}

Esta información genera las siguientes preguntas:
\begin{itemize}
    \item ¿Qué es el espacio de puntos $\pt (PA)$ del marco de parches de $A$?
    \item En particular, ¿qué es el espacio de puntos de $P\mathcal{O}S$?
    \item ¿Son diferentes estos espacios de puntos?
\end{itemize}



Para el espacio $S$ el espacio de Skula $^fS$ es el espacio de puntos de $N\mathcal{O}S$ y de $NA$. Nuestra intuición nos podría llevar a pensar que el espacio de parches $^pS$ es el espacio de puntos de $P\mathcal{O}S$ o de $PA$ o de ambos. Esto se aclarara más adelante con algunos ejemplos.\\

Existen casos donde $^pS$ es el espacio de puntos $P\mathcal{O}S$. Por ejemplo, si $S$ es $T_2$, entonces $^pS=S$ y $\mathcal{O}S\to P\mathcal{O}S$ es un isomorfismo. Sin embargo, en general tenemos que buscar un poco más para encontrar el espacio de puntos.\\

Otra pregunta que podríamos hacernos es ¿$P\mathcal{O}S$ es siempre espacial?
La respuesta a esto no es afirmativa, pues existe una colección de ejemplos que lo contradice (estos ejemplos también se presentaran después).

\begin{lem}\label{Lema9.2.3}
    Sea $S$ un espacio sobrio. Si el encaje $\pi\colon P\mathcal{O}S\to \mathcal{O}^pS$ es un isomorfismo, entonces $S$ es fuertemente apilado
\end{lem}

\begin{proof}
    Consideremos $Q\in \mathcal{Q}S$. Por el Lema \ref{Lema8.5.11} es suficiente verificar que $v_F=[Q']$ donde $F\in \mathcal{O}S^\wedge$ es el filtro correspondiente a $Q$. Por el Lema \ref{Lema7.3.1} tenemos que $\pi(v_F)=Q'=\pi([Q'])$ y, por hipótesis, $\pi$ es inyectiva, es decir, $v_F=[Q']$.
\end{proof}

El recíproco no es cierto, pues es posible tener un espacio fuertemente apilado donde $\pi$ no es un isomorfismo.

\subsection{Los puntos ``ordinarios'' del marco de parches}

Notemos que la composición $PA\to P\mathcal{O}S\to \mathcal{O}^pS$ proporciona un morfismo de marcos suprayectivo y, a su vez, este indica que existe alguna conexión entre $^pS$ y el espacio de puntos $\pt(PA)$. En particular, existe una función continua $^pS\to \pt(\mathcal{O}^pS)\to \pt (PA)$, donde el espacio de en medio es la reflexión sobria de $^pS$. Lo que haremos ahora será obtener una descripción explicita de esta función y se mostrará que $^pS$ es un subespacio de $\pt (PA)$.\\

Recordemos que para $p\in A$, $p\in \pt A$ si y solo si $p$ es un elemento $\wedge$-irreducible. En particular, en $PA$ sus puntos son los núcleos de parches que, como elementos de $PA$ son $\wedge$-irreducibles. Además, cuando consideramos al ensamble $NA$, $\pt (NA)=\{w_p\mid p\in \pt A\}$ y al ser $\pt$ un funtor contravariante, se cumple que $\pt(NA)\to \pt(PA)$ es una inclusión, es decir, si $w_P\in \pt(NA)$, entonces $w_p\in \pt(PA)$.\\

Consideremos $p\in \pt A$, entonces 
\[
w_p(x)= \left\{ \begin{array}{lcc} 1 & \mbox{ si } & x \nleq p \\ \\ p & \mbox{ si } & x \leq p \end{array} \right.
\]
para $x\in A$. Sea $P=\nabla(w_p)=\{x\in A\mid x\nleq p\}$ el filtro completamente primo asociado a $p$ y al mismo tiempo, el filtro de admisibilidad de $w_p$. Al ser este un filtro abierto, $w_p$ es el mayor elemento de su bloque, ¿quién es el menor elemento $v_p$? Para responder lo anterior usamos la derivada $f_p=f_P=\dot{\bigvee}\{v_y\mid y\in P\}$. Así,
\[
f_p(x)= \left\{ \begin{array}{lcc} 1 & \mbox{ si } & x \nleq p \\ \\ \leq p & \mbox{ si } & x \leq p \end{array} \right.
\]
para $x\in A$. Después veremos que $f_p(0)=0\neq p$ puede ocurrir.

\begin{lem}\label{Lema9.3.1}
    En la situación anterior se cumple que $w_p=u_p\vee v_P=f_p\circ u_p$ y $w_p\in \pt (PA)$.
\end{lem}

\begin{proof}
    Observemos que $f_p\circ u_p\leq v_P\circ u_p=u_p\vee v_P\leq w_p$ y así, por la descripción de $f_p$ tenemos
    \[
    (f_p\circ u_p)(x)=f_p(p\vee x)=\left\{ \begin{array}{lcc} 1 & \mbox{ si } & x \nleq p \\ \\ p & \mbox{ si } & x \leq p \end{array} \right.=w_p(x)
    \]
    para $x\in A$.\\

    Sabemos que $u_p$ y $v_P$ pertenecen a $PA$, entonces $w_p\in PA$. Además, $w_p\in \pt(NA)$ de aquí que $w_p\in \pt(PA)$.
\end{proof}

El resultado anterior proporciona un encaje $\alpha\colon S\to \pt (PA)$, en donde a cada $p\in \pt A$ le corresponde un $w_p\in \pt (PA)$ y así se impone una topología en el conjunto $S$ usando la topología dada en $\pt (PA)$. Para describir la topología impuesta, se usan los conjuntos abiertos subbásicos canónicos $U_{PA}(u_a)$ y $U_{PA}(v_F)$ de $\pt (PA)$.\\

Aquí $a$ es un elemento arbitrario de $A$ y $F$ es un filtro abierto arbitrario. Recordemos que $Q=S\setminus F$ está en $\mathcal{Q}S$ y $F$ está determinado por 
\[
x\in F\Leftrightarrow Q\subseteq U_A(x), 
\]
donde $x\in A$.

\begin{lem}\label{Lema9.3.2}
    Para la situación anterior tenemos 
    \[
    w_p\in U_{PA}(u_a)\Leftrightarrow p\in U_A(a)\quad \mbox{ y }\quad w_p\in U_{PA}(v_F)\Leftrightarrow p \in Q'
    \]
    para cada $a\in A$, $F\in A^\wedge$ y $p\in S$.
\end{lem}

\begin{proof}
    Por definición tenemos que 
    \[
    w_p\in U_{PA}(u_a)\Leftrightarrow u_a\nleq w_p\Leftrightarrow u_a(p)\neq p\Leftrightarrow a\vee p\neq p\Leftrightarrow a\nleq p\Leftrightarrow p\in U_A(a)
    \]

    Para la otra equivalencia, recordemos que $v_F$ es un núcleo ajustado. Así, por definición tenemos
    \[
    w_p\in U_{PA}(v_F)\Leftrightarrow v_F\nleq w_p\Leftrightarrow F\nsubseteq \nabla (w_p)\Leftrightarrow p\in F\Leftrightarrow p\in Q'
    \]
\end{proof}

Este resultado muestra que $\alpha$ es una función continua cuando $S$ lleva la topología de parches.

\begin{thm}\label{Teorema9.3.3}
    Sea $A\in \Frm$ y $S=\pt A$. El encaje $^pS\to \pt PA$ exhibe a $^pS$ como un subespacio de $\pt (PA)$.
\end{thm}

Este resultado localiza la que se espera que sea gran parte de $\pt (PA)$.

\begin{dfn}\label{Definicion9.3.4}
    Un punto de $PA$ que no es de la forma $w_p$ para algún $p\in \pt A$ es un punto salvaje.
\end{dfn}

Dado que $\pt(PA)$ es sobrio, pero $^pS$ no necesita serlo, entonces deben existir puntos salvajes para algunos marcos $A$. Más adelante se muestran como son algunos de estos.\\

\textbf{Pregunta:} ¿Es $\pt(PA)$ solo la reflexión sobria de $^pS$?\\

Ciertamente la reflexión sobria de $^pS$ debe estar dentro de $\pt(PA)$. Sabemos que $^{+p}S$ es solo la clausura frontal de $^pS$ en $\pt(PA)$. El problema está en si es todo $\pt(PA)$. No se ha podido responder toda esta pregunta. Por el momento tiende a la opinión de que la respuesta es si.

\subsection{Los puntos salvajes del ensamble de parches}

Comenzamos con un ejemplo de punto salvaje.

\begin{ej}\label{Ejemplo9.4.1}
    El ensamble de parches de la reflexión sobria de la topología cofinita contiene un punto salvaje. Ver subsección \ref{Ejemplos Rosy}.
\end{ej}

Cada punto salvaje se adjunta a uno de los puntos $w_p$ de forma canónica.

\begin{lem}\label{Lema9.4.2}
    Sea $A\in \Frm$, $S=\pt A$ $PA$ su marco de parches. Para cada punto $m\in \pt(PA)$, el elemento $p=m(0)$ es un punto de $A$ y es el único elemento tal que $u_p\leq m\leq w_a$.
\end{lem}

\begin{proof}
    El núcleo $m\in PA$ es $\wedge$-irreducible en $PA$. En particular, este no es $\tp$ y así $p\neq 1$. Consideremos $x, y\in A$ con $x\wedge y\leq p$. Notemos que $u_x, u_y\in PA$ y $u_x\wedge u_y=u_{x\wedge y}\leq m$ y como $m\in \pt(PA)$, entonces $u_x\leq m$ o $u_y\leq m$. Así tenemos 
    \[
    x=u_x(0)\leq m(0)=p\quad\mbox{ o }\quad y=u_y(0)\leq m(0)=p,
    \]
    es decir, $p\in S$.\\

    Por construcción tenemos $u_p\leq m\leq w_p$. Consideremos cualquier $a\in A$ con $u_a\leq m\leq w_a$. Evaluando en $0$ tenemos
    \[
    a=u_a(0)\leq m(0)=p\leq w_a(0)=a,
    \]
    es decir, $a=p$.
\end{proof}

Esto muestra que cualquiera que sean los puntos de $PA$, cada uno tiene un padre $p$ el cual es un punto de $A$ y la imagen de un punto de $PA$.\\

¿Qué tiene un punto de un marco que permite asociarlo con puntos salvajes? Recordemos que cada elemento máximo es automáticamente un punto, pero hay puntos que no son máximos.

\begin{lem}\label{Lema9.4.3}
    Si el punto $p$ del marco $A$ es máximo, entonces $u_p=w_p$ y $p$ no tiene puntos salvajes asociados.
\end{lem}

\begin{proof}
    Por la maximalidad de $p$ tenemos que para $x\in A$
    \[
    u_p=\left\{ \begin{array}{lcc} 1 & \mbox{si} & x \nleq p\\ \\ 
    p & \mbox{si} & x\leq p  \end{array} \right.=w_p
    \]
    y el intervalo $[u_p, w_p]$ colapsa, es decir, no hay nada entre estos.
\end{proof}

Conocemos varias condiciones en un marco que aseguran que todos los puntos sean máximos. Por ejemplo, este es el caso cuando el marco es ajustado o cuando es $\infty$-arreglado. Para tal marco, el Lema \ref{Lema9.4.3} nos dice que la situación del parche es simple.

\begin{thm}\label{Teorema9.4.4}
    Si cada punto del marco $A$ es máximo, entonces $A$ no tiene puntos salvajes y los dos espacios $^p(\pt A)$ y $\pt(PA)$ son esencialmente el mismo.
\end{thm}

Por supuesto, el Lema \ref{Lema9.4.3} no dice que un punto no máximo deba tener un punto salvaje asociado. De hecho, como veremos, no esta nada claro que permite o impide la existencia de puntos salvajes.\\

Sabemos que para un espacio $T_1$ todo punto es máximo. Así tenemos el siguiente resultado.

\begin{cor}\label{Corolario9.4.5}
    Si $A$ es un marco con espacio de puntos $T_1$, entonces $A$ no tiene puntos salvajes y $^p\pt A\simeq \pt (PA)$.
\end{cor}

¿Qué podemos decir de los puntos en $\pt(PA)$? Establecemos un poco de notación para ser usada con $m\in \pt(PA)$ arbitrario y obtener algunas propiedades. Por supuesto, si $m$ no es salvaje, entonces casi todo lo que hacemos ya se conoce.\\

Para $m\in\pt(PA)$ sea $p=m(0)$ el punto asociado y sea $M=\nabla(m)$ su filtro de admisibilidad. En general, este no necesita ser abierto. Sea lo que sea, $M$ tiene un mínimo núcleo asociado $v_M$, el menor compañero de $m$. No se sabe si $v_M\in PA$.\\

Consideremos $\mathcal{M}$ el conjunto de todos los filtros abiertos $F$ con $F\subseteq M$. Así $\mathcal{M}$ podría ser vacío. Sea $K=\bigvee \mathcal{M}$ donde este supremo está tomado en el copo de todos los filtros en $A$. Ya que
\[
v_K=\bigvee\{v_F\mid F\in \mathcal{M}\},
\]
entonces $v_K\leq v_M\leq m\leq w_p$ y $v_k\in PA$.

\begin{lem}
    Usando la notación anterior, para cada marco $A$ y $m\in \pt(PA)$ tenemos 
    \[
    m=u_p\vee v_M=u_p\vee v_K\quad\mbox{ y }\quad G\cap H\subseteq M\Rightarrow G\subseteq M\; \mbox{ o }\; H\subseteq M,
    \]
    para cualesquiera filtros abiertos $G$ y $H$.
\end{lem}

\begin{proof}
    Consideremos $\kappa=u_p\vee v_K$ de modo que $\kappa\leq u_p\vee v_M\leq m$. Así, una comparación $m\leq \kappa$ es suficiente para la primera parte.\\

    Como $m\in PA$ este es un supremo de núcleos $u_a\wedge v_F$ para ciertos $a\in A$ y filtro abierto $F$. Para tal núcleo tenemos $u_a\wedge v_F\leq m$ y por lo tanto, como $m\in \pt(PA)$ se cumple que $u_a\leq m$ o $v_F\leq m$. Esto da $a\leq p$ o $F\subseteq M$ y por lo tanto 
    \[
    u_a\wedge v_F\leq u_a\leq u_p\leq \kappa \quad \mbox{o}\quad u_a\wedge v_F\leq v_F\leq v_\kappa\leq \kappa
    \]
    se cumple, es decir, siempre se cumple que $u_a\wedge v_F\leq \kappa$. En particular, $m\leq \kappa$ ya que $m$ es el supremo de los núcleos $u_a\leq v_F$ considerados.\\

    Para la segunda parte, consideremos los filtros abiertos $G, H$ con $G\cap H\subseteq M$. Entonces $v_G\wedge v_H=v_{G\wedge H}\leq v_M\leq m$ y por lo tanto se cumple que $v_G\leq m$ o $v_H\leq m$, es decir, $G\subseteq M$ o $H\subseteq M$.
\end{proof}

Hay mucho que no se sabe sobre esta situación. Se concluye esta sección con lo que se cree es una pregunta muy importante.\\

Sea $A$ un marco arbitrario con espacio de puntos $S$. Consideremos el encaje topológico $^pS\to \pt(PA)$ descrito antes. Sabemos que $^pS$ no es necesariamente sobrio, pero $\pt(PA)$ es sobrio. La reflexión sobria de $^pS$ vive dentro de $\pt(PA)$ y es solo la clausura frontal de $^pS$. Esto lleva a la pregunta crucial

\textbf{Pregunta:} Para un marco $A$ ¿bajo que circunstancias es la reflexión sobria de $^pS$ solo el espacio $\pt(PA)$?\\

Es posible que sea así, pero aun no se ha encontrado una prueba o contraejemplo.

\section{Ejemplos}\label{Ejemplos Rosy}

En esta sección se presenta un resumen de los diferentes ejemplos que incluye Sexton en \cite{R.S.} así como el objetivo de cada uno de estos.

\subsection{La topología cofinita y la topología conumerable}

\subsection{La topología subregular de $\mathbb{R}$}

\subsection{La topología máxima compacta}

\subsection{Una construcción de pegado}

\subsection{La topología lider para árboles}