\chapter{Marcos arreglados vs cocientes compactos}\label{MarcosKC}

Mientras

\begin{dfn}\label{KOMPACT}
	Un marco $A$ tiene la propiedad $\mathrm{KC}$ si cada cociente compacto de $A$ es cerrado. En otras palabras, cada sublocal compacto es cerrado.
\end{dfn}

\begin{dfn}\label{ccquotien}
    Consideremos $A$ un marco y $F\in A^\wedge$. El cociente compacto $A_F$ es cerrado si $A_F=A_{u_d}$ para algún $d\in A$..
\end{dfn}

Si $A$ es un marco arreglado, quisiéramos saber si el marco cociente, es decir, el marco $A_j$ para $j\in NA$, es arreglado.
Con lo anterior, queremos trasladar la Definición \ref{Definición8.2.1} para $A_j$ cuando $j\in NA$, de modo que, para todo $F\in A_j^\wedge$, si
\[
x\in F \Rightarrow d\vee x=1
\]
con $d$ similar al de la definición, pero para este caso tenemos que $v_y\in NA_j$ y $0_{A_j}=j(0)$.

\begin{prop}\label{tidyquout}
    Si $A$ es un marco arreglado, entonces $A_j$ es arreglado.
\end{prop}

\begin{proof}
Es fácil verificar que $F\subseteq j_*F$. Como $A$ es arreglado y $F\in A^\wedge$, es cierto que 
\[
x\in F\Rightarrow \hat{d}\vee x=1,
\]
donde $\hat{d}=d(\alpha)=f^\alpha(0)$.\\
Si $\hat{d}\leq d$, entonces $d\vee x=1$, para $d=d(\alpha)=f^\alpha(j(0))$.\\

Así, por el Corolario \ref{finftyf}
\[
\hat{d}=\hat{d}(\alpha)\leq j(\hat{d}(\alpha))=j(\hat{f}^\alpha(0))\leq f^\alpha(j(0))=d(\alpha)=d.
\]

Por lo tanto, si $x\in F$, entonces $d\vee x=1$ y $A_j$ es arreglado.
\end{proof}

\begin{prop}\label{KCquout}
    Si $A$ cumple $KC$, entonces $A_j$ cumple $KC$ para cada $j\in N(A).$
\end{prop}

\begin{proof}
Consideremos $k\in NA_j$ tal que $(A_j)_k$ es compacto. 
Como todo filtro abierto es admisible, tenemos que $\nabla(k)\in A_j^\wedge$ 
y, por la Proposición \ref{fF}, $j_*\nabla(K)\in A^\wedge$.\\

Sea $l=j_*\circ  k\circ j^*\in NA$, entonces $A_l$ es un cociente compacto de $A$ y existe $a\in A$ tal que $l=u_a$. Así, tenemos el siguiente diagrama 
\[
\begin{tikzcd}
	A \arrow[r, "j^*"'] \arrow[rrr, "l", bend left] & A_j \arrow[r, "k"'] & (A_j)_k \arrow[r, "j_*"'] & A_j\subseteq A
	\end{tikzcd}\]
y $a\vee x=k(j(x))$. Por lo tanto, si $x=a$, $k(j(x))=a$.\\

Necesitamos que $k=u_b$ para algún $b\in A_j$. Si $x\in A_j$ y $b=j(a)$
\[
\begin{split}
u_b(x)= b\vee x= b\vee j(x)& =j(j(a)\vee j(x))\\
& =j(k(j(a))\vee x)\\
& =j(u_a(x))\\
& =j(k(x))\\	
&=k(x).
\end{split}
\]
Por lo tanto $u_b=k$.
\end{proof}

\begin{prop}\label{KCT1}
    Si $A$ es un marco $KC$, entonces $A$ es un marco $T_1$.
\end{prop}

\begin{proof}
Sabemos que $A$ es $T_1$ si y solo si para todo $p\in \pt A$, $p$ es máximo. Sean $p\in \pt A$ y $a\in A$ tales que $p\leq a\leq 1$. Consideremos 
\[
w_p(x)=\left\{\begin{array}{lcc}
1 & \mbox{ si } & x\nleq p\\
\\
p & \mbox{ si } & x\leq p
\end{array}\right.
\]
para $x\in A$. $P=\nabla(w_p)=\{x\in A\mid x\nleq p\}$ es un filtro completamente primo (en particular, $P\in A^\wedge$). Como $A$ es $KC$, entonces $A_{w_p}$ es un cociente compacto cerrado. Luego $u_p=w_p$, también
\[
u_p(a)=a\quad \mbox{ y }\quad w_p(a)=1.
\]
es decir, $a=1$. Por lo tanto $p$ es máximo. 
\end{proof}

