\chapter{Marcos arreglados vs cocientes compactos}\label{MarcosKC}

Mientras buscábamos alternativas para estudiar el comportamiento de los marcos arreglados, nos sugirieron consultar el artículo \emph{``Between $T_1$ and $T_2$''} de Albert Wilansky (ver \cite{A.W.}). Como su título lo sugiere, en el presentan propiedades espaciales que se ubican 
entre los axiomas clásicos de separación $T_1$ y $T_2$. Hasta este momento, nos interesamos un poco más en la propiedad $\mathrm{KC}$. 

\begin{dfn}\label{KCEspacial}
Un espacio topológico es llamada $\mathrm{KC}$ si cada conjunto compacto es cerrado y un espacio es llamada $\mathrm{US}$ si cada sucesión convergente tiene exactamente un límite al cual converge.
\end{dfn}

En \cite{A.W.}, prueban que 
\[
T_2\Rightarrow \mathrm{KC} \Rightarrow \mathrm{US} \Rightarrow T_1
\]
y además, dan ejemplos para mostrar que las flechas de regreso no siempre se cumplen.\\

Notemos que la Definición \ref{KCEspacial} es muy parecida a la Definición \ref{empaquetado}, con la omisión de los conjuntos compactos saturados. Como nuestro interés está puesto principalmente en los marcos, lo ideal es trasladar 
la noción $\mathrm{KC}$ hacía este lenguaje.

\section{Marcos $\mathrm{KC}$}\label{Marcoskc}
    
Sabemos que la noción de subespacio está en correspondencia biyectiva con la de sublocal. A su vez, los sublocales están en correspondencia biyectiva con los núcleos por medio de los cocientes que estos producen. De esta manera, parece ser que el 
camino adecuado que necesitamos para establecer la definición es a través de los núcleos $j\in NA$ y los cocientes $A_j$. La siguiente respuesta que necesitamos contestar es ¿qué sería lo equivalente a cociente compacto?

Primero. recordemos que es un marco compacto.

\begin{dfn}\label{marcocompacto}
Consideremos $A$ un marco. 
\begin{enumerate}
\item Una \emph{cubierta} de $A$ es un subconjunto $X\subseteq A$ tal que $\bigvee X=1$. 
\item Una \emph{subcubierta} de $X$ es un subconjunto $Y\subseteq X$ tal que $\bigvee Y=1$.
\item $A$ es \emph{compacto} si cada cubierta tiene una subcubierta finita.
\end{enumerate}
\end{dfn}

Podemos producir algunos cocientes especiales por medio de algunos núcleos, por ejemplo: para $a\in A$, los núcleos $u_a$ y $v_a$ proporcionan los cocientes $A_{u_a}$ y $A_{v_a}$, respectivamente. A estos cocientes podriamos llamarlos \emph{cociente cerrado} y \emph{cociente abierto}, debido al núcleo que los produce (el núcleo cerrado y el núcleo abierto, respectivamente). 
Lo anterior nos lleva a pensar que necesitamos un núcleo especial para definir el cociente compacto. 

\begin{prop}\label{cocientecompacto}
Sean $A$ un marco y $j\in NA$. $A_j$ es compacto si y solo si $\nabla(j)\in A^\wedge$, es decir, si todo filtro admisible es abierto.
\end{prop}

De esta manera, podemos dar nuestra definición de que un marco sea $\mathrm{KC}$.

\begin{dfn}\label{KOMPACT}
	Un marco $A$ tiene la propiedad $\mathrm{KC}$ si cada cociente compacto de $A$ es cerrado. En otras palabras, cada sublocal compacto es cerrado.
\end{dfn}

Notemos que todo lo anterior es equivalente a decir que para $F\in A^\wedge$,
\[
A_F=A_{u_d}
\]
para algún $d\in A$.

\begin{dfn}\label{ccquotien}
    Consideremos $A$ un marco y $F\in A^\wedge$. El cociente compacto $A_F$ es cerrado si $A_F=A_{u_d}$ para algún $d\in A$.
\end{dfn}

El filtro $F\in A^\wedge$ produce un intervalo $[v_F, w_F]$ y que a su vez proporciona una familia de cocientes compactos. Si el marco principal es $KC$, entonces para todo $j\in [v_F, w_F]$, $A_j$ es un cociente compacto cerrado. Con esto en mente, es fácil darse cuenta de que 
\[
\mathrm{KC} \Rightarrow \mbox{ Arreglado}.
\]
Veamos cual es la relación de la propiedad $\mathrm{KC}$ con las diferentes propiedades en marcos que conocemos.\\

Cuando trabajamos con el marco cociente, lo ideal es que este herede algunas de las propiedades del marco principal. En este caso, queremos ver si para $A$ arreglado o $\mathrm{KC}$, $A_j$ hereda este par de 
propiedades. Comenzaremos con arreglado.\\

Si $A$ es un marco arreglado, queremos trasladar la Definición \ref{Definición8.2.1} para $A_j$ cuando $j\in NA$, de modo que, para todo $F\in A_j^\wedge$, si
\[
x\in F \Rightarrow d\vee x=1
\]
con $d$ similar al de la definición, pero para este caso tenemos que $v_y\in NA_j$ y $0_{A_j}=j(0)$.

\begin{prop}\label{tidyquout}
    Si $A$ es un marco arreglado, entonces $A_j$ es arreglado.
\end{prop}

\begin{proof}
Es fácil verificar que $F\subseteq j_*F$. Como $A$ es arreglado y $F\in A^\wedge$, es cierto que 
\[
x\in F\Rightarrow \hat{d}\vee x=1,
\]
donde $\hat{d}=d(\alpha)=f^\alpha(0)$.\\
Si $\hat{d}\leq d$, entonces $d\vee x=1$, para $d=d(\alpha)=f^\alpha(j(0))$.\\

Así, por el Corolario \ref{finftyf}
\[
\hat{d}=\hat{d}(\alpha)\leq j(\hat{d}(\alpha))=j(\hat{f}^\alpha(0))\leq f^\alpha(j(0))=d(\alpha)=d.
\]

Por lo tanto, si $x\in F$, entonces $d\vee x=1$ y $A_j$ es arreglado.
\end{proof}

Ahora es el turno de $\mathrm{KC}$.
\begin{prop}\label{KCquout}
    Si $A$ cumple $KC$, entonces $A_j$ cumple $KC$ para cada $j\in N(A).$
\end{prop}

\begin{proof}
Consideremos $k\in NA_j$ tal que $(A_j)_k$ es compacto. 
Como todo filtro abierto es admisible, tenemos que $\nabla(k)\in A_j^\wedge$ 
y, por la Proposición \ref{fF}, $j_*\nabla(K)\in A^\wedge$.\\

Sea $l=j_*\circ  k\circ j^*\in NA$, entonces $A_l$ es un cociente compacto de $A$ y existe $a\in A$ tal que $l=u_a$. Así, tenemos el siguiente diagrama 
\[
\begin{tikzcd}
	A \arrow[r, "j^*"'] \arrow[rrr, "l", bend left] & A_j \arrow[r, "k"'] & (A_j)_k \arrow[r, "j_*"'] & A_j\subseteq A
	\end{tikzcd}\]
y $a\vee x=k(j(x))$. Por lo tanto, si $x=a$, $k(j(x))=a$.\\

Necesitamos que $k=u_b$ para algún $b\in A_j$. Si $x\in A_j$ y $b=j(a)$
\[
\begin{split}
u_b(x)= b\vee x= b\vee j(x)& =j(j(a)\vee j(x))\\
& =j(k(j(a))\vee x)\\
& =j(u_a(x))\\
& =j(k(x))\\	
&=k(x).
\end{split}
\]
Por lo tanto $u_b=k$.
\end{proof}

No hay mucho que hacer con las propiedades de separación. Si se cumple la propiedad $\mathbf{(H)}$, por la naturaleza de $\mathrm{KC}$, esta también se cumple. Con $T_1$ requerimos un poco más de trabajo.

\begin{prop}\label{KCT1}
    Si $A$ es un marco $KC$, entonces $A$ es un marco $T_1$.
\end{prop}

\begin{proof}
Sabemos que $A$ es $T_1$ si y solo si para todo $p\in \pt A$, $p$ es máximo. Sean $p\in \pt A$ y $a\in A$ tales que $p\leq a\leq 1$. Consideremos 
\[
w_p(x)=\left\{\begin{array}{lcc}
1 & \mbox{ si } & x\nleq p\\
\\
p & \mbox{ si } & x\leq p
\end{array}\right.
\]
para $x\in A$. $P=\nabla(w_p)=\{x\in A\mid x\nleq p\}$ es un filtro completamente primo (en particular, $P\in A^\wedge$). Como $A$ es $KC$, entonces $A_{w_p}$ es un cociente compacto cerrado. Luego $u_p=w_p$, también
\[
u_p(a)=a\quad \mbox{ y }\quad w_p(a)=1.
\]
es decir, $a=1$. Por lo tanto $p$ es máximo. 
\end{proof}

De esta manera tenemos que $\mathrm{KC}$ tiene un comportamiento parecido a la propiedad espacial puesto
\[
\mathbf{(H)} \Rightarrow \mathrm{KC} \Rightarrow \mbox{ Arreglado }\Rightarrow T_1.
\]

