\chapter{Preliminares}\label{Preliminares}

Para el desarrollo de toda esta teoría, primero necesitamos introducir aquellos conceptos centrales sobre los cuales trabajaremos. El concepto fundamental (y por tal motivo se menciona en el titulo del capítulo), es el concepto de marco. Recordemos que éste parte del concepto de retícula.\\

\begin{dfn}\label{Copo}
Sea $S$ un conjunto. Un \emph{orden parcial} sobre $S$ es una relación binaria \emph{``$\leq$''} la cual es 
\begin{enumerate}
\item \emph{Reflexiva:} para todo $a\in S$, $a\leq a$,
\item \emph{Transitiva:} si $a\leq b$ y $b\leq c$, entonces $a\leq c$,
\item \emph{Antisimétrica:} si $a\leq b$ y $b\leq a$, entonces $a=b$. 
\end{enumerate}
Un \emph{conjunto parcialmente ordenado} (o \emph{copo} de manera abreviada), es un conjunto equipado de un orden parcial. Si en la definición solo se cumplen las condiciones 1 y 2, entonces tenemos un \emph{preorden parcial}.
\end{dfn}

\begin{dfn}\label{Supremo}
Si $S$ es un copo y $A\subseteq S$. Decimos que un elemento $a\in S$ es un \emph{supremo} (la mínima cota superior), para $A$ y escribimos $a=\bigvee A$, si
\begin{enumerate}
\item $a$ es una cota superior de $A$; es decir, $s\leq a$ para todo $s\in A$
\item si $b\in S$ satisface que $s\leq b$ para todo $s\in A$, entonces $a\leq b$.
\end{enumerate}
\end{dfn}

Si $A$ es $\varnothing$, entonces utilizamos $0$ para denotar $\bigvee \varnothing$, donde $0$ es el elemento mínimo de $S$. De hecho, para cualesquiera dos elementos $a, b\in A$, podemos calcular $a\vee b$, es decir, ``$\vee$'' es una operación binaria. Además si para cualquier subconjunto finito $A$, $\vee$ y $0$ cumplen

\begin{enumerate}
    \item $a\vee a=a$,
    \item $a\vee b= b\vee a$,
    \item $a\vee (b\vee c)=(a\vee b) \vee c$,
    \item $a\vee 0=a$,
\end{enumerate}

para cualesquiera $a,b,c\in A$, obtenemos un conjunto con la estructura $(S, \vee, 0)$ la cual se conoce como \emph{estructura de semiretícula} o a veces \emph{$\vee-$semiretícula}.\\

De manera dual, en cualquier copo podemos considerar la noción de \emph{ínfimo}, (cota inferior más grande), definida invirtiendo todas las desigualdades de la Definición \ref{Supremo}. Así, escribimos $\bigwedge A$, $\wedge$ y $1$ para los análogos de $\bigvee A$, $\vee$ y $0$. Por lo tanto, obtenemos una  estructura $(S, \wedge, 1)$ que también es una semiretícula (o \emph{$\wedge-$semiretícula}).

\begin{dfn}\label{Reticula}
Una \emph{retícula} es un conjunto con dos operaciones binarias ($\vee$ y $\wedge$) y dos elementos distinguidos ($0$ y $1$) tal que $\vee$ (respectivamente $\wedge$) es asociativa, conmutativa, idempotente y tienen a $0$ (respectivamente 1) como elementos neutros.
\end{dfn} 

\begin{dfn}\label{Reticuladistributiva}
Consideremos una retícula $(S, \leq, \vee, \wedge, 0, 1)$. Si esta  cumple las siguientes leyes distributivas
\[
a\wedge (b\vee c)=(a\wedge b)\vee (a\wedge c)\quad\mbox{ y }\quad a\vee (b\wedge c)=(a\vee b)\wedge (a\vee c)
\]
para todo $a, b, c\in S$. Entonces decimos $S$ es una \emph{retícula distributiva}.
\end{dfn}

\begin{dfn}\label{Semiretículacompleta}
Decimos que una $\vee-$semiretícula es \emph{completa} si para cualquier subconjunto $A$ (no solo finito) existe $\bigvee A$. 
\end{dfn}

\begin{dfn}\label{frm}
Un \emph{marco} es una retícula completa $(S, \leq, \wedge, \bigvee, 0, 1)$ que cumple la siguiente ley distributiva, la cual se conoce como  \emph{ley distributiva para marcos} \emph{(\textbf{LDM})}, y dice lo siguiente:
\[
x\wedge\bigvee Y=\bigvee\{x\wedge y\mid y\in Y\}
\]
para cualesquiera $x\in S$ y $Y\subseteq S$.
\end{dfn}

\begin{ej}\label{ejem1}
Para todo espacio topológico $S$ se tienen dos familias de subconjuntos: los subconjuntos abiertos, que denotamos por $\mathcal{O}S$ y sus subconjuntos cerrados, denotados por $\mathcal{C}S$. De esta forma para cualquier $(S,\mathcal{O}S)$, $\mathcal{O}S$ tiene la estructura  de retícula completa $$(\mathcal{O}S, \subseteq, \cap,\bigcup, S,\emptyset).$$
Además la familia de subconjuntos abiertos cumple la LDM. Es decir, $\mathcal{O}S$ es un marco.\\

Consideremos $U\subseteq S$, denotamos por

\[
U,\qquad U^-,\qquad U^\circ,
\]
como el complemento, la cerradura y el interior de $U$ en $S$, respectivamente.\\ 

Como mencionamos en el ejemplo, $\mathcal{O}S$ es un marco, en la literatura a $\mathcal{O}S$ también se le conoce como el marco de abiertos de $S$ o como la retícula de conjuntos abiertos. Al ser $\mathcal{O}S$ una retícula, en ocasiones también se le denota por $\Omega(S)$. En estas notas aparecerán ambas notaciones para referirnos a los subconjuntos abiertos del espacio topológico $S$.\\

Calcular ínfimos arbitrarios se puede realizar de la siguiente forma 
\[
\bigwedge U=\left(\bigcap U\right)^\circ.
\]
\end{ej}

\begin{dfn}\label{morf}
Sean $A$ y $B$ dos marcos arbitrarios. Un \emph{morfismo de marcos} es una función $f\colon A\rightarrow B$ tal que para cualesquiera $a,b\in A$ y $X\subseteq A$ se cumple lo siguiente:
\begin{itemize}
\item $a\leq b$, $f(a)\leq f(b)$.
\item $f(0_A)=0_B$ y $f(1_A)=1_B$.
\item $f(a\wedge b)= f(a)\wedge f(b)$.
\item $f(\bigvee X)=\bigvee f(X)$.
\end{itemize}
\end{dfn}

Utilizando las Definiciones \ref{frm} y \ref{morf} podemos demostrar que la composición de morfismos de marcos es también un morfismo de marcos. Además, el morfismo identidad ($\id$) actúa como elemento neutro en la composición. Como consecuencia tenemos una categoría, la cual denotaremos por $\Frm$ y es conocida como la \emph{categoría de marcos}.

\begin{dfn}\label{adjder}
Sea $f:A\to B$ una función monótona. El \emph{adjunto derecho de $f$} es una función monótona $f_*:B\to A$ tal que $f(a)\leq b\Leftrightarrow a\leq f_*(b)$ para cualesquiera $a\in A$ y $b\in B$. Denotamos a $f$ como $f^*$ y escribimos \emph{$f^*\dashv f_*$}
\end{dfn}

La relación que existe entre los morfismos de marcos y el adjunto de un morfismo se enuncia en el siguiente resultado.

\begin{prop}\label{Adjuntoder}
Todo morfismo de marcos tiene adjunto derecho
\end{prop}

\begin{proof}
Sea $f\colon A\rightarrow B$ un morfismo de marcos, definimos $f_{*}\colon B\to A$ como $f_{*}(b)=\bigvee\{a\in A\mid f(a)\leq b\}$, con esta definición es inmediato que $f(a)\leq b\Rightarrow a\leq f_{*}(b)$ para cualesquiera $a\in A$ y $b\in B$. Ahora, si $a\leq f_{*}(b)=\bigvee \{a\in A\mid f(a)\leq b\}$ al aplicar $f$, como este es un morfismo de marcos y respeta supremos arbitrarios, nos queda que $f(a)\leq f_{*}(b)=\bigvee \{f(a)|a\in A, f(a)\leq b\}\leq b$. Por lo tanto, $f\dashv f_{*}$.
\end{proof}

Podemos construir la categoría opuesta a $\Frm$, esta recibe el nombre de \emph{categoría de locales}. Por la Proposición \ref{Adjuntoder} tenemos que para cualesquiera $A, B\in \Frm$, si
\[
f\colon A\rightarrow B \Longrightarrow f_*\colon B\rightarrow A
\]
donde $f$ es un morfismo de marcos y $f_*$ es su adjunto derecho. Notemos que $f_*$ invierte el dominio y codominio de $f$. Por lo tanto, podemos considerar un morfismo de locales como 
\[
f\colon B\rightarrow A 
\]
tales que $f$ preserva los ínfimos y sus adjuntos izquierdos ($f^*$), son un morfismo de marcos. Así, invirtiendo la composición de los morfismos de marcos obtenemos la categoría $\Loc=\Frm^{\op}$.\\

Recordemos que existe una adjunción entre $\Frm$ y $\Top$ (la categoría de espacios topológicos)

\[\begin{tikzcd}
	{\Top} \\
	{} & {} & {} \\
	{\Frm^{\op}}
	\arrow["{\pt}"{name=0, swap}, from=3-1, to=1-1, shift right=5]
	\arrow["{\mathcal{O}}"{name=1, swap}, from=1-1, to=3-1, shift right=5]
	\arrow["\dashv"{rotate=0}, from=1, to=0, phantom]
\end{tikzcd}\]
donde $\pt$ se le conoce como el funtor de puntos y $\mathcal{O}$ como el funtor de abiertos. En la siguiente sección mencionaremos de manera breve la construcción del funtor $\pt$.\\

De esta manera $\mathcal{O}(f)\colon \mathcal{O}S\to \mathcal{O}T$ es un morfismo de marcos y $f$ es una función continua $f\colon T\to S$, donde $\mathcal{O}(f)[U]=f^{-1}[U]$ para $U\in \mathcal{O}S$.

\section{El espacio de puntos de un marco}\label{Epuntos}

Sabemos que si $S$ es un espacio topológico, entonces la familia de todos los conjuntos abiertos, $\mathcal{O}S$, es un marco. Y la asignación $S\mapsto \mathcal{O}S$ está dada por un funtor contravariante.\\

Como mencionamos en la sección anterior el funtor $\mathcal{O}$ tiene un adjunto que lleva a cada marco a un espacio topológico. Esta es la construcción del espacio de puntos.

\begin{dfn}
    Sea $A$ un marco. Un \emph{caracter} de $A$ es un morfismo de marcos $h\colon A\to \mathbf{2}$, donde $\mathbf{2}$ es el marco de dos elementos.
\end{dfn}

Existe una correspondencia entre caracteres de un marco y otros dos dispositivos: elementos $\wedge-$irreducibles y filtros completamente primos. Estos últimos serán introducidos más adelante.

\begin{dfn}\label{infirre}
    Sea $A$ un marco. Un elemento $p\in A$ es $\wedge-$irreducible si $p\neq 1$ y si $x\wedge y\leq p$, entonces $x\leq p$ o $y\leq p$ se cumple para cada $x, y \in A$.
\end{dfn}

Para convertir un marco en un espacio, primero debemos saber quienes son los puntos.

\begin{dfn}
    Sea $A$ un marco. El \emph{espacio de puntos} de $A$ es la colección de todos los elementos $\wedge-$irreducibles de $A$. Denotamos a estos por $\pt A$.
\end{dfn}

Notemos que para que $\pt A$ sea un espacio, este necesita una topología.

\begin{dfn}
    Sea $A$ un maco con espacio de puntos $S=\pt A$, vistos como elementos $\wedge-$irreducibles. Definimos 

    \[
    U_A(a)=\{p\in S\mid a\nleq p\}
    \]
    para cada elemento $A\in A$. 
\end{dfn}

Si el claro el marco bajo el cual se esta trabajando, eliminamos el subindice $A$ y solo escribimos $U(a)$.\\

Se puede demostrar que las intersecciones finitas y las uniones arbitrarias de conjuntos de esta forma, también tienen la misma estructura, es decir,

\[
U(a)\cap U(b)=U(a\wedge b)\qquad\mbox{ y }\qquad\bigcup\{U(a)\mid a\in X\}=U\left(\bigvee X\right)
\]
se cumplen para todo $a, b\in A$ y $X\subseteq A$. Con lo mencionado antes tenemos el siguiente lema.

\begin{lem}\label{ReflexionEspacial}
    Sea $A$ un marco con espacio de puntos $S=\pt A$. La colección de conjuntos $\{U(a)\mid a\in A\}$ forman una topología en $S$ y 

    \begin{equation}\label{MorfismoRE}
            U_a(\_)\colon A\to \mathcal{O}S
    \end{equation}
    es un morfismo suprayectivo de marcos.
\end{lem}
La asignación presentada en \ref{MorfismoRE} es conocida como la \emph{reflexión espacial}.\\

Para agregar información adicional con respecto al espacio de puntos, enunciamos este último lema.

\begin{lem}
    Sea $A$ un marco con espacio de puntos $S$. El orden de especialización en $S$ es el orden inverso heredado de $A$. 
\end{lem}

En el Capítulo \ref{Parches}, haremos uso de la construcción del espacio de puntos para dar algunos otros resultados. Por el momento hablaremos del comportamiento algebraico que tienen los marcos.

\section{Estudio algebraico de los marcos}

\begin{dfn}\label{Cociente}
Un \emph{cociente de un marco} $A$ es un morfismo suprayectivo $$f:A\to B$$
\end{dfn}

\begin{dfn}
Sea $f:A\to B$ un morfismo de marcos, una \emph{congruencia} (o marco de congruencias), es el conjunto generado por la relación de equivalencia $x\sim y\Leftrightarrow f(x)=f(y)$.
\end{dfn}

Trasladando lo anterior al lenguaje de retículas de conjuntos abiertos tenemos que para un morfismo $h\colon \mathcal{O}S\to\mathcal{O}T$, tenemos una congruencia

\[
E_h=\{(U,V)\mid h(U)=h(V)\}.
\]
En particular, para un subespacio $X\subseteq S$, el encaje $j$ produce la congruencia

\[
E_X=\{(U,V)\mid U\cap X=V\cap Y\},
\]
tales congruencias pueden usarse como representaciones de subespacio.\\

\begin{dfn}\label{Filtro}
    Sea $A$ un marco. Un subconjunto $F\subseteq A$ es un \emph{filtro} si
    \begin{itemize}
        \item $1\in F$.
        \item $a\leq b$, $a\in F$, entonces $b\in F$.
        \item $a, b\in F$, entonces $a\wedge b\in F$.
    \end{itemize}
\end{dfn}
Decimos que un filtro es \emph{propio} si $0\notin F$. Si agregamos ciertas propiedades particulares obtenemos otros tipos de filtros.

\begin{dfn}\label{Tfiltros}
    Sea $A$ un marco. Decimos que un filtro propio $F$ en $A$ es:
    \begin{itemize}
        \item \emph{Primo} si $x\vee y\in F$, entonces $x\in F$ o $y\in F$.
        \item \emph{Completamente primo} si $\bigvee X\in F$, entonces $X\cap F\neq \emptyset$, para cada $X\subseteq A$.
        \item \emph{Abierto} (o de \emph{Scott}) si $\bigvee X\in F$, entonces $X\cap F\neq \emptyset$, para cada  conjunto dirigido $X\subseteq A$.
    \end{itemize}
\end{dfn}

Los filtros mencionados en la definición anterior están relacionados de la siguiente manera. 

\begin{prop}
    Un filtro es completamente primo si y solo si este es primo y abierto.
\end{prop}

Además, de manera equivalente podemos decir que un filtro es completamente primo si

\[
\bigcap_{i\in J} U_i\in F, \mbox{ entonces } \exists k\in J \mbox{ tal que }U_k\in F.
\]
Notemos que esta noción está dada en el lenguaje de retículas abiertas.\\

Para un espacio $S$ si consideramos $x\in S$ tenemos el filtro completamente primo de vecindades abiertas de $x$

\[
F(x)=\{U\in \mathcal{O}S\mid x\in U\}.
\]


\begin{prop}\label{CaracterizacionFabiertos} 
Sea $\mathcal{F}_a$ la familia de filtros abiertos, entonces 
    \begin{enumerate}
        \item si $F_1, F_2\in \mathcal{F}_a$, entonces $F_1\cap F_2 \in \mathcal{F}_a$.
        \item si $\mathcal{U}\subseteq \mathcal{F}_a$ una familia dirigida, entonces $\bigcup \mathcal{U}\in \mathcal{F}_a$.
    \end{enumerate}
\end{prop}

En la sección anterior mencionamos que los filtros completamente primos estaban en correspondencia con otros dispositivos. Aclaramos esto en el siguiente lema.

\begin{lem}
    Sea $A$ un marco. Los dispositivos 
    \begin{itemize}
        \item caracteres de $A$,
        \item filtros completamente primos de $A$,
        \item elementos $\wedge-$irreducibles de $A$
    \end{itemize}
    están en correspondencia biyectiva por pares.
\end{lem}