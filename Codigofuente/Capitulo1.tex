\chapter{Preliminares}\label{Preliminares}

El desarrollo de esta investigación gira en torno de dos ejes principales: la teoría de marcos y la topología de un espacio. 
Por tal motivo, en este capítulo nos encargamos de presentar toda la información necesaria para su desarrollo y comprensión.
Para consultar información adicional de la que aquí mencionaremos, recomendamos consultar \cite{P.T.} o \cite{J.P.}. 
Algunas de las cuestiones espaciales que aquí serán mencionadas pueden encontrarse también en \cite{R.S.}, \cite{R.S.2} y \cite{R.S.3}.\\

Algunos de los resultados que se muestran en este capítulo son muy conocidos, por tal motivo, los presentamos si prueba. 
Los que van acompañados de su demostración es porque consideramos que son más relevantes o simplemente porque no encontramos la prueba 
en la literatura (por ejemplo, la Proposición \ref{fF}).

\section{Teoría de marcos}\label{marcos}

Trabajar con marcos, hasta cierto punto, es trabajar con estructura sencillas de definir, pues estos se construyen a partir de un caso particular de retículas. De esta manera, 
lo ideal es comenzar con estructuras ordenadas.

\begin{dfn}\label{Copo}
Sea $S$ un conjunto. Un \emph{orden parcial} sobre $S$ es una relación binaria \emph{``$\leq$''} la cual es 
\begin{enumerate}
\item \emph{Reflexiva:} para todo $a\in S$, $a\leq a$,
\item \emph{Transitiva:} si $a\leq b$ y $b\leq c$, entonces $a\leq c$,
\item \emph{Antisimétrica:} si $a\leq b$ y $b\leq a$, entonces $a=b$. 
\end{enumerate}
Un \emph{conjunto parcialmente ordenado} (o \emph{copo} de manera abreviada), es un conjunto equipado de un orden parcial. Si en la definición solo se cumplen las condiciones 1 y 2, entonces tenemos un \emph{preorden parcial}.
\end{dfn}

\begin{dfn}\label{Supremo}
Si $S$ es un copo y $A\subseteq S$. Decimos que un elemento $a\in S$ es un \emph{supremo} (la mínima cota superior), para $A$ y escribimos $a=\bigvee A$, si
\begin{enumerate}
\item $a$ es una cota superior de $A$; es decir, $s\leq a$ para todo $s\in A$
\item si $b\in S$ satisface que $s\leq b$ para todo $s\in A$, entonces $a\leq b$.
\end{enumerate}
\end{dfn}

Si $A$ es $\varnothing$, entonces utilizamos $0$ para denotar $\bigvee \varnothing$, donde $0$ es el elemento mínimo de $S$. De hecho, para cualesquiera dos elementos $a, b\in A$, podemos calcular $a\vee b$, es decir, ``$\vee$'' es una operación binaria. Además si para cualquier subconjunto finito $A$, $\vee$ y $0$ cumplen

\begin{enumerate}
    \item $a\vee a=a$,
    \item $a\vee b= b\vee a$,
    \item $a\vee (b\vee c)=(a\vee b) \vee c$,
    \item $a\vee 0=a$,
\end{enumerate}

para cualesquiera $a,b,c\in A$, obtenemos un conjunto con la estructura $(S, \vee, 0)$ la cual se conoce como \emph{estructura de semiretícula} o a veces \emph{$\vee-$semiretícula}.\\

De manera dual, en cualquier copo podemos considerar la noción de \emph{ínfimo}, (cota inferior más grande), definida invirtiendo todas las desigualdades de la Definición \ref{Supremo}. Así, escribimos $\bigwedge A$, $\wedge$ y $1$ para los análogos de $\bigvee A$, $\vee$ y $0$. Por lo tanto, obtenemos una  estructura $(S, \wedge, 1)$ que también es una semiretícula (o \emph{$\wedge-$semiretícula}).

\begin{dfn}\label{Reticula}
Una \emph{retícula} es un conjunto con dos operaciones binarias ($\vee$ y $\wedge$) y dos elementos distinguidos ($0$ y $1$) tal que $\vee$ (respectivamente $\wedge$) es asociativa, conmutativa, idempotente y tienen a $0$ (respectivamente 1) como elementos neutros.
\end{dfn} 

\begin{dfn}\label{Reticuladistributiva}
Consideremos una retícula $(S, \leq, \vee, \wedge, 0, 1)$. Si esta  cumple las siguientes leyes distributivas
\[
a\wedge (b\vee c)=(a\wedge b)\vee (a\wedge c)\quad\mbox{ y }\quad a\vee (b\wedge c)=(a\vee b)\wedge (a\vee c)
\]
para todo $a, b, c\in S$. Entonces decimos $S$ es una \emph{retícula distributiva}.
\end{dfn}

\begin{dfn}\label{Semiretículacompleta}
Decimos que una $\vee-$semiretícula es \emph{completa} si para cualquier subconjunto $A$ (no solo finito) existe $\bigvee A$. 
\end{dfn}

\begin{dfn}\label{frm}
Un \emph{marco} es una retícula completa $(S, \leq, \wedge, \bigvee, 0, 1)$ que cumple la siguiente ley distributiva, la cual se conoce como  \emph{ley distributiva para marcos} \emph{(\textbf{LDM})}, y dice lo siguiente:
\[
x\wedge\bigvee Y=\bigvee\{x\wedge y\mid y\in Y\}
\]
para cualesquiera $x\in S$ y $Y\subseteq S$.
\end{dfn}

\begin{ej}\label{ejem1}
Para todo espacio topológico $S$ se tienen dos familias de subconjuntos: los subconjuntos abiertos, que denotamos por $\mathcal{O}S$ y sus subconjuntos cerrados, denotados por $\mathcal{C}S$. De esta forma para cualquier $(S,\mathcal{O}S)$, $\mathcal{O}S$ tiene la estructura  de retícula completa $$(\mathcal{O}S, \subseteq, \cap,\bigcup, S,\emptyset).$$
Además la familia de subconjuntos abiertos cumple la LDM. Es decir, $\mathcal{O}S$ es un marco.\\

Consideremos $U\subseteq S$, denotamos por

\[
U,\qquad U^-,\qquad U^\circ,
\]
como el complemento, la cerradura y el interior de $U$ en $S$, respectivamente.\\ 

Como mencionamos en el ejemplo, $\mathcal{O}S$ es un marco, en la literatura a $\mathcal{O}S$ también se le conoce como el marco de abiertos de $S$ o como la retícula de conjuntos abiertos. Al ser $\mathcal{O}S$ una retícula, en ocasiones también se le denota por $\Omega(S)$. En estas notas aparecerán ambas notaciones para referirnos a los subconjuntos abiertos del espacio topológico $S$.\\

Calcular ínfimos arbitrarios se puede realizar de la siguiente forma 
\[
\bigwedge U=\left(\bigcap U\right)^\circ.
\]
\end{ej}

\begin{dfn}\label{morf}
Sean $A$ y $B$ dos marcos arbitrarios. Un \emph{morfismo de marcos} es una función $f\colon A\rightarrow B$ tal que para cualesquiera $a,b\in A$ y $X\subseteq A$ se cumple lo siguiente:
\begin{itemize}
\item $a\leq b$, $f(a)\leq f(b)$.
\item $f(0_A)=0_B$ y $f(1_A)=1_B$.
\item $f(a\wedge b)= f(a)\wedge f(b)$.
\item $f(\bigvee X)=\bigvee f(X)$.
\end{itemize}
\end{dfn}

Utilizando las Definiciones \ref{frm} y \ref{morf} podemos demostrar que la composición de morfismos de marcos es también un morfismo de marcos. Además, el morfismo identidad ($\id$) actúa como elemento neutro en la composición. Como consecuencia tenemos una categoría, la cual denotaremos por $\Frm$ y es conocida como la \emph{categoría de marcos}.

\begin{dfn}\label{adjder}
Sea $f:A\to B$ una función monótona. El \emph{adjunto derecho de $f$} es una función monótona $f_*:B\to A$ tal que $f(a)\leq b\Leftrightarrow a\leq f_*(b)$ para cualesquiera $a\in A$ y $b\in B$. Denotamos a $f$ como $f^*$ y escribimos \emph{$f^*\dashv f_*$}
\end{dfn}

La relación que existe entre los morfismos de marcos y el adjunto de un morfismo se enuncia en el siguiente resultado.

\begin{prop}\label{Adjuntoder}
Todo morfismo de marcos tiene adjunto derecho
\end{prop}

\begin{proof}
Sea $f\colon A\rightarrow B$ un morfismo de marcos, definimos $f_{*}\colon B\to A$ como $f_{*}(b)=\bigvee\{a\in A\mid f(a)\leq b\}$, con esta definición es inmediato que $f(a)\leq b\Rightarrow a\leq f_{*}(b)$ para cualesquiera $a\in A$ y $b\in B$. Ahora, si $a\leq f_{*}(b)=\bigvee \{a\in A\mid f(a)\leq b\}$ al aplicar $f$, como este es un morfismo de marcos y respeta supremos arbitrarios, nos queda que $f(a)\leq f_{*}(b)=\bigvee \{f(a)|a\in A, f(a)\leq b\}\leq b$. Por lo tanto, $f\dashv f_{*}$.
\end{proof}

Podemos construir la categoría opuesta a $\Frm$, esta recibe el nombre de \emph{categoría de locales}. Por la Proposición \ref{Adjuntoder} tenemos que para cualesquiera $A, B\in \Frm$, si
\[
f\colon A\rightarrow B \Longrightarrow f_*\colon B\rightarrow A
\]
donde $f$ es un morfismo de marcos y $f_*$ es su adjunto derecho. Notemos que $f_*$ invierte el dominio y codominio de $f$. Por lo tanto, podemos considerar un morfismo de locales como 
\[
f\colon B\rightarrow A 
\]
tales que $f$ preserva los ínfimos y sus adjuntos izquierdos ($f^*$), son un morfismo de marcos. Así, invirtiendo la composición de los morfismos de marcos obtenemos la categoría $\Loc=\Frm^{\op}$.\\

Recordemos que existe una adjunción entre $\Frm$ y $\Top$ (la categoría de espacios topológicos)

\[\begin{tikzcd}
	{\Top} \\
	{} & {} & {} \\
	{\Frm^{\op}}
	\arrow["{\pt}"{name=0, swap}, from=3-1, to=1-1, shift right=5]
	\arrow["{\mathcal{O}}"{name=1, swap}, from=1-1, to=3-1, shift right=5]
	\arrow["\dashv"{rotate=0}, from=1, to=0, phantom]
\end{tikzcd}\]
donde $\pt$ se le conoce como el funtor de puntos y $\mathcal{O}$ como el funtor de abiertos. En la siguiente subsección mencionaremos de manera breve la construcción del funtor $\pt$.\\

De esta manera $\mathcal{O}(f)\colon \mathcal{O}S\to \mathcal{O}T$ es un morfismo de marcos y $f$ es una función continua $f\colon T\to S$, donde $\mathcal{O}(f)[U]=f^{-1}[U]$ para $U\in \mathcal{O}S$.

\subsection{El espacio de puntos de un marco}\label{Epuntos}

Sabemos que si $S$ es un espacio topológico, entonces la familia de todos los conjuntos abiertos, $\mathcal{O}S$, es un marco. Y la asignación $S\mapsto \mathcal{O}S$ está dada por un funtor contravariante.\\

Como mencionamos antes de comenzar esta subsección, el funtor $\mathcal{O}$ tiene un adjunto que lleva a cada marco a un espacio topológico. Esta es la construcción del \emph{espacio de puntos}.

\begin{dfn}
    Sea $A$ un marco. Un \emph{caracter} de $A$ es un morfismo de marcos $h\colon A\to \mathbf{2}$, donde $\mathbf{2}$ es el marco de dos elementos.
\end{dfn}

Existe una correspondencia entre caracteres de un marco y otros dos dispositivos: elementos $\wedge-$irreducibles y filtros completamente primos. Estos últimos serán introducidos más adelante.

\begin{dfn}\label{infirre}
    Sea $A$ un marco. Un elemento $p\in A$ es $\wedge-$irreducible si $p\neq 1$ y si $x\wedge y\leq p$, entonces $x\leq p$ o $y\leq p$ se cumple para cada $x, y \in A$.
\end{dfn}

Para convertir un marco en un espacio, primero debemos saber quienes son los puntos.

\begin{dfn}\label{Espacio de puntos}
    Sea $A$ un marco. El \emph{espacio de puntos} de $A$ es la colección de todos los elementos $\wedge-$irreducibles de $A$. Denotamos a estos por $\pt A$.
\end{dfn}

Notemos que para que $\pt A$ sea un espacio, este necesita una topología.

\begin{dfn}\label{Reflexion espacial}
    Sea $A$ un maco con espacio de puntos $S=\pt A$, vistos como elementos $\wedge-$irreducibles. Definimos 

    \[
    U_A(a)=\{p\in S\mid a\nleq p\}
    \]
    para cada elemento $A\in A$. 
\end{dfn}

Si es claro el marco bajo el cual se esta trabajando, eliminamos el subindice $A$ y solo escribimos $U(a)$.\\

Se puede demostrar que las intersecciones finitas y las uniones arbitrarias de conjuntos de esta forma, también tienen la misma estructura, es decir,

\[
U(a)\cap U(b)=U(a\wedge b)\qquad\mbox{ y }\qquad\bigcup\{U(a)\mid a\in X\}=U\left(\bigvee X\right)
\]
se cumplen para todo $a, b\in A$ y $X\subseteq A$. Con lo mencionado antes tenemos el siguiente lema.

\begin{lem}\label{ReflexionEspacial}
    Sea $A$ un marco con espacio de puntos $S=\pt A$. La colección de conjuntos $\mathcal{O}S=\{U(a)\mid a\in A\}$ forman una topología en $S$ y 

    \begin{equation}\label{MorfismoRE}
            U_a(\_)\colon A\to \mathcal{O}S
    \end{equation}
    es un morfismo suprayectivo de marcos.
\end{lem}
El morfismo presentado en (\ref{MorfismoRE}) es conocido como la \emph{reflexión espacial} y al conjunto $\mathcal{O}S$ se le conoce como \emph{el marco de abiertos}.\\

Para agregar información adicional con respecto al espacio de puntos, enunciamos este último lema.

\begin{lem}\label{Topologiapt}
    Sea $A$ un marco con espacio de puntos $S$. El orden de especialización en $S$ es el orden inverso heredado de $A$. 
\end{lem}

En el Capítulo \ref{Parches}, haremos uso de la construcción del espacio de puntos para dar algunos otros resultados. Por el momento hablaremos de otros objetos que están en correspondencia biyectiva con los 
elementos del espacio de puntos.

\begin{dfn}\label{Filtro}
    Sea $A$ un marco. Un subconjunto $F\subseteq A$ es un \emph{filtro} si
    \begin{itemize}
        \item $1\in F$.
        \item $a\leq b$, $a\in F$, entonces $b\in F$.
        \item $a, b\in F$, entonces $a\wedge b\in F$.
    \end{itemize}
\end{dfn}
Decimos que un filtro es \emph{propio} si $0\notin F$. Si agregamos ciertas propiedades particulares obtenemos otros tipos de filtros.

\begin{dfn}\label{Tfiltros}
    Sea $A$ un marco. Decimos que un filtro $F$ en $A$ es:
    \begin{itemize}
        \item \emph{Primo} si $x\vee y\in F$, entonces $x\in F$ o $y\in F$. Además, $F$ requiere ser propio.
        \item \emph{Completamente primo} si $\bigvee X\in F$, entonces $X\cap F\neq \emptyset$, para cada $X\subseteq A$. Además. requiere ser propio.
        \item \emph{Abierto} (o de \emph{Scott}) si $\bigvee X\in F$, entonces $X\cap F\neq \emptyset$, para cada  conjunto dirigido $X\subseteq A$.
    \end{itemize}
\end{dfn}

A manera de notación, agrupamos a los distintos filtros en los siguientes conjuntos:
\[
Fil(A)=\{\mbox{Filtros en $A$}\}, \quad A^\wedge=\{\mbox{F. abiertos en $A$}\}, \quad CFil=\{\mbox{F. c. primos en $A$}\}.
\]
Los filtros mencionados en la Definición \ref{Tfiltros} están relacionados de la siguiente manera. 

\begin{prop}
    Un filtro es completamente primo si y solo si este es primo y abierto.
\end{prop}

Además, de manera equivalente podemos decir que un filtro es completamente primo si

\[
\bigcap_{i\in J} U_i\in F, \mbox{ entonces } \exists k\in J \mbox{ tal que }U_k\in F.
\]
Notemos que esta noción está dada para los abiertos del espacio de puntos, es decir, elementos de $\mathcal{O}S$.\\

Para un espacio $S$ si consideramos $x\in S$, tenemos el filtro completamente primo de vecindades abiertas de $x$

\[
F(x)=\{U\in \mathcal{O}S\mid x\in U\}.
\]


\begin{prop}\label{CaracterizacionFabiertos} 
Sea $\mathcal{F}_a$ la familia de filtros abiertos, entonces 
    \begin{enumerate}
        \item si $F_1, F_2\in \mathcal{F}_a$, entonces $F_1\cap F_2 \in \mathcal{F}_a$.
        \item si $\mathcal{U}\subseteq \mathcal{F}_a$ una familia dirigida, entonces $\bigcup \mathcal{U}\in \mathcal{F}_a$.
    \end{enumerate}
\end{prop}

Como mencionamos antes, los elementos de $\pt A$ están en correspondencia biyectiva con algunos otros dispositivos que se pueden definir para el marco $A$, uno de estos son 
precisamente los filtros completamente primos. 

\begin{lem}
    Sea $A$ un marco. Los dispositivos 
    \begin{itemize}
        \item caracteres de $A$,
        \item filtros completamente primos de $A$,
        \item elementos $\wedge-$irreducibles de $A$
    \end{itemize}
    están en correspondencia biyectiva por pares.
\end{lem}

Si $f^*\colon A\to B$ es un morfismo de marcos y $F\subseteq A$, $G\subseteq B$ filtros en $A$, $B$, respectivamente, podemos producir nuevos filtros como sigue
\begin{equation}\label{Imagenfiltros}
b\in f^*F \Leftrightarrow f_*(b)\in F\quad \mbox{ y }\quad a\in f_*G \Leftrightarrow f^*(a)\in G
\end{equation}
donde $a\in A, b\in B$ y $f_*$ es el adjunto derecho de $f^*$. Aquí $f^*F\subseteq B$ y $f_*G\subseteq A$ son filtros en $B$ y $A$, respectivamente.\\

\begin{prop}\label{fF}
    Para $f=f^*\colon A\to B$ un morfismo de marcos y $G\in B^\wedge$, se cumple que $f_*G\in A^\wedge$.
\end{prop}

\begin{proof}
Por (\ref{Imagenfiltros}), $f_*G$ es un filtro en $A$. Necesitamos que $f_*G$ satisfaga la condición de filtro abierto. Sea $X\subseteq A$ tal que $\bigvee X\in f_*G$, con $X$ dirigido. Entonces
\[
Y=\{f(x)\mid x\in X\}
\] 
es dirigido y $f(\bigvee X)=\bigvee f[X]=\bigvee Y\in G$. Como $G$ es un filtro abierto, existe $y=f(x)\in Y$ tal que $y\in G$. Así $x\in f_*G$, de modo que, $f_*G\in A^\wedge$.
\end{proof}

Este última proposición será de gran utilidad para probar algunos otros resultados en los Capitulos \ref{Relación MA y AH} y \ref{MarcosKC}.

\subsection{Estudio algebraico de los marcos}\label{E. algebraico}

Como en toda estructura algebraica, en ocasiones es más sencillo obtener información de la estructura si ponemos nuestra atención en sus \emph{cocientes}. Esto mismo ocurre en la categoría $\Frm$. 
De manera similar a cualquier otra estructura, podemos obtener lo cocientes a través de una relación de equivalencia. Con esta, definir ciertas congruencias y por medio de las congruencias obtener 
el cociente.\\

La ventaja que proporciona trabajar con los marco es que tenemos las siguientes correspondencias biyectivas:
\[
\mbox{Congruencias }\leftrightarrow \mbox{ Conjuntos implicativos }\leftrightarrow \mbox{ Núcleos}
\]
Para observar más detalles sobre estas correspondencias, se puede consultar \cite{A.Z.}. Nosotros solo mencionaresmos los detalles necesarios.

\begin{dfn}
Sea $f:A\to B$ un morfismo de marcos, una \emph{congruencia} (o marco de congruencias), es el conjunto generado por la relación de equivalencia $x\sim y\Leftrightarrow f(x)=f(y)$.
\end{dfn}

Trasladando lo anterior al lenguaje de retículas de conjuntos abiertos tenemos que para un morfismo $h\colon \mathcal{O}S\to\mathcal{O}T$, tenemos una congruencia

\[
E_h=\{(U,V)\mid h(U)=h(V)\}.
\]
En particular, para un subespacio $X\subseteq S$, el encaje $j$ produce la congruencia

\[
E_X=\{(U,V)\mid U\cap X=V\cap Y\},
\]
tales congruencias pueden usarse como representaciones de subespacio.\\

\begin{dfn}\label{Cociente}
Un \emph{cociente de un marco} $A$ es un morfismo suprayectivo $$f:A\to B$$
\end{dfn}

\begin{dfn}
Sea $f:A\to B$ un morfismo de $\bigvee-$retículas, el \emph{kernel} de $f$ es la relación de equivalencia dada por $x\sim y\Leftrightarrow f(x)=f(y)$
\end{dfn}

Sabemos que si $f=f^*$ es un morfismo de marcos, entonces este tiene adjunto derecho $f_*$. El siguiente resultado relaciona a mabos para obtener el kernel. 
\begin{lem}
Sea $f:A\to B$ un morfismo de $\bigvee-$retículas, el kernel $k$ de $f$ es 
$$k=f_* \circ f^*$$
\end{lem}

\begin{dfn}
Sea $A$ una $\bigvee-$retícula. Decimos que $F\subseteq A$ es \emph{$\bigwedge-$cerrado} o \emph{cerrado bajo ínfimos} si $\bigwedge X\in F$ para cualquier $X\subseteq F$.
\end{dfn}

\begin{dfn}\label{Implicativo}
Sea $A\in \Frm$. Decimos que $F\subseteq A$ es un \emph{conjunto implicativo} de $A$ si éste es $\bigwedge-$cerrado y cumple que para $a\in F$, $(x\succ a)\in F$, donde $x\in A$.
\end{dfn}

El cociente adecuado para un marco $A$ es cierto conjunto de puntos fijos. Por el momento dejaremos los detalles importantes de esta afirmación para la 
siguiente sección. 

\begin{dfn}
Sea $A\in \Frm$ y $j\colon A\to A$. $A_j$ es el \emph{conjunto de puntos fijos} y se define como $$A_j=\{a\in A|a=j(a)\}.$$
\end{dfn}

\section{El ensamble de un marco}
Si $A$ es un marco, podemos definir operadores $j\colon A\to A$ y al solicitarles algunas condiciones especificas, estos pueden permitirnos construir nuevos marcos.
Los detalles de esta sección pueden consultarse en \cite{J.M.}, \cite{A.Z.} o en el compendio de notas sobre toería de marcos de Simmons (ver \cite{H.S.}).
\subsection{Operadores en $A$}
\begin{dfn}\label{Operadores}
Sea $A\in\Frm$. Decimos que:
\begin{enumerate}
\item Una \emph{derivada} en $A$ es una función $j:A\to A$ tal que 
\begin{enumerate}
\item $a\leq j(a)$ para cualquier $a\in A$, (el operador infla).
\item Si $a\leq b\Rightarrow j(a)\leq j(b)$ para cualesquiera $a,b\in A$, (el operador es monótono).
\end{enumerate}
$DA$ denotará al conjunto de todos los operadores derivada sobre $A$.

\item $j$ es un \emph{operador cerradura} si $j\in DA$ y $j^2=j$.\\
$CA$ denotará al conjunto de todos los operadores cerradura sobre $A$.
\item $j$ es un \emph{núcleo} si $j\in CA$ y $j(a)\wedge j(b)\leq j(a\wedge b)$.\\
$NA$ denotará al conjunto de todos los núcleos sobre $A$.
\end{enumerate}
\end{dfn}

Notemos que por la forma en que definimos estos tres operadores tenemos la siguiente relación entre ellos
$$NA\subseteq CA\subseteq DA.$$

\begin{ej}\label{ua,va,wa}
Si $A\in \Frm$ para cualquier $a\in A$ definimos las funciones
\begin{itemize}
\item $u_a\colon A\to A, \, u_a(x)=a\vee x, \mbox{    }\forall x\in A$ \emph{(núcleo cerrado)}
\item $v_a\colon A\to A, \, v_a(x)=(a\succ x), \mbox{    }\forall x\in A$ \emph{(núcleo abierto)}
\item $w_a\colon A\to A, \, w_a(x)=((x\succ a)\succ a), \mbox{    }\forall x\in A$ \emph{(núcleo regular)}
\end{itemize}
Es sencillo verificar que los tres operadores definidos son núcleos.
\end{ej}

Si $j\in NA$ es un núcleo arbitrario, este puede obtenerse por medio de los núcleos del Ejemplo \ref{ua,va,wa}. 

\begin{lem}\label{Lemaadecuado}
Para cada marco $A$, núcleo $j\in NA$ y $a\in A$, donde $b=j(0)$, se cumple lo siguiente
\begin{enumerate}
    \item $u_a\leq j \Longleftrightarrow a\leq j(0)$.
    \item $v_a\leq j \Longleftrightarrow j(a)=1$.
    \item $j\leq w_a \Longleftrightarrow j(a)=a$.
    \item $w_a\leq j \Longleftrightarrow j=w_b$.
    \item $u_a$ y $v_a$ son complementados en $NA$.
\end{enumerate}
\end{lem}

\begin{lem}\label{Rgnucleos}
Para cada marco $A$ tenemos que 
\[
\bigvee\{u_{j(a)}\wedge v_a\mid a\in A\}=j=\bigwedge\{w_a\mid a\in A_j\}
\]
donde $j\in NA$.
\end{lem}

\begin{lem}\label{Composicionnucleos}
Para cada marco $A$ tenemos lo siguiente
\[
v_b\vee j\vee u_a= v_b \circ j\circ u_a\qquad\mbox{ y }\qquad w_a \vee j= w_a \circ j \circ w_a 
\]
para cada $a,b\in A$ y $j\in NA$.
\end{lem}

\begin{thm}\label{Teorema5.10}
Para cada marco $A$ consideremos cualquier elemento $a\in A$ y núcleo $k\in NA$ con $u_a\leq k\leq w_a$. Entonces 
\[
j\vee w_a=w_a\circ j\circ k=w_d
\]
donde $d=w_a(j(a))$.
\end{thm}

\begin{lem}\label{Lema3.0.4}
Para cada marco $A$ tenemos que 
\[
w_{(b\succ a)}=v_b\vee w_a
\]
para cada $a,b\in A$.
\end{lem}

Haciendo uso de las definiciones presentadas en la Subsección \ref{E. algebraico} podemos enunciar el siguiente resultado. 

\begin{lem}
Para cualesquiera $A\in\Frm$ y $j\in CA$, $A_j$ es un conjunto implicativo si y solo si $j\in NA$. Además, si $j\in NA$, $A_j$ es un marco y $j^*:A\to A_j$, $j^*(x)=j(x)$ es un morfismo de marcos.
\end{lem}

Notemos que $j^*$ no es mas que la restricción del morfismo $j$ en $A_j$. Además, el conjunto de puntos fijos nos permite factorizar cualquier morfismo de marcos a través del morfismo $j^*$.

\begin{thm}
Sean $A\in\Frm$, $j\in NA$. Si $f:A\to B$ es un morfismo de marcos y $j\leq k$, donde $k$ es el kernel de $f$, entonces existe un único morfismo de marcos $f^\#:A_j\to B$ tal que hace conmutar el siguiente diagrama
\[\begin{tikzcd}
	A && B \\
	\\
	{A_j}
	\arrow["f", from=1-1, to=1-3]
	\arrow["{j^*}"', from=1-1, to=3-1]
	\arrow["{f^\#}"', dashed, from=3-1, to=1-3]
\end{tikzcd}\]
\end{thm}

Sabemos que un operador derivada es un morfismo monótono y que infla. Ahora, si además de estas dos condiciones le pedimos que cumpla lo que se menciona en la siguiente definición, obtenemos nuevos operadores. A estos les llamaremos operadores estables y prenúcleos.

\begin{dfn}
Sea $j\in DA$. Para cualesquiera $a,b\in A$ tenemos que
\begin{enumerate}
\item $j$ es un \emph{operador estable} si $j(a)\wedge b\leq j(a \wedge b)$.\\
$SA$ denotará al conjunto de todos los operadores estables sobre $A$.

\item $j$ es un \emph{prenúcleo} si $j(a\wedge b)=j(a) \wedge j(b)$.\\
$PA$ denotará al conjunto de todos los prenúcleos sobre $A$.
\end{enumerate}
\end{dfn}

\begin{obs}
\noindent
\begin{enumerate}
\item Todo prenúcleo es un operador estable.
\item Si una derivada es estable e idempotente, entonces es un núcleo.
\item Los operadores $\id, \tp\in DA$. Donde $\tp(x)=1$ para todo $x\in A$ e $\id$ es el operador identidad.
\end{enumerate}
Así para cualquier $A\in \Frm$ tenemos está nueva relación entre los operadores definidos hasta este momento $$NA\subseteq PA\subseteq SA\subseteq DA.$$
\end{obs}

\begin{dfn}\label{OrdenDA}
Sea $A\in \Frm$ definimos un orden para los elementos de $DA$ de la siguiente manera
$$j\leq k\Leftrightarrow j(a)\leq k(a), \mbox{ para cualquier }j,k\in DA\mbox{  y  }\forall a\in A.$$
\end{dfn}

\begin{dfn}\label{InfSupPuntual}
Sea $A\in \Frm$. Para cada $F\subseteq DA$ definimos
\begin{enumerate}
\item El \emph{ínfimo puntual} como $\bigwedge F:A\to A$,
$$(\bigwedge F)(x)=\bigwedge\{f(x)|x\in X\}.$$
\item El \emph{supremo puntual} como $\dot{\bigvee} F:A\to A$,
$$(\dot{\bigvee} F)(x)=\bigvee\{f(x)|x\in X\}.$$
\end{enumerate}
\end{dfn}

\begin{obs}
\noindent
\begin{enumerate}
\item $CA$, $NA$, $PA$, $SA$, al ser subconjuntos de $DA$, podemos dotarlos del mismo orden presentado en la Definición \ref{OrdenDA} para sus elementos, es decir, junto con $DA$ son conjuntos parcialmente ordenados. De manera particular los operadores $\id$ y $\tp$ son los elementos distinguidos 0 y 1 respectivamente en estos conjuntos ordenados.
\item $DA$, $SA$, $PA$, $CA$ y $NA$ son cerrados bajo ínfimos puntuales. Además, $SA$ y $DA$ son cerrados bajo supremos puntuales.
\item $SA$ es un marco. En consecuencia $(f\succ g)\in SA$, con $f,g\in SA$.  
\end{enumerate}
\end{obs}

En otras palabras la observación anterior nos dice que podemos dotar de una estructura de retícula completa a los operadores derivada y a los operadores estables. Además, podemos definir la implicación sobre $SA$. En consecuencia $SA$ es un marco, sin embargo, ¿esta implicación sólo se puede calcular en $SA$? La respuesta la encontramos en el siguiente lema.

\begin{lem}\label{MarcoNA}
Sean $A\in \Frm$, $f\in SA$ y $k\in NA$, entonces $(f\succ k)\in NA$. Donde la implicación es la que existe en $SA$.
\end{lem}

Observemos que hemos dotado al conjunto $NA$ de una implicación. De esta manera, lo dicho en el Lema \ref{MarcoNA} y por el comentario del párrafo anterior tenemos un esbozo de la demostración de un teorema importante de la teoría de marcos.

\begin{thm}[Teorema fundamental de la teoría de marcos]\label{TeoEnsamble}
Sea $A\in \Frm$, entonces $NA$ es un marco.
\end{thm}

Al marco $NA$ también se le conoce como \emph{el ensamble} de $A$.\\

Es momento de ver como se compartan los diferentes operadores difinidos hasta este momento bajo composiciones.

\begin{dfn}
Sean $A\in \Frm$ y $f\in DA$, definimos recursivamente sobre los ordinales 
\[
f^0=\id, \quad f^{\alpha +}=f\circ f^\alpha,\quad f^\alpha=\dot{\bigvee}\{f^\beta\mid \beta< \alpha\}
\]
cuando $\alpha$ es un ordinal límite y $\alpha +$ denota al ordinal sucesor de $\alpha$.
\end{dfn}

De esta manera producimos una cadena de derivadas 

\[
\id\leq f\leq f^2\leq f^3\leq\ldots \leq f^\alpha\leq \ldots
\]
Esta cadena se detiene en algún punto, es decir, hay un ordinal $\delta$ tal que $f^\theta=f^{\theta +}$ para cualquier $\delta\leq \theta$. Al mínimo ordinal para el cual se detiene la cadena lo denotamos por $\infty$. Por como construimos la cadena $f^\infty$ es idempotente.

\begin{lem}\label{finfinito}
Sean $A\in \Frm$ y $f\in DA$, entonces $f^\infty$ es el menor operador cerradura mayor que $f$.
\end{lem}

El siguiente resultado nos dice que, cuando consideramos al operador $f\in SA$, entonces $f^\infty\in NA$. De hecho, cuando esto ocurra, a $f^\infty$ lo llamaremos \emph{la cerradura idempotente}.\\

\begin{thm}\label{Cerraduraidempotente}
Sea $A\in \Frm$. La asignación $-^\infty\colon SA\to SA$ es un núcleo en $SA$. Además, el conjunto de puntos fijos de $-^\infty$ es $NA$.
\end{thm}

Si tomamos un elemento $a\in A$, podemos asignarle un único elemento del marco $NA$. La forma de hacerlo se explica en la siguiente definición. 

\begin{dfn}\label{Morfismoeta}
Para cualquier $A\in \Frm$,  la \emph{función $\eta_A$} se define como  
\begin{equation*}
\begin{split}
\eta_A\colon & A\to NA\\
& a\mapsto u_a
\end{split}
\end{equation*}
\end{dfn}

Este morfismo es epimorfismo y en general no es suprayectivo, de hecho:

\begin{lem}
Sea $A\in \Frm$, el morfismo $\eta_A\colon A\to NA$ es suprayectivo si y sólo si $A$ es booleano.
\end{lem}

\begin{dfn}
Decimos que un morfismo de marcos $f\colon A\to B$ resuelve \emph{el problema de la complementación booleana} en $A$ si para cualquier $a\in A$, $f(a)\in B$ tiene complemento.
\end{dfn}

Para cualquier $A\in \Frm$, $\eta_A$ resuelve el problema de la complementación booleana en $A$, pues $u_a$ es un elemento del marco $NA$.

\subsection{Núcleos espacialmente inducidos}

Para un marco espacial $A=\mathcal{O}S$ existe una clase de núcleos que contiene todos los núcleos $u_\circ$ y $v_\circ$ descritos en la subsección anterior (además de estos, contiene muchos más). Estos capturan el ``contenido espacial'' de $\mathcal{O}S$ en un sentido que veremos a continuación.

\begin{dfn}\label{Definicion5.3.1}
    Sea $S$ un espacio topológico. Para cada $E\in \mathcal{P}S$ definimos 
    \[
    [E](U)=(E\cup U)^\circ
    \]
    donde $U\in \mathcal{O}S$.
\end{dfn}

La definición anterior nos permite obtener una función $[ \_ ]\colon \mathcal{O}S\to \mathcal{O}S$. No es complicado verificar que $[E]$ es un núcleo en el marco $\mathcal{O}S$.

\begin{dfn}\label{Definicon5.3.2}
    Para un espacio topológico $S$, un núcleo en $\mathcal{O}S$ es \emph{espacialmente inducido} si este tiene la forma $[E]$ para algún $E\subseteq S$.
\end{dfn}

La razón por la cual se denomina a este núcleo espacialmente inducido se debe a que cada función continua entre dos espacios topológicos $\phi\colon T\to S$ produce un morfismo de marcos $\phi^{-1}\colon \mathcal{O}S \to \mathcal{O}T$ entre sus topologías. Este morfismo tiene el kernel $\ker(\phi^{-1})$ caracterizado por 
\[
V\subseteq \ker(\phi^{-1})(U)\Leftrightarrow \phi^{-1}(V)\subseteq \phi^{-1}(U)
\]
para $U, V\in \mathcal{O}S$. Precisamente $\ker(\phi^{-1})$ coincide con nuestro núcleo espacialmente inducido.

\begin{thm}
    Sea $\phi$ una función continua como la de antes. Sea $E=S\setminus \phi(T)$ el complemento del rango de $\phi$. Entonces $\ker(\phi^{-1})=[E]$.
\end{thm}

\begin{proof}
    Para cada $U,V\in \mathcal{O}S$ tenemos 
    \[
    \begin{split}
    V\subseteq \ker(\phi^{-1})(U) &\Leftrightarrow \phi^{-1}(V)\subseteq \phi^{-1}(U)\\
    & \Leftrightarrow (\forall t\in T)[\phi(t)\in V\Rightarrow \phi(t)\in U]\\
    & \Leftrightarrow (\forall s\in S)[s\in V\cap \phi(T)^{-1}\Rightarrow s\in U]\\
    & \Leftrightarrow (\forall s\in S)[s\in V\Rightarrow s\in E\cup U]\\
    & \Leftrightarrow V\subseteq E\cup U.
    \end{split}
    \]
    y al calcular interior tenemos que $V\subseteq \ker(\phi^{-1})(U)\Leftrightarrow V\subseteq (E\cup U)^\circ =[E](U)$.
\end{proof}

Esto muestra como un morfismo de marcos inducido espacialmente produce un núcleo inducido espacialmente. Recíprocamente, todo núcleo espacialmente inducido surge de esta manera. Para ver esto consideremos cualquier espacio $S$ y subconjunto $E\subseteq S$. Sea $T=S\setminus E$ con la topología de subespacio, así el encaje $\phi\colon T\to S$ es continuo. Entonces $E=S\setminus \phi(T)$ y por lo tanto $[E]$ es el kernel del encaje.\\

En un marco espacial, es posible determinar explícitamente la operación implicación.

\begin{lem}\label{ImplicacionTop}
    Sea $S$ un espacio topológico. La implicación en el marco espacial $\mathcal{O}S$ está dada por 
    \[
    W\succ M=(W'\cup M)^\circ
    \]
    para cada $W, M\in \mathcal{O}S$.
\end{lem}

\begin{proof}
    Para cualesquier $U, W, M\in \mathcal{O}S$ tenemos 
    \[
    U\subseteq (W\succ M)\Leftrightarrow U\cap W\subseteq M\Leftrightarrow U\subseteq (W'\cup M)\Leftrightarrow U\subseteq (W'\cup M)^\circ.
    \]
\end{proof}

Cada marco lleva sus núcleos distinguidos $u$ y $v$, ¿qué son estos para una topología?

\begin{lem}\label{UVTop}
     Para un espacio topológico $S$ tenemos que 
     \[
     i)\,u_W=[W]\quad\mbox{ y }\quad ii)\,v_W=[W']
     \]
     para cada $W\in \mathcal{O}S$.
\end{lem}

\begin{proof}
    \begin{enumerate}[$i)$]
        \item Sean $W, M\in \mathcal{O}S$. Sabemos que $u_W(M)=W\cup M=(W\cup M)^\circ=[W](M)$.
        \item Consideremos $M\in \mathcal{O}S$. Por el Lema \ref{ImplicacionTop} la implicación en $\mathcal{O}S$ está dada por $(W\succ M)=(W'\cup M)^\circ$.De aquí que 
        \[
        v_W(M)=(W\succ M)=(W'\cup M)^\circ=[W'](M).
        \]
    \end{enumerate}
\end{proof}

Cada subconjunto $E$ de un espacio $S$ determina un núcleo $[E]$ en la topología. Sin embargo, los núcleos $[E]$ no necesitan determinar al subconjunto $E$.\\

\subsection{El funtor $N( \_ )$}

En esta subsección veremos que $\eta_A\colon A\to NA$ define un funtor. Esto lo hacemos al verificar que la asignación $a\mapsto u_a$ proporciona cierta propiedad universal. Antes de eso mencionamos un par de observaciones.\\

Sabemos que los núcleos $u_a$ y $v_a$ son complementos entre si en $NA$, es decir, el encaje $\eta_A$ crea elementos complementados para elementos de $A$. Además, sabemos que para todo $j\in NA$
\[
j=\bigvee\{u_{j(a)}\wedge v_a\mid a\in A\}.
\]

\begin{lem}\label{Lema6.2.1}
    Para cada $A\in \Frm$ y morfismos de marcos $g, h\colon NA\to B$, si $g\circ \eta_A=h\circ \eta_A$, entonces $g=h$. En otras palabras, $\eta_A$ es un epimorfismo.
\end{lem}

\begin{proof}
    Consideremos los morfismos $g, h$ tales que $g\circ \eta_A=h\circ \eta_A$. De esta manera $g(u_a)=h(u_a)$ para todo $a\in A$. Como $u_a$ es complementado por $v_a$, es decir, $u_a\wedge v_a=\id$ y $u_a\vee v_a=\tp$, se puede verificar que $g(v_a)=h(v_a)$.\\

    Ahora, consideremos $j\in NA$, entonces 
    $j=\bigvee\{u_{j(a)}\wedge v_a\mid a\in A\}$. Así
    \[
    \begin{split}
    g(j)& =\bigvee\{g(u_{j(a)}\wedge g(v_a)\mid a\in A\}
    \\
    & =\bigvee\{h(u_{j(a)}\wedge h(v_a)\mid a\in A\}=h(j).
    \end{split}
    \]
    Por lo tanto $g=h$.
\end{proof}

Un morfismo $f\colon A\to B$ \emph{resuelve el problema de complementación} para $A$ si para $a\in A$, $f(a)$ tiene complemento en $B$.

\begin{thm}\label{Teorema6.2.2}
    Para cada marco $A$, el morfismo $\eta_A\colon A\to NA$ resuelve universalmente el problema de la complementación para $A$. Es decir, para cada morfismo $f\colon A\to B$ existe un único morfismo $f^\#$ tal que el siguiente diagrama conmuta.
    \[\begin{tikzcd}
	A && B \\
	& {} \\
	NA
	\arrow["f", from=1-1, to=1-3]
	\arrow["{\eta_A}"', from=1-1, to=3-1]
	\arrow["{f^\#}"', from=3-1, to=1-3]
\end{tikzcd}\]
\end{thm}

\begin{proof}
    Por el Lema \ref{Lema6.2.1}, $\eta_A$ es un epimorfismo, de esta manera, de existir el morfismo $f^\#$, este debe ser único.\\

    Para cada $j\in NA$, consideremos 
    \[
    f^\#(j)=\bigvee\{f(j(x))\wedge f(x)'\mid x\in A\},
    \]
    donde $f(x)'$ es el complemento de $f(x)$ es $B$. Se puede verificar que el morfismo $f^\#\colon NA\to B$ es monótono y además es un $\wedge-$morfismo.\\

    Para $b\in B$, consideremos la siguiente composición
    \[
    A\to B\to [b,1_B],
    \]
    donde $[b, 1_B]$ es un intervalo en $B$. Sea $\langle b\rangle$ el kernel de la composición anterior, de esta manera
    \[
    y\leq \langle b\rangle(x)\Leftrightarrow b\vee f(y)\leq b\vee f(x)\Leftrightarrow f(y)\leq b\vee f(x)
    \]
    para todo $x,y\in A$. Verifiquemos que el morfismo $f_b\colon B\to NA$, dado por la asignación $b\mapsto \langle b\rangle$ es el adjunto derecho de $f^\#$, es decir, debemos verificar que $f^\#(j)\leq b\Leftrightarrow j\leq \langle b\rangle$ para $j\in NA$ y $b\in B$.\\

    Supongamos que $f^\#(j)\leq b$ y sea $x\in A$ tal que $y=j(x)$. De esta manera
    \[
    f(y)\wedge f(x)'\leq f^\#(j)\leq b\Leftrightarrow j(x)=y\leq f(y)\leq b\vee f(x),
    \]
    es decir $j(x)\leq \langle b\rangle(x)$.\\

    De manera reciproca, supongamos que $j\leq \langle b\rangle$ y consideremos $x\in A$. De esta manera $j(x)\leq \langle b\rangle(x)$, de modo que $f(j(x))\leq b\vee f(x)$. Así $f(j(x))\wedge f(x)'\leq b$. Como lo anterior se cumple para todo $x\in A$, en particular se cumple para $f^\#$, es decir, $f^\#(j)\leq b$.\\

    Lo anterior también muestra que $f^\#$ es un morfismo de marcos.\\

    Por último, veamos que el diagrama conmuta. Sean $x, a\in A.$ Así
    \[
    f(a\vee x)\wedge f(x)'=(f(a)\vee f(x))\wedge f(x)'=f(a)\wedge f(x)'\leq f(a).
    \]
    Además, $f(a\vee 1)\wedge f(1)'=f(a)$. Por lo tanto 
    \[
    (f^\#\circ \eta_A)(a)=f^\#(u_a)=\bigvee\{f(a\vee x)\wedge f(x)'\mid x, a\in A\}=f(a)
    \]
    que es lo que queríamos.
\end{proof}

La prueba del Teorema \ref{Teorema6.2.2}, de manera indirecta, proporciona un funtor. En este punto es importante mencionar lo siguiente: \textbf{Toda propiedad universal define un funtor}.

\begin{thm}\label{Teorema6.2.3}
    La asignación $A\mapsto NA$ es la relación entre objetos del funtor $N( \_ )\colon \Frm \to\Frm$ y el morfismo $\eta_A\colon A\to NA$ es una transformación natural. En otras palabras, para todo $A, B\in \Frm$ y cada morfismo $f\in \Frm$, el siguiente diagrama conmuta para un único morfismo $Nf$.
    \[\begin{tikzcd}
	A & NA \\
	B & NB
	\arrow["{\eta_A}", from=1-1, to=1-2]
	\arrow["f"', from=1-1, to=2-1]
	\arrow["Nf", from=1-2, to=2-2]
	\arrow["{\eta_B}"', from=2-1, to=2-2]
\end{tikzcd}\]
\end{thm}

\begin{proof}
    En el diagrama anterior, la imagen de cada elemento de $A$ bajo la composición $\eta_B\circ f$ es complementada en $NB$, y así, por el Teorema \ref{Teorema6.2.2} existe un único morfismo $Nf\colon NA\to NB$ que hace conmutar el cuadrado.\\

    Resta verificar que este es un funtor, es decir, para
    \[\begin{tikzcd}
	A & B & C
	\arrow["f", from=1-1, to=1-2]
	\arrow["g", from=1-2, to=1-3]
\end{tikzcd}\]
se cumple que $N(g\circ f)=Ng\circ Nf$. Notemos que el diagrama 
\[\begin{tikzcd}
	A & B & C \\
	NA & NB & NC
	\arrow["f", from=1-1, to=1-2]
	\arrow["{\eta_A}"', from=1-1, to=2-1]
	\arrow["g", from=1-2, to=1-3]
	\arrow["{\eta_B}"', from=1-2, to=2-2]
	\arrow["{\eta_C}"', from=1-3, to=2-3]
	\arrow["Nf"', from=2-1, to=2-2]
	\arrow["Ng"', from=2-2, to=2-3]
\end{tikzcd}\]
conmuta. De esta manera $Ng\circ Nf$ es la única flecha que hace conmutar el rectángulo. Por lo tanto $N(g\circ f)=Ng\circ Nf$.
\end{proof}

De manera adicional, tenemos la siguiente relación entre el funtor $N$ y los núcleos abiertos y cerrados.

\begin{cor}\label{Corolario6.2.4}
    Para $f\in \Frm(A, B)$ y $a\in A$ se cumple lo siguiente:
    \begin{enumerate}
        \item $(Nf)u_a=u_{f(a)}$.
        \item $(Nf)v_a=v_{f(a)}$.
    \end{enumerate}
\end{cor}

\begin{proof}
    \begin{enumerate}
        \item Es la asignación del cuadro en el Teorema \ref{Teorema6.2.3}.
        \item Sabemos que los núcleos $u_a$ y $v_a$ son complementos en $NA$. Además, $Nf$ es un morfismo de marcos. Así 
        \[
        \begin{split}
        u_{f(a)}\wedge (Nf)(v_a)&=(Nf)(u_a\wedge v_a)=\id\\  u_{f(a)}\vee (Nf)(v_a)&=(Nf)(u_a\vee v_a)=\tp.
        \end{split}
        \]
        de esta manera $(Nf)(v_a)$ es el complemento de $u_{f(a)}$ en $NA$, pero el complemento es único, es decir, $(Nf)(v_a)=v_{f(a)}$.
    \end{enumerate}
\end{proof}


\section{Aspectos topológicos}\label{Topología}

Si $S$ es un espacio topológico, entonces este puede cumplir distintas propiedades (axiomas de separación, compacidad, sobriedad, entre otras). En este trabajo, una de las construcciones en las que más centramos nuestra atención es la del \emph{espacio de parches}. La anterior es motivada por la siguiente situación: si $S$ es un espacio $T_2$, entonces todo conjunto compacto (saturado) es cerrado. 
Lo primero que podemos preguntarnos es: ¿qué pasa si el espacio no es $T_2$? El espacio de parches, como veremos más adelante, soluciona este ``defecto''.\\
    
En esta sección se mencionan los antecedentes topológicos que se necesitan para comprender que es el espacio parches. 
De igual manera mencionamos algunas otras propiedades topológicas y la relación que existe entre todas estas. La información aquí presentada es parte de la tesis 
doctoral de Sexton (ver \cite{R.S.}).

\subsection{Espacios sobrios}\label{Espacios sobrios}

Las cuestiones de la sobriedad la veremos desde dos enfoques. Primero veremos la versión que se presenta en la 
tesis de Sexton. La segunda será presentada brevemente en el Capítlo \ref{Axiomas de separacion}.

\begin{dfn}\label{irreducible}
    Un conjunto cerrado $X$ no vacío es \emph{irreducible} en $S$ si para cada $U, V$ abiertos disjuntos se cumple que 

    \[
    U\cap X\neq \emptyset, V\cap X\neq \emptyset \Rightarrow U\cap V\cap X\neq \emptyset.
    \]
\end{dfn}

\begin{dfn}\label{sobrio}
    Para un espacio $S$ decimos que este es \emph{sobrio} si es $T_0$ y cada conjunto $X$ cerrado irreducible es la cerradura de un único punto, es decir,

    \[
    X=\overline{\{x\}}
    \]

    con $x\in S$.
\end{dfn}

Se puede verificar que si $S$ es un espacio $T_2$, entonces este es sobrio, pero las propiedades de sobriedad y $T_1$ no son comparables.\\

Si tenemos un espacio que no es sobrio, entonces existe una manera de ``sobrificarlo''.

\begin{dfn}\label{Reflexion sobria}
    Sea $S$ un espacio topológico. La \emph{reflexión sobria} de $S$, denotada por $^+S$, es el espacio topológico cuyos puntos son los conjuntos cerrados irreducibles de $S$. Para $U\in \mathcal{O}S$ sea $^+U\subseteq S$ dado por 

    \[
    X\in\, ^+U \Leftrightarrow U\cap X\neq \emptyset
    \]
para cada $X\in\,^+S$. 
\end{dfn}

De esta manera tenemos el espacio topológico $(^+S, \mathcal{O}^+S)$, donde 

\[
\mathcal{O}^+S=\{^+U\mid U\in \mathcal{O}S\}
\]

También, tenemos un morfismo de marcos $\_ ^+\colon \mathcal{O}S\to \mathcal{O} ^+S$ dado por la reflexión sobria y la función continua de $S$ a $^+S$ dada por la asignación $p\mapsto p^-$.

\begin{lem}
    Para cada función continua $\Phi\colon S\to T$ de un espacio arbitrario $S$ a un espacio sobrio $T$, existe una única función continua $^+\Phi\colon ^+S\to T$ tal que el siguiente diagrama conmuta

    \[\begin{tikzcd}
	S && T \\
	& {^+S}
	\arrow["\Phi", from=1-1, to=1-3]
	\arrow[from=1-1, to=2-2]
	\arrow["{^+\Phi}"', from=2-2, to=1-3]
\end{tikzcd}\]
\end{lem}

\begin{lem}
    Si un espacio $S$ tiene reflexión sobria que es $T_1$, entonces $S$ es sobrio y $T_1$.
\end{lem}

\begin{proof}
    Notemos que $S\subseteq\, ^+S$, donde $S$ tiene la topología del subespacio. Supongamos que $X\subseteq S$ es un conjunto cerrado irreducible de $S$ y consideremos la clausura $X^-$ de $X$ en $^+S$. Para $U\in \mathcal{O}^+S$ tenemos
    \[
    X^{-}\cap U\Rightarrow X\cap U\Rightarrow X\cap (S\cap U),
    \]
    es decir $X^-$ es cerrado irreducible en $^+S$. Ahora, si $^+S$ es $T_1$ y sobrio, entonces $X^{-}=\{p\}$ para algún $p\in ^+S$ y por lo tanto $X=\{p\}$.
\end{proof}

Los siguiente resultados involucran al espacio de puntos de un marco y a espacios sobrios.
\begin{lem}
    El espacio de puntos $\pt A$ de un marco $A$ es sobrio
\end{lem}

\begin{lem}
    Sea $S$ un espacio topológico. El espacio de puntos de $\mathcal{O}S$ es la reflexión sobria de $S$.
\end{lem}

Para ver esto, notemos que los subconjuntos cerrados irreducibles de $S$ son precisamente los complementos de los elementos $\wedge$-irreducibles. De esta manera encontramos que 

\[
S\to \pt(\mathcal{O}S
\]
\[
p\mapsto \overline{\{p\}}'
\]
es el mapeo que da la reflexión.

\subsection{Conjuntos saturados}\label{Conjuntos saturados}

Si consideramos $p, q\in S$ puntos en el espacio $S$, no tenemos manera de compararlos. Para solucionar esto, se define el ``orden de especialización''.

\begin{dfn}\label{O. especializacion}
    Sea $S$ un espacio topológico. El \emph{orden de especialización} en $S$ es la relación ``$\sqsubseteq$'' dada por 
    \[
    p\sqsubseteq q \Leftrightarrow p^- \subseteq q^-
    \]
    donde $p, q\in S$. De manera equivalente tenemos que $ p\sqsubseteq q \Leftrightarrow p\in q^-$.
\end{dfn}

Esta comparación es un preorden y esta es un orden parcial precisamente cuando el espacio es $T_0$. Si el espacio es $T_1$, entonces el orden está dado por la igualdad.\\

Recordemos que si tenemos un marco arbitrario, por lo visto en \ref{Epuntos}, podemos asignarle a este un espacio topológico por medio de su espacio de puntos.

\begin{lem}
    Sea $A$ un marco con espacio de puntos $\pt A$. El orden de especialización en $S$ es el orden inverso del orden heredado de $A$.
\end{lem}

Usando este orden parcial en un espacio $T_0$, podemos introducir el concepto de saturación.

\begin{dfn}\label{SupInf}
    Sea $(S, \leq)$ un conjunto parcialmente ordenado. Para cada $E\subseteq S$, $\downarrow{E}$ y $\uparrow{E}$ son, respectivamente, la sección inferior y la sección superior generada por $E$, es decir,

    \[
    \downarrow{E}=\{x\mid (\exists e\in E) [x\leq e]\}\quad \mbox{ y } \quad \uparrow{E}=\{x\mid (\exists e\in E) [x\geq e]\}
    \]
    respectivamente. Decimos que $\uparrow{E}$ es la \emph{saturación} de $E$ y $E$ es \emph{saturado} si $E=\uparrow{E}$.
\end{dfn}

Para $p\in S$ denotamos $\downarrow{p}$ por $\downarrow{\{p\}}$ y $\uparrow{p}$ por $\uparrow{\{p\}}$. Se puede verificar que $\uparrow\uparrow{E}=\uparrow{E}$ y de esta manera la saturación de $E$ es un conjunto saturado.\\

Si $S$ es cualquier espacio topológico y $p\in S$, tenemos que $\downarrow{p}=p^-$. Sin embargo, esto no es cierto para subconjuntos arbitrarios de $S$. Usualmente $\downarrow{E}\neq E^-$ aunque hay una clase de topologías para las cuales $\downarrow{(\_)}$ y $(\_)^-$ coinciden (las topologías de Alexandorff).\\

En un espacio topológico, la saturación de un subconjunto se puede obtener sin hacer referencia al orden de especialización.

\begin{lem}\label{LemSup}
    Sea $S$ un espacio topológico. Para cada subconjunto $E\subseteq S$ tenemos que
    \[
    \uparrow{E}=\bigcap\{U\in \mathcal{O}S\mid E\subseteq U\}
    \]
\end{lem}

\begin{proof}
    Consideremos $x\in \uparrow{E}$, entonces $\exists\, e\in E$ tal que $e \sqsubseteq x$, es decir, $e^-\subseteq x^-$. De aquí que para $U\in\mathcal{O}S$, si $e\in U$ entonces $x\in U$. Por lo tanto
    \[
    x\in \bigcap \{U\in \mathcal{O}S\mid e\in U\}\subseteq \bigcap \{U\in \mathcal{O}S\mid E\subseteq U\},
    \]
    es decir, $\uparrow E\subseteq \bigcap \{U\in \mathcal{O}S\mid E\subseteq U\}$.\\

    Para la otra contención, consideremos $x\in \bigcap\{U\in \mathcal{O}S\mid E\subseteq U\}$. Notemos que si $E\subseteq U$, entonces $x\in U$ para todo $U\in \mathcal{O}S$. De aquí que al menos existe algún $e\in E$ tal que $e^-\subseteq x^-$, es decir, $e\sqsubseteq x$. 
\end{proof}

De esta manera cada subconjunto abierto es saturado. Sin embargo, el reciproco no siempre ocurre, por lo general, hay muchos conjuntos saturados que no son abiertos. Por ejemplo, en un espacio $T_1$ todos los conjuntos son saturados. En un espacio de Alexandroff ocurre lo contrario, cada saturado es abierto.\\

En cualquier conjunto parcialmente ordenado, la familia de conjuntos saturados es cerrada bajo uniones e intersecciones arbitrarias. En particular, los conjuntos saturados forman la topología de Alexandroff.

\begin{dfn}\label{Alexandroff}
    Sea $(S, \leq)$ un orden parcial. La \emph{topología de Alexandroff} en $S$ es la topología que consta de todos los conjuntos saturados
\end{dfn}

\begin{dfn}
    Una cubierta abierta para un conjunto $A$ es una colección $\mathcal{U}$ de conjuntos abiertos tales que 
    \[
    A\subseteq \bigcup \mathcal{U}
    \]
    se cumple.

    \begin{itemize}
        \item Una subcubierta de una cubierta abierta $\mathcal{U}$ para $A$ es una subcolección $\mathcal{V}\subseteq \mathcal{U}$ la cual forma una cubierta abierta para $A$.
        \item Una cubierta $\mathcal{U}$ es dirigida si esta es $\subseteq$-dirigida, es decir, para cada $U,V\in \mathcal{U}$, existe algún $W\in \mathcal{U}$ tal que $U\cup V\subseteq W$.
        \item Un conjunto $X$ de un espacio topológico $S$ es compacto si cada cubierta abierta de $X$ tiene una subcubierta abierta finita.
    \end{itemize}
\end{dfn}

A veces es conveniente usar una formulación equivalente de compacidad. Reescribimos la definición en términos de cubiertas abiertas dirigidas. 

\begin{lem}
    Sea $S$ un espacio topológico. Un conjunto $X\subseteq S$ es compacto si y solo si para cada cubierta abierta dirigida $\mathcal{W}$ existe algún $W\in \mathcal{W}$ tal que $X\subseteq W$.
\end{lem}

\begin{proof}
\begin{description}
    
    \item[$\Rightarrow )$] Consideremos un subconjunto compacto $X$ y una cubierta dirigida de abiertos $\mathcal{W}$, es decir $X\subseteq\bigcup\mathcal{W}$. De aquí que debe existir una cubierta finita, digamos $\mathcal{V}$. Al ser $\mathcal{W}$ dirigida, tenemos que $\bigcup\mathcal{V}\in \mathcal{W}$ y al ser $X$ compacto se cumple que $X\subseteq \bigcup \mathcal{V}$.
    
    \item[$\Leftarrow )$] Consideremos $X\subseteq S$ tal que para cada cubierta abierta dirigida $\mathcal{W}$ existe $W\in \mathcal{W}$, con $X\subseteq W$. Notemos que si $\mathcal{U}$ es cualquier cubierta abierta, entonces agregando todas las uniones finitas obtenemos una cubierta abierta dirigida $\mathcal{U}_0$. Por hipótesis, existe $U\in\mathcal{U}_0$ tal que $X\subseteq U$. Luego, al ser $U$ solo uniones finitas de $\mathcal{U}$, esta es una subcubierta finita de $X$. Por lo tanto $X$ es compacto.
    \end{description}
    \end{proof}
    
\begin{dfn}\label{Csaturado}
    Para un espacio topológico $S$, denotamos por $\mathcal{Q}S$ a la colección de conjuntos compactos saturados de $S$.
\end{dfn}

El conjunto vacío está en $\mathcal{Q}S$. De igual manera para $p\in S$, la saturación $\uparrow{p}$ está en $\mathcal{Q}S$. Esto puede ser generalizado.

\begin{lem}
    Sea $K$ un subconjunto compacto de un espacio $S$. La saturación $\uparrow{K}$ está en $\mathcal{Q}S.$
\end{lem}

\begin{proof}
    Debemos probar que para $K\subseteq S$ un conjunto compacto se cumple que $\uparrow{K}$ también lo es. Consideremos $\mathcal{U}$ una cubierta abierta dirigida de $\uparrow K$. De aquí que $\mathcal{U}$ es también una cubierta abierta dirigida de $K$. Luego, como $K$ es compacto existe $U\in \mathcal{U}$ tal que $K\subseteq U$. Se puede verificar que $U$ es un conjunto saturado que contiene a $K$, de aquí que $\uparrow K\subseteq U$. Por lo tanto $\uparrow K$ es compacto.
\end{proof}

De esta manera tenemos tres familias distinguidas de subconjuntos de $S$, los abiertos $\mathcal{O}S$, los cerrados $\mathcal{C}S$ y los compactos saturados $\mathcal{Q}S$.\\

Sabemos que la unión de dos conjuntos compactos es compacta. De igual manera, se puede comprobar que la unión de dos conjuntos saturados es saturada. Esto nos da como resultado el siguiente lema.

\begin{lem}\label{Union compactosaturado}
    La unión de dos conjuntos saturados compactos es saturada compacta.
\end{lem}

Por otro lado, la unión de una familia arbitraria de conjuntos compactos saturados no necesita ser compacta saturada. Para ver esto, consideremos la unión de todos los $\uparrow{p}$ para cada punto $p$ de un espacio $S$ que no es compacto.\\

Tampoco es el caso que la intersección de cualesquiera dos conjuntos compactos saturados deba ser compacta saturada.\\

El siguiente resultado es el que nos motiva a estudiar lo que en la Sección \ref{Parche puntos} aparece como \emph{construcción del espacio de parches}.

\begin{lem}\label{compacto saturado}
    En un espacio $T_2$ cada conjunto compacto saturado es cerrado.
\end{lem}

\begin{proof}
    Consideremos un espacio $S$ que es $T_2$, en consecuencia $S$ es $T_1$ y así todo conjunto del espacio es saturado. Consideremos $Q\subseteq S$ un conjunto compacto. veamos que $Q$ es cerrado. Para ello consideremos $p\notin Q$ y veamos que para $\mathcal{U}(p)$ una vecindad abierta de $p$ se cumple que 
    \[
    Q\cap \mathcal{U}(p)=\emptyset.
    \]
    Consideremos $q\in Q$ y al ser $S$ un espacio $T_2$ tenemos que existen $U_q, V_q\in \mathcal{O}S$ tales que 
    \[
    p\in U_q,\quad q\in V_q, \quad U_q\cap V_q=\emptyset.
    \]
    De aquí que $\{V_q\mid q\in Q\}$ forma una cubierta abierta para $Q$ y, por la compacidad, podemos extraer una subcubierta finita $\{V_i\mid i\in \mathcal{I}\}$, donde $\mathcal{I}$ es un subconjunto finito de $Q$. Considerando $\mathcal{U}(p)=\{U_i\mid i\in \mathcal{I}\}$ tenemos la vecindad de $p$ que buscábamos.
\end{proof}

El siguiente resultado nos da una caracterización similar a la regularidad definida para espacios topológicos, pero en lugar de considerar conjuntos en $\mathcal{C}S$ los consideramos en $\mathcal{Q}S$. 

\begin{cor}
    Sea $S$ un espacio $T_2$. Para un punto $p$ y conjunto $Q$ compacto (saturado), que no contiene a $p$ existe conjuntos abiertos $U, V$ tales que
    \[
    p\in U, \qquad Q\subseteq V, \qquad U\cap V=\emptyset.
    \]
\end{cor}

\subsection{La topología frontal (topología de Skulla)}.

La topología frontal de un espacio $S$ es la topología más fina que hace a todos los cerrados originales conjuntos \emph{clopen} (conjuntos que son cerrados y abiertos al mismo tiempo). Esto puede parecer una topología muy poco interesante, pero veremos que tiene alguna relevancia para las construcciones sin puntos que se verán más adelante.

\begin{dfn}\label{Frontal}
    El espacio frontal, denotado por $^fS$ de un espacio topológico $S$ tiene los mismos puntos que $S$, pero la topología más fina $\mathcal{O}^fS$ generada por 
    \[
    \{U\cap X\mid U\in \mathcal{O}S, X\in \mathcal{C}S\}.
    \]
\end{dfn}

Se puede verificar que $\{U\cap p^-\mid U\in \mathcal{O}S, p^-\in \mathcal{C}S\}$ es también una base para la topología frontal en $S$. De hecho los conjuntos $U\cap p^-$ para $U\in \mathcal{O}S$ forman las vecindades abiertas frontales de $p\in S$.\\

Para $E\subseteq S$, escribimos $E^\Box$ y $E^=$ para el interior y la clausura frontal de $E$, respectivamente. Estos están relacionados como sigue 
\[
E^\circ\subseteq E^\Box \subseteq E\subseteq E^=\subseteq E^-
\]
además, pueden ser distintos.\\

Notemos que $p\in E^=$ si y solo si para todo $U\in \mathcal{O}S$, si $p\in U$, entonces $E\cap U\cap p^-\neq \emptyset$ se cumple. Además, $^fS$ no es el espacio discreto. Para complementar esto tenemos el siguiente resultado.

\begin{lem}
    Sea $S$ un espacio topológico
    \begin{itemize}
        \item Si $S$ es $T_1$, entonces $^fS$ es discreto.
        \item Si $^fS$ es discreto, entonces $S$ es $T_0$.
        \item Si $S$ es $T_0$, entonces $^{ff}S$ es discreto.
    \end{itemize}
\end{lem}

En la tercera parte del lema anterior, se usa el segundo espacio frontal $^{ff}S$. Hay ejemplos de espacios sobrios $S$ tales que $\mathcal{O}S$, $\mathcal{O}^fS$, $\mathcal{O}^{ff}S=\mathcal{P}S$ son distintos.

\begin{thm}\label{Sobriofrontal}
    Si $S$ es un espacio sobrio, entonces también lo es $^fS$.
\end{thm}

\begin{proof}
    Observemos que al ser $S$ un espacio sobrio, este es $T_0$. De aquí que $^fS$ también es $T_0$. De está manera debemos probar que cada conjunto cerrado irreducible en $^fS$ es la cerradura de un punto.\\

    Consideremos $F\subseteq S$ un conjunto cerrado tal que es irreducible en $^fS$. Entonces para cada $U, V\in \mathcal{O}S$ y $X, Y\in \mathcal{C}S$ se cumple
    \[
    F\cap X\cap U\neq \emptyset,\; F\cap Y\cap V\neq \emptyset\; \Rightarrow \; F\cap X\cap Y\cap U\cap V\neq \emptyset.
    \]
    Veamos ahora que $F^-$ es cerrado irreducible en $S$. Supongamos que $F^-\cap U\neq \emptyset$ y $F^-\cap V\neq \emptyset$. Por las propiedades de la clausura tenemos que $F\cap U\neq \emptyset$ y $F\cap V\neq \emptyset$ y así, por ser irreducible en $^fS$, se cumple que $F\cap U\cap V\neq\emptyset$. Por lo tanto $F^-\cap U\cap V\neq \emptyset$, es decir $F^-$ es cerrado irreducible en $S$.\\

    Por hipótesis $S$ es sobrio, de aquí que $F^-=p^-$ para algún $p\in S$. Observemos que si $p\in F$ se cumple que $F^= =F$. Supongamos que $p\in U\in \mathcal{O}S$, entonces $F^-\cap U\neq \emptyset$ y por lo tanto $F\cap U\neq \emptyset$. Así cada vecindad abierta del punto $p$ de la forma $U\cap p^-$ también intersecta a $F$. Luego por la definición de la cerradura frontal tenemos que $p\in F^==F$.
\end{proof}

Recordando lo visto en la Sección \ref{Espacios sobrios}, si consideramos un espacio $T$ que es $T_0$, tenemos que 
\[\begin{tikzcd}
	T & {^+T}
	\arrow[hook, from=1-1, to=1-2]
\end{tikzcd}\]
En particular, si tenemos $T\subseteq S$ para algún espacio sobrio $S$, entonces $T\subseteq ^+T\subseteq S$. Podemos identificar cual es este espacio $S$ que es sobrio.

\begin{thm}\label{Frontalsobrio}
    Sean $S$ un espacio sobrio y $T\subseteq S$ un subconjunto arbitrario. Entonces $T=T^=$ en $S$ precisamente cuando, como subespacio, $T$ es sobrio.
\end{thm}

\begin{proof}
\begin{description}
    \item[$\Rightarrow )$] Primero, supongamos que $T$ es cerrado frontal. Notemos que, como subespacio, los cerrados de $T$ son de la forma $T\cap X$, donde $X\in \mathcal{C}S$. Además dicho conjunto es la clausura de un punto si y solo si $T\cap X=T\cap p^-$ para algún $p\in T$.\\

    Supongamos que $T\cap X$ es cerrado irreducible en $T$. De esta manera, para $U, V\in \mathcal{O}S$ se cumple que 
    \[
    T\cap X\cap U\neq \emptyset,\quad T\cap X\cap V\neq \emptyset \quad \Rightarrow \quad T\cap X\cap U\cap V\neq \emptyset
    \]
    Además, para los mismos $U, V$ se cumple que 
    \[
    \begin{split}
    (T\cap X)^-\cap U\neq \emptyset & \Rightarrow T\cap X\cap U\neq \emptyset\\
    (T\cap X)^-\cap V\neq \emptyset & \Rightarrow T\cap X\cap V\neq \emptyset
        \end{split}
    \]
    lo cual implica que $(T\cap X)^-$ es cerrado irreducible en $S$. Por hipótesis $S$ es sobrio, de aquí que $(T\cap X)^-=p^-$ para un único $p\in (T\cap X)^-$. Notemos que $p\in (T\cap X)^-\subseteq X^-=X$, es decir, $p^-\subseteq X$. Luego 
    \[
    T\cap p^-\subseteq T\cap X\subseteq T\cap (T\cap X)^-= T\cap p^-,
    \]
    es decir, $T\cap X=T\cap p^-$. Por lo tanto, basta mostrar que $p\in T$.\\

    Para cada $U\in \mathcal{O}S$, como $(T\cap X)^-=p^-$. tenemos 
    \[
    \begin{split}
        p\in U\; &\Rightarrow \; (T\cap X)^-\cap U\neq \emptyset\\
        & \Rightarrow \; T\cap X\cap U\neq \emptyset\\
        &\Rightarrow \; T\cap U\cap p^-\neq \emptyset
    \end{split}
    \]
    y así $p\in T^==T$.
    \item[$\Leftarrow )$] Supongamos que, como subespacio, $T$ es sobrio. Demostraremos que $T^=\subseteq T$.\\

    Consideremos cualquier punto $p\in T^=$, entonces $p\in U$ si y solo si $T\cap U\cap p^-\neq \emptyset$ para cada $U\in \mathcal{O}S$. En particular, para $U=S$ tenemos que $T\cap p^-\neq \emptyset$ y así este conjunto es cerrado en $T$. Veamos que  $T\cap p^-$ es iireducible. Consideremos $U, V\in \mathcal{O}S$, entonces
    \[
    T\cap p^-\cap U\neq \emptyset,\, T\cap p^-\cap V\neq \emptyset \Rightarrow p\in U\cap V,
    \]
    de modo que $T\cap p^-\cap U\cap V\neq \emptyset$ .Por hipótesis, $T$ es sobrio, entonces $T\cap p^-=T\cap q^-$ para algún $q\in T\cap p^-$. En particular, $q\in p^-$. Luego para cada $U\in \mathcal{O}S$ tenemos 
    \[
    \begin{split}
        p\in U & \Rightarrow T\cap p^-\cap U\neq \emptyset\\
        & T\cap q^- \cap U\neq \emptyset\\
        & q^-\cap U\neq \emptyset\\
        & q\in U,
    \end{split}
    \]
    es decir, $p\in q^-$ y por lo tanto, $p^-=q^-$. Como $S$ es $T_0$, entonces $p=q\in T$. Luego $T^=\subseteq T$ lo cual implica que $T^==T$.
\end{description}
\end{proof}

\begin{cor}
    Sea $T$ un espacio $T_0$ y supongamos que $T\subseteq S$ para algún espacio sobrio $S$. Entonces la reflexión sobria $^+T$ de $T$ es la clausura frontal de $T$ en $S$.
\end{cor}

\begin{proof}
    Sabemos que $T\subseteq ^+ T\subseteq S$. Además, por el Teorema \ref{Frontalsobrio}, tenemos que $^+T$ es cerrado frontal. Así $T^=\subseteq\, ^+T$. Similarmente, sabemos que $T\subseteq T^=\subseteq S$, por el Teorema \ref{Frontalsobrio}, $T^=$ es sobrio. Así $^+T\subseteq T^=$.
\end{proof}

\subsection{El espacio de parches}\label{Parche puntos}

Como vimos en el Lema \ref{compacto saturado}, los espacios $T_2$ cumplen que todo conjunto compacto saturado es cerrado, pero el regreso no se cumple.  En esta sección daremos el nombre de \emph{empaquetados} a los espacios que cumplen con esta propiedad. Nuestro objetivo será el encontrar una manera de empaquetar cualquier espacio arbitrario.

\begin{dfn}\label{empaquetado}
    Un espacio topológico $S$ es \emph{empaquetado} si todo conjunto compacto saturado es cerrado. 
\end{dfn}

Observemos que la propiedad de ser empaquetado es más débil que ser $T_2$, la pregunta que surge es ¿qué relación existe entre empaquetado y $T_1$?

\begin{lem}\label{Empaquetado y T1}
    Un espacio topológico que es $T_0$ y empaquetado es $T_1$.
\end{lem}

\begin{proof}
    Sabemos que $(\uparrow p)$ es saturado y para ver que $(\uparrow p)$ es compacto notemos que 
    \[
    (\uparrow p)=\bigcap\{U\in \mathcal{O}S\mid (p)\subseteq U\},
    \]
    además $(\uparrow p)\subseteq \bigcup U$ y se le puede extraer una cubierta finita. Por lo tanto $(\uparrow p)$ es compacto saturado.\\

    Al ser $(\uparrow p)$ compacto saturado, por hipótesis de empaquetado, $(\uparrow p)$ es cerrado. Ahora, consideremos dos puntos distintos $p, q$ tales que $p\sqsubseteq q$ no se cumple. Entonces, por la definición del orden de especialización tenemos que 
    \[
    p^-\nsubseteq q^- \Leftrightarrow p\in (q^-)'\mbox{ y } q\in (\uparrow p)',
    \]
    donde $(q^-)'$, $(\uparrow p)'$ son abiertos y $(q^-)' \cap (\uparrow p)'=\emptyset$. Por lo tanto tenemos una separación $T_1$ de $p$ y $q$.
\end{proof}

Podemos tener espacios empaquetados que no son $T_0$, estos no serán abordados aquí, pues suponemos que todos nuestros espacios son al menos $T_0$, pero por lo visto en el Lema \ref{Empaquetado y T1}, $T_0+$empaquetado se encuentra entre $T_2$ y $T_1$.\\

Algunos espacios tienen una propiedad aún más fuerte que empaquetado. Es útil tener un nombre para esto.

\begin{dfn}
    $S$ es \emph{estrictamente empaquetado} si todo conjunto compacto saturado es finito. 
\end{dfn}

De esta manera un espacio no es empaquetado si tiene al menos un conjunto compacto saturado que no es cerrado. En otras palabras, no tiene suficientes conjuntos cerrados, o equivalentemente, suficientes conjuntos abiertos. Podemos corregir este defecto agregando a la topología nuevos conjuntos abiertos para formar un topología más grande. \\

Para un espacio $S$ consideremos la familia 
\[
pbase=\{U\cap Q'\mid U\in \mathcal{O}S, Q\in \mathcal{Q}S\}
\]
que por el Lema \ref{Union compactosaturado} es cerrada bajo intersecciones binarias y por lo tanto forman una base para una nueva topología.\\

Al considerar $Q=\emptyset$ tenemos que la $pbase$ contiene a la topología original y al dejar $U=S$ vemos que la $pbase$ contiene a los complementos de cada conjunto compacto saturado.

\begin{dfn}
    Para un espacio topológico $S$, consideramos $^pS$ el espacio con los mismos puntos que $S$ y la topología $\mathcal{O}^PS$ generada por la $pbase$.
\end{dfn}

En otras palabras, $\mathcal{O}^pS$ es la topología más pequeña que contiene todos los conjuntos abiertos originales y también el complemento de todos los conjuntos compactos saturados de $S$. Notemos que hacer esto puede crear nuevos conjuntos compactos saturados que no son cerrados en $S$.\\

Usando esta construcción tenemos que $S$ es empaquetado si y solo si $^pS=S$.

\begin{lem}
    Sea $S$ un espacio topológico. Si $U\in \mathcal{O}^pS$ entonces $U\in\mathcal{O}^fS$.
\end{lem}

\begin{proof}
    Consideremos un conjunto $Q$ compacto saturado. Notemos que, por la Definición \ref{Frontal}, basta con verificar que $Q'=\bigcup\{q^-\mid q^-\in \mathcal{C}S\}$.\\

    Como $Q$ es saturado, entonces $Q=\uparrow Q$, de aquí que $Q'=\downarrow Q$ en el orden de especialización. Luego si $q\in Q'$ entonces $p^-\subseteq Q'$ y por lo tanto 
    \[
    Q'=\bigcup\{q^-\mid p\notin Q\}=\bigcup\{q^-\mid q^-\in \mathcal{C}S\}
    \]
\end{proof}

El resultado anterior no dice que la topología de parches es intermedia entre la topología original y la topología frontal. En otras palabra tenemos que

\[
\mathcal{O}S \hookrightarrow \mathcal{O}^PS \hookrightarrow \mathcal{O}^fS
\]
para cada espacio $S$.\\

Es momento de saber como se comporta el espacio de parches con las propiedades de separación.

\begin{lem}\label{ParcheT1}
    El espacio de parches de un espacio $T_0$ es $T_1$.
\end{lem}

\begin{proof}
    Sea $S$ un espacio $T_0$ y consideremos $p, q\in S$ tales que $p^-\nsubseteq q^-$, es decir, $p\notin q^-$. Como $S$ es $T_0$ tenemos que $\exists \,U\in \mathcal{O}S$ tal que $q\notin U\ni p$. Observemos que $q\notin \uparrow p$, donde $\uparrow p\in \mathcal{Q}S$. Entonces $(\uparrow p)'\in \mathcal{O}^PS$ y además 
    \[
    p \notin (\uparrow p)'\ni q,
    \]
    es decir, el espacio de parches es $T_1$.
\end{proof}

\begin{lem}\label{Parcheidem}
    Si $S$ es un espacio $T_1$, entonces $^{PP}S=^PS$.
\end{lem}

\begin{proof}
    Recordemos que en un espacio $T_1$ todos los conjuntos son saturados. De esta manera, si $S$ es $T_1$, entonces todos los subconjuntos de $S$ y $^PS$ son saturados. Además cada conjunto compacto de $^PS$ es también compacto de $S$. Así cada subconjunto compacto saturado de $^PS$ es también compacto saturado y por lo tanto $^PS=\,^{PP}S$ como queríamos.  
\end{proof}

\begin{cor}
    Para cada espacio $S$ que es $T_0$ tenemos que $^{PPP}S=^{PP}S$.
\end{cor}

\begin{proof}
    Sea $S$ un espacio $T_0$. Por el Lema \ref{ParcheT1} tenemos que $^PS$ es $T_1$ y así, al aplicar el Lema \ref{Parcheidem} para $^PS$ obtenemos $^{PP}(^PS)=\,^P(^PS)$.
\end{proof}

\begin{lem}\label{ParcheT2}
    El espacio de parches de un espacio $T_2$ es el mismo.
\end{lem}

\begin{proof}
    Sea $S$ un espacio $T_2$. Por el Lema \ref{compacto saturado} tenemos que todo conjunto compacto saturado es cerrado, es decir, $S$ es empaquetado. Si $S$ es empaquetado, $^PS=S$ que es lo que queríamos probar.
\end{proof}

El Teorema \ref{Sobriofrontal} nos dice que si un espacio es sobrio, entonces su espacio frontal también lo es. Para el espacio de parches no ocurre lo mismo, es decir, si el espacio es sobrio, el espacio de parches no necesariamente es sobrio.

\begin{lem}\label{Lem4.3.1}
    Sean $S$ un espacio sobrio y $F$ un subconjunto cerrado irreducible de $^PS$ (es decir, $F\in \mathcal{C}^PS$ y es cerrado irreducible en  $^PS$). Entonces $F$ está conformado por un único punto o es infinito.
\end{lem}

\begin{proof}
    Sea $F$ un conjunto cerrado e irreducible en $^PS$, entonces $F^-$ es cerrado e irreducible en $S$, pues si $\mathcal{O}S\subseteq \mathcal{O}^PS$ implica que $(\mathcal{O}S'\supseteq (\mathcal{O}^PS)'$. Por hipótesis $S$ es sobrio, y por lo tanto $F^-=p^-$ para un único $p\in S$. Como $p\in F^-$ tenemos que si 
    \[
    p\in U\in \mathcal{O}S\Rightarrow F\cap U\cap p^-=F\cap U\neq \emptyset
    \]
    y por lo tanto $p\in F^=$. .De esta manera tenemos que $F^-=F^=$ y como $F$ es cerrado frontal $F^= =F$ y en consecuencia $p\in F$.\\

    Supongamos que $F\neq \{p\}$ y, por contradicción, supongamos que $F$ finito. Así 
    \[
    F=\{p, q_0, q_1, \dots , q_n\}
    \] 
    para algunos puntos $q_0, \dots q_n$ distintos de $p$. Para cada $q_i$ tenemos que $q_i\in F\subseteq p^-$, es decir, $q_i\sqsubseteq p$ en el orden de especialización. Luego $p\in F\cap q_i^-$, pues de lo contrario $p\sqsubseteq q_i$ y eso implicaría que $p=q_i$ lo cual no es posible. De esta manera los conjuntos 
    \[
    F\cap (\uparrow p)', F\cap (q_0^-)', \dots , F\cap (q_n^-)'\neq \emptyset.
    \]
    Pero $F\cap (\uparrow p)'\cap (q_0^-)'\cap \dots \cap (q_n^-)'= \emptyset$ lo cual contradice que $F$ es irreducible en $^PS$. Por lo tanto $F$ es infinito.
\end{proof}

Este argumento se puede refinar de diferentes maneras para obtener más información.

\begin{lem}\label{Lem4.3.2}
    El espacio de parches de un espacio $T_1$ y sobrio es $T_1$ y sobrio.
\end{lem}

\begin{proof}
    Sean $S$ un espacio $T_1$ y sobrio y $F$ un conjunto cerrado irreducible de $^PS$. Similar al Lema \ref{Lem4.3.1} tenemos $F^-=p^-$ para algún $p\in S$. Por hipótesis, $S$ es $T_1$, es decir
    \[
    p\in F\subseteq F^-=p^-=\{p\},
    \]
    es decir, $F=\{p\}$. Por lo tanto, como $F$ es un conjunto cerrado irreducible en $^PS$, entonces $^PS$ es $T_1$
\end{proof}

\subsection{Propiedades funtoriales del espacio de parches}\label{P. funtoriales spuntos}

Para ver la funtorialidad de la construcción de parches, debemos preguntarnos lo siguiente:

\begin{enumerate}
    \item ¿Es posible ver la construcción de parches 
    \[
    S\mapsto\, ^PS
    \]
    como la asignación de objetos de un funtor en la categoría de espacios topológicos o en alguna subcategoría adecuada?

    \item ¿Es posible ver la función continua 
    \[
    ^PS\to S
    \]
    como una transformación natural entre este funtor y el funtor identidad?
\end{enumerate}

Supongamos que $\phi \colon T\to S$ es una función continua entre espacios topológicos. Esto da tres lados de un cuadrado
\[\begin{tikzcd}
	T & S \\
	{^pT} & {^pS}
	\arrow["\phi", from=1-1, to=1-2]
	\arrow[from=2-1, to=1-1]
	\arrow[from=2-2, to=1-2]
\end{tikzcd}\]
donde cada lado es continuo.\\ 

¿Bajo que circunstancias existe una función continua $^p\phi \colon \,^pT\to \,^pS$ que hace que el cuadrado conmute?\\

Como funciones, ambas aplicaciones verticales son funciones identidad. Por lo tanto, si existe una función $^p\phi$, entonces como función es solo $\phi$.\\

Así se puede plantear la siguiente pregunta: \emph{¿Cuándo una función continua $\phi\colon T\to S$ es también ``\textbf{parche continua}''?, es decir, es continua en relación con las topologías de parches}.

\begin{dfn}\label{Parchecontinua}
    Decimos que una función continua $\phi\colon T\to S$ es \emph{parche continua} si envía conjuntos compactos saturados a conjuntos compactos saturados, es decir, si $\phi ^{-1}(Q)\in \mathcal{Q}T$ siempre que $Q\in \mathcal{Q}S$. 
\end{dfn}

Por lo tanto, si $\phi$ convierte conjuntos compactos saturados, entonces ciertamente es parche continuo. Pero se peude sospechar que esta condición que es suficiente para la continuidad del parche no sea necesaria.\\

Sea $f^*\colon A\to B$ un morfismo de marcos y $f_*$ su adjunto derecho. En general, $f_*$ preserva ínfimos arbitrarios, pero no necesariamente preserva supremos. Nos fijaremos en aquellos morfismos para los que $f_*$ preserva ciertos supremos.

\begin{dfn}\label{Scottcontino}
    Para un morfismo de marcos $f^*$ y su adjunto derecho $f_*$ dados como antes, decimos que el adjunto derecho $f_*$ es \emph{Scott-continuo} si 
    \[
    f_*(\bigvee Y)=\bigvee f_*(Y)
    \]
    para cada subconjunto dirigido $Y$ de $B$.
\end{dfn}

Sabemos que cada función continua $\phi\colon T\to S$ da un morfismo de marcos 

\[\begin{tikzcd}
	{\mathcal{O}S} && {\mathcal{O}T}
	\arrow["{\phi^*}", shift left=2, from=1-1, to=1-3]
	\arrow["{\phi_*}", shift left=2, from=1-3, to=1-1]
\end{tikzcd}\]
entre las topologías. Podemos imponer la condición extra de Scott-continuidad en $\phi_*$. Donde esta no debe confundirse con la continuidad dada por $\phi$.

\begin{lem}\label{Pcontinua y Scontinua}
    Sea $\phi$ una función continua como la de antes y supongamos que el espacio $T$ es sobrio. Si $\phi_*$ es Scott-continua, entonces $\phi$ convierte conjuntos compactos saturados y por lo tanto $\phi$ es parche continua.
\end{lem}

\begin{thm}
    Sea $\phi$ una función continua como la de antes y supongamos que ambos espacios, $S$ y $T$ son sobrios. Entonces $\phi_*$ es Scott-continua si y solo si las siguientes condiciones se cumplen
    \begin{itemize}
        \item $\phi$ convierte conjuntos compactos saturados.
        \item Si $Y\in \mathcal{C}T$, entonces $\downarrow \phi(Y)\in \mathcal{C}S$.
    \end{itemize}
\end{thm}

\subsection{El triángulo fundamental de un espacio}

La inclusión $\iota\colon \mathcal{O}S\to \mathcal{O}^fS$, en cierto modo, es un morfismo de marcos que resuelve el problema de complementación para $\mathcal{O}S$. En esta subsección, usamos la información anterior para encontrar el espacio de puntos del ensamble de un marco $A$.

\begin{lem}\label{Lema6.3.1}
    Para cada espacio $S$ existe un único morfismo de marcos $\sigma$ tal que el siguiente diagrama conmuta.
\[\begin{tikzcd}
	{\mathcal{O}S} & {\mathcal{O} ^fS} \\
	{N\mathcal{O}S}
	\arrow["\iota", from=1-1, to=1-2]
	\arrow["{\eta_{\mathcal{O}S}}"', from=1-1, to=2-1]
	\arrow["\sigma"', from=2-1, to=1-2]
\end{tikzcd}\]
\end{lem}

\begin{proof}
    Es un caso particular del Teorema \ref{Teorema6.2.2}.
\end{proof}

El resultado anterior es cierto para cualquier espacio $S$. En la practica, es más conveniente utilizar $S=\pt A$, donde $A\in \Frm$. Además, para este caso $S$ resulta ser sobrio.

Observemos que el morfismo $\sigma$ actúa como una transformación natural cuando el espacio $S$ varia. Nuestro objetivo es encontrar una descripción especifica para el morfismo $\sigma$.

\begin{dfn}\label{Definicion6.3.2}
    Sea $S\in \Top$. Definimos $\sigma\colon N\mathcal{O}S\to \mathcal{O}^fS$ al morfismo dado por 
    \[
    \sigma(j)=\bigcup\{j(W)\setminus W\mid W\in \mathcal{O}S\}
    \]
    para cada $j\in N\mathcal{O}S$.
\end{dfn}

Para verificar que $\sigma$ es un morfismo de marcos usaremos una manera equivalente de definirlo.

\begin{lem}\label{Lema6.3.3}
    Para $\sigma$ como en la Definición \ref{Definicion6.3.2} tenemos 
    \[
    p\in \sigma(j)\Leftrightarrow p\in j(\overline{p}')
    \]
    para cada $p\in S$. Además, $\sigma$ es un $\wedge-$morfismo.
\end{lem}

\begin{proof}
    Consideremos $p\in S$ arbitrario. Si $p\in \sigma(j)$, entonces 
    \[
    p\in \bigcup\{j(W)\setminus W\mid W\in \mathcal{O}S\}.
    \]
    Así, $p\in j(W)$ y $p\notin W$ para algún abierto $W$. De aquí que 
    \[
    W\subseteq \overline{p}' \quad \mbox{ y }\quad p\in j(W)\subseteq j(\overline{p}')
    \]
    que es lo que queríamos.\\

    Recíprocamente, supongamos que $p\in j(\overline{p}')$. Estableciendo $W=\overline{p}'$ tenemos que $p\in j(W)$ y $p\notin W$. Por lo tanto $p\in \sigma(j)$.\\

    Por último, de manera general, consideremos $j, k\in NA$. Entonces
    \[
    p\in \sigma (j\wedge k)\Leftrightarrow p\in (j\wedge k)(\overline{p}')\Leftrightarrow p\in j(\overline{p}'\cap k(\overline{p}')\Leftrightarrow p\in \sigma(j)\wedge \sigma(k)
    \]
    para verificar que $\sigma$ es un $\wedge-$morfismo.
\end{proof}

En la Definición \ref{Definicion5.3.1} se introducen los núcleos espacialmente inducidos. El siguiente resultado nos da más información sobre ellos.

\begin{lem}\label{Lema6.3.4}
    Para $\sigma$ definido como antes tenemos que $\sigma([E])=E^\Box$ para cada $E\in \mathcal{P}S$.
\end{lem}

\begin{proof}
    Consideremos $E\in \mathcal{P}S$. Para cada $U\in \mathcal{O}S$ tenemos
    \[
    [E](U)\setminus U=(E\cup U)^\circ \setminus U\subseteq E^\Box.
    \]
    Así, $\sigma([E])=\bigcup\{[E](W)\setminus W\mid W\in \mathcal{O}S\}\subseteq E^\Box$ se cumple.\\

    Para la otra contención, supongamos que $p\in E^\Box$. Así existe $U\in \mathcal{O}S$ tal que
    \[
    p\in U\cap \overline{p}'\subseteq E\subseteq \Rightarrow p\in U\subseteq E\cup \overline{p}'\Rightarrow p\in [E](\overline{p}').
    \]
    Por lo tanto $p\in [E](\overline{p}')\setminus \overline{p}'$, es decir, $p\in \sigma([E])$.
\end{proof}

Hasta este punto hemos mostrado que $\sigma$ es un $\wedge-$morfismo. Para verificar que es un morfismo de marcos basta con mostrar quien es su adjunto derecho.

\begin{thm}\label{Teorema6.3.5}
    Para cada espacio $S$ el par de asignaciones 
    \[\begin{tikzcd}
	{N\mathcal{O}S} && {\mathcal{O}^fS}
	\arrow["\sigma", shift left=3, from=1-1, to=1-3]
	\arrow["{[ \_ ]}", shift left=3, from=1-3, to=1-1]
\end{tikzcd}\]
forman un par adjunto. Además, $\sigma$ es un morfismo de marcos.
\end{thm}

\begin{proof}
    Por la definición de adjunción, debemos verificar que $\sigma(j)\subseteq E\Leftrightarrow j\leq [E]$ se cumple para todo $j\in N\mathcal{O}S$ y $E\in \mathcal{O}^fS$.\\

    Supongamos que $\sigma(j)\subseteq E$. Así, para cada $U\in \mathcal{O}S$ y $p\in S$ tenemos 
    \[
    \begin{split}
        p\in j(U) &\Rightarrow  p\in (j(U)\setminus U)\;\mbox{ o }\; p\in U\\
        & \Rightarrow p\in \sigma(J) \; \mbox{ o }\; p\in U\\
        & \Rightarrow p\in E \;\mbox{ o }\; p\in U\\
        & \Rightarrow p\in E\cup U=(E\cup U)^\circ.
    \end{split}
    \]
    Por lo tanto $j(U)\subseteq [E](U)$, y al ser $U$ arbitrario se cumple que $j\leq [E]$.\\

    Recíprocamente, supongamos que $j\leq [E]$. Entonces $\sigma(J)\subseteq \sigma([E])$. Por el Lema \ref{Lema6.3.4} sabemos que $\sigma([E])=E^\Box$ para todo $E\in \mathcal{P}S$. De aquí que $\sigma(j)\subseteq E^\Box\subseteq E$ que es lo que queríamos.\\

    Por lo tanto $[ \_ ]$ es el adjunto derecho de $\sigma$.\\

    Para ver que es un morfismo de marcos debemos recordar que cualquier función monótona con adjunto derecho respeta supremos arbitrarios. Resta verificar el comportamiento de $\sigma$ a través de los núcleos $\id$ y $tp$.
    \[
    \begin{split}
    \sigma(\id)& =\bigcup\{\id(W)\setminus W\mid W\in \mathcal{O}S\}=\bigcup\{W\setminus W\mid W\in \mathcal{O}S\}=\emptyset\\
    \sigma(\tp)& =\bigcup\{\tp(W)\setminus W\mid W\in \mathcal{O}S\}=\bigcup\{S\setminus W\mid W\in \mathcal{O}S\}=S.
    \end{split}
    \]
\end{proof}

Para terminar esta subsección retomamos lo dicho en el Lema \ref{Lema6.3.1} y verificamos que, así definido, $\sigma$ hace conmutar el diagrama de dicho resultado.

\begin{lem}\label{Lema6.3.6}
    Con los datos anteriores el siguiente diagrama conmuta
    \[\begin{tikzcd}
	{\mathcal{O}S} & {\mathcal{O} ^fS} \\
	{N\mathcal{O}S}
	\arrow["\iota", from=1-1, to=1-2]
	\arrow["{\eta_{\mathcal{O}S}}"', from=1-1, to=2-1]
	\arrow["\sigma"', from=2-1, to=1-2]
\end{tikzcd}\]
\end{lem}

\begin{proof}
    Consideremos $U\in \mathcal{O}S$, entonces $(\sigma\circ \eta_{\mathcal{O}S})(U)=\sigma(u_U)$. Luego, para $V\in \mathcal{O}S$ se cumple que 
    \[
    \sigma(u_U(V))=\sigma(U\cup V)=\sigma((U\cup V)^\circ)=\sigma([U])=U^\Box=U.
    \]
\end{proof}

\subsection{El espacio de puntos del ensamble}

Es momento de ver la relación entre el ensamble de la topología de un espacio ($N\mathcal{O}S$ y la topología de Skula. El siguiente Teorema resulta de juntar el Teorema \ref{Teorema6.2.3} y el Lema \ref{Lema6.3.1}.

\begin{thm}\label{Teorema6.4.1}
    Para $A\in \Frm$ y $S=\pt A$ el siguiente diagrama conmuta.

    \[\begin{tikzcd}
	A & {\mathcal{O}S} \\
	NA & {N\mathcal{O}S} & {\mathcal{O}^fS}
	\arrow["{U_A}", from=1-1, to=1-2]
	\arrow["{\eta_A}"', from=1-1, to=2-1]
	\arrow["{\eta_{\mathcal{O}S}}"', from=1-2, to=2-2]
	\arrow["\iota", from=1-2, to=2-3]
	\arrow["{NU_A}"', from=2-1, to=2-2]
	\arrow["{\sigma_{\mathcal{O}S}}"', from=2-2, to=2-3]
\end{tikzcd}\]
\end{thm}

A manera de notación, denotamos la composición inferior del diagrama por $\sigma_{\mathcal{O}S}\circ NU_A=\Sigma_A$. Se puede verificar que este morfismo le asigna al ensamble su espacio de puntos.

\begin{lem}\label{Lema6.4.2}
    Para cada $j\in NA$ las tres condiciones son equivalentes:
    \begin{enumerate}
        \item $j$ es $\wedge-$irreducible (en $NA$).
        \item $j$ es 2-valuado (cada valor de $j$ es 0 o 1).
        \item $a=j(0)$ es $\wedge-$irreducible (en $A$) y $j=w_a$.
    \end{enumerate}
\end{lem}

\begin{proof}
    \begin{description}
        \item[$1)\Rightarrow 2)$] Supongamos que $j$ es $\wedge-$irreducible. Verificaremos que $j$ es de la forma 
        \[
        j(x)= \left\{ \begin{array}{lcc} 1, & \mbox{ si } & x \nleq a \\ \\  a, & \mbox{ si } & x \leq a \end{array} \right.
        \]
        donde $a=j(0)$. Notemos que la anterior es la única manera posible en la que un núcleo puede tomar dos valores.\\

        Sabemos que $u_x\wedge v_x=id\leq j$ se cumple para todo $x\in A$. Por lo tanto $u_x\leq j$ o $v_x\leq j$. Supongamos lo primero, es decir, para todo $y\in A$ $u_x(y)\leq j(y)$. De manera particular
        \[
        x=u_x(0)\leq j(0)=a\quad \mbox{ y }\quad j(x)=j(a)=a,
        \]
        pues $0\leq x$ y $a=j(0)\leq j(x)$. Observemos que esto ocurre cuando $x\leq a$.\\
        Supongamos ahora que $x\nleq a$. Así $v_x\leq j$ y $10v_x(x)\leq j(x)$, es decir, $j(x)=1$. Por lo tanto, $j$ tiene la forma que requeríamos.

        \item[$2)\Rightarrow 3)$] Supongamos que $j$ es $2-$valuado y consideremos $x\wedge y\leq a$ para algunos $x, y\in A$. Entonces
        \[
        j(x)\wedge j(y)\leq j(x\wedge y)\leq a
        \]
        por la forma de $j$. Como $j(x), j(y)\in \{a, 1\}$, entonces $j(x)=a$ o $j(y)=a$ y por lo tanto $x\leq a$ o $y\leq a$, nuevamente por la forma de $j$. Como $a=j(0)$ se cumple que $a\neq 1$ y así $a$ es $\wedge-$irreducible.\\

        Ahora debemos probar que $w_a(x)=j(x)$, con $j$ definido como antes, siempre que $a$ es $\wedge-$irreducible. Supongamos que $x\leq a$ entonces 
        \[
        ((x\succ a)\succ a)=(1\succ a)=a.
        \]

        Si $x\nleq a$ entonces 
        \[
        (x\succ a)\wedge a=(x\succ a)\wedge ((x\succ a)\succ a))\leq a \quad\mbox{ y }\quad x\leq w_a(x)
        \]
        así $w_a(x)=((x\succ a)\succ a)\nleq a$. Como $a$ es $\wedge-$irreducible se debe cumplir que $(x\succ a)\leq a$. Por lo tanto $((x\succ a)\succ a)=1$, es decir,
        \[
        w_a(x)= \left\{ \begin{array}{lcc} 1, & \mbox{ si } & x \nleq a \\ \\  a, & \mbox{ si } & x \leq a \end{array} \right.
        \]
        
        \item[$3)\Rightarrow 1)$]  Queremos probar que $w_a$ es $\wedge-$irreducible en $NA$ siempre que $a$ es $\wedge-$irreducible en $A$. Ya hemos demostrado que cuando $a$ es $\wedge-$irreducible, entonces $w_a$ es $2-$valuado. También sabes que si $k\wedge l\leq w_a$ entonces 
        \[
        k(a)\wedge l(a)\leq w_a(a)\Rightarrow k(a)\leq a \;\mbox{ o } \;l(a)\leq a\Rightarrow k\leq w_a\;\mbox{ o }\;l\leq w_a,
        \]
        es decir, $w_a$ es $\wedge-$irreducible en $NA$.
    \end{description}
\end{proof}

Considerando $S=\pt A$ y $T=\pt NA$, lo anterior nos proporciona un par inverso de biyecciones entre $S$ y $T$, es decir

\[\begin{tikzcd}
	S & T
	\arrow["\phi", shift left=3, from=1-1, to=1-2]
	\arrow["\psi", shift left=3, from=1-2, to=1-1]
\end{tikzcd}\]
donde $\phi(p)=w_p$ y $\psi(m)=m(0)$. De esta manera, el espacio de puntos de $NA$ tiene esencialmente los mismos puntos que $S$, pero en una topología diferente. ¿Cuál es esta topología?

\begin{lem}\label{Lema6.4.3}
    Sean $A\in \Frm$, $S=\pt A$ y $T=\pt NA$. Para $\mathcal{O}T$ se cumple que $\mathcal{O}^fS$ es la topología que provoca que las siguientes funciones sean un par de homeomorfismos.

\[\begin{tikzcd}
	S & T
	\arrow["\phi", shift left=3, from=1-1, to=1-2]
	\arrow["\psi", shift left=3, from=1-2, to=1-1]
\end{tikzcd}\]    
\end{lem}

\begin{proof}
    Los conjuntos abiertos usuales de $S$ (los elementos de $\mathcal{O}S$) son de la forma
    \[
    U_A(x)=\{p\in S\mid x\nleq p\}
    \]
    para $x\in A$. Los abiertos de $T$ son de la forma 
    \[
    U_{NA}=\{m\in T\mid j\nleq m\}
    \]
    para $j\in NA$.\\

    También sabemos que si $j\in NA$, entonces $j=\bigvee\{v_x\wedge u_{j(x)}\mid x\in A\}$. Por lo tanto, los conjuntos $U_{NA}(u_x)$ y $U_{NA}(v_x)$ con $x\in A$ forman una sub-base de $T$, ya que $U( \_ )$ es un morfismo de marcos.\\

    Luego
    \[
    \begin{split}
        \phi^{-1}(U_{NA}(u_x))=\psi (U_{NA}(u_x))&=\{\psi (m)\mid u_x\nleq m \in T\}\\
        &=\{m(0)\mid u_x\nleq m\in T\}\\
        &=\{p\mid x\nleq p\in S\}\\
        &=U_A(x).
    \end{split}
    \]

    Sabemos que $v_x\wedge u_x=\id\leq m$ y $v_x\vee u_x=\tp$ se cumple para todo $x\in A$ y $m\in T$. De esta manera se debe cumplir que $v_x\leq m$ o $u_x\leq m$. En otras palabras
    \[
    x\in U_{NA}(v_x)\Leftrightarrow m\notin U_{NA}(u_x)
    \]
    para cada $x\in A$ y $m\in T$. Así
    \[
    \phi^{-1}(U_{NA}(v_x))=\psi (U_{NA}(v_x))=\psi(U_{NA}(u_x)')=U_A(x)'
    \]
    pues $\psi$ es un morfismo biyectivo de marcos.\\

    De manera similar tenemos que $\psi^{-1}(U_A(x))=U_{NA}(u_x)$ y $\psi^{-1}(U_A(x)')=U_{NA}(v_x)$. Por lo tanto los conjuntos $U_A(x)$ y $(U_A(x))'$ forman una sub-base para la topología inducida en $S$.Además, esta es la topología de Skula, $\mathcal{O}^fS$.
\end{proof}

\section{El Teorema de Hoffman-Mislove}

En esta última sección relacionamos algunos de los conceptos vistos hasta este momento para un marco $A$ y los trasladamos a la topología de su espacio de puntos. De manera especifica, nos interesa descubrir la correspondencia biyectiva que proporcina el Teorema de Hoffman-Mislove.\\

Recordemos que, para un espacio $S$, $\mathcal{Q}S$ es el conjunto de todos los subconjuntos compactos saturados de $S$. Para cada conjunto $Q\in \mathcal{Q}S$ podremos obtener un filtro abierto $\nabla(Q)$ en $A$ dado por 
\[
x\in \nabla(Q) \Leftrightarrow Q\subseteq U_A(x)
\]
donde $U_A$ es la reflexión espacial presentada en \ref{MorfismoRE}. Veremos que cada filtro abierto surge de esta manera de un único $Q$ compacto saturado.

\begin{lem}\label{Lema3.4.1}
        Sea $F$ un filtro abierto de $A$. Consideramos a $M$ como el conjunto de elementos máximos en $A\setminus F$. Entonces para cada $a\in A\setminus F$ existe algún $m\in M$ tal que $a\leq m$.
\end{lem}

\begin{proof}
    Notemos que como $F$ es un filtro abierto arbitrario, entonces el complemento $A\setminus F$ es cerrado bajo supremos dirigidos. De esta manera, para $A\setminus F$ podemos hacer uso del Lema de Zorn para obtener $m\in M$ tal que $a\leq m$, con $a\in A\setminus F$.
\end{proof}

Se puede verificar que para un marco $A$, cada elemento máximo es un elemento $\wedge-$irreducible. En otras palabras, para cada $m\in M$ tenemos que $m\in \pt A$, es decir, $m\neq 1$ y si $x\wedge y\leq m$, entonces $x\leq m$ o $y\leq m$ se cumple.\\

De esta manera, para $S=\pt A$, como $M\subseteq S$, podemos reformular el lema anterior de la siguiente manera.

\begin{cor}\label{HM1}
    La equivalencia
    \[
    M\subseteq U_A(a)\Leftrightarrow a\in F
    \]
    se cumple para cada $a\in A$.
\end{cor}

\begin{proof}
    Primero, supongamos que $a\in F$. Si $m\in M$, entonces $m\notin F$, es decir, $a\nleq m$. Por lo tanto $m\in U_A(a)$.\\

    Ahora, supongamos que $a\notin F$. Por el Lema \ref{Lema3.4.1} existe algún $m\in M$ tal que $a\leq m$, de modo que $m\notin U_A(a)$ y por lo tanto $M\nsubseteq U_A(a)$, es decir, $M\subseteq U_A(a)$ implica que $a\in F$.
\end{proof}

Este resultado tiene otra consecuencia más importante.

\begin{lem}
    El conjunto $M$ es compacto en $S=\pt A$.
\end{lem}

\begin{proof}
    Consideremos cualquier cubierta abierta $\{U_A(x)\mid x\in X\}$ de $M$. De manera usual, supongamos que el conjunto $X\subseteq A$ es dirigido. Sea $a=\bigvee X$, entonces
    \[
    M\subseteq \bigcup \{U_A(x)\mid x\in X\}=U_A(a)
    \]
    y por lo tanto $a\in F$. Pero $F$ es un filtro abierto y $X$ es un conjunto dirigido, de modo que $x\in F$ para algún $x\in X$. Esto nos da $M\subseteq U_A(x)$ para obtener la subcubierta requerida.
\end{proof}

Ahora, sea $Q$ la saturación de $M$. Como cada conjunto abierto es saturado, tenemos que $Q$ y $M$ tienen exactamente los mismos súper conjuntos abiertos. En particular, $Q$ es compacto, y por lo tanto, $Q\in \mathcal{Q}S$.\\

Notemos también que para cada $x\in A$ tenemos
\begin{equation}\label{HMcaracterizacion}
x\in F\Leftrightarrow M\subseteq U_A(x)\Leftrightarrow Q\subseteq U_A(x)
\end{equation}
de modo que el filtro abierto $F$ surge del conjunto compacto saturado $Q$ como queríamos. Veamos ahora que $Q$ es el único conjunto compacto saturado asignado a $F$ de esta manera.\\

Supongamos que existen dos conjuntos $P$ y $Q$ que cumplen lo mencionado antes. Entonces 
\[
P\subseteq U_A(x)\Leftrightarrow Q\subseteq U_A(x)
\]
para cada $x\in A$. Lo anterior puede reformularse como 
\[
(\exists p\in P)[x\leq p]\Leftrightarrow (\exists q\in Q)[x\leq q]
\]
para cada $x\in A$. Consideremos cualquier $p\in P$ y sea $x=p$. Entonces existe algún $q\in Q$ con $p\leq q$ y por lo tanto $q\sqsubseteq p$. Como $Q$ es saturado, esto da que $p\in P$ y por lo tanto $Q\subseteq P$. Similarmente vemos que $P\subseteq Q$.\\

Esto muestra que podemos obtener un filtro abierto de un único $Q\in \mathcal{Q}S$ de manera canónica. Resta ver que tenemos un proceso inverso, es decir, dado cualquier filtro abierto, por medio de este obtener un conjunto compacto saturado.

\begin{lem}\label{HM2}
    Si $S=\pt A$, $F$ un filtro abierto y $Q$ la saturación del conjunto $M$ definido antes, entonces $Q=S\setminus F$.
\end{lem}

\begin{proof}
    Consideremos cualquier $q\in Q$. Como $Q$ es la saturación de $M$, existe algún $m\in M$ tal que $m\sqsubseteq q$ en el orden de especialización de $S$. Entonces $q\leq m\notin F$ en el orden original del marco $A$, es decir, $q\notin F$. De esta manera $Q\subseteq (S\setminus F)$.\\

    Recíprocamente, supongamos que $p\in S\setminus F$. Si $p\in S\setminus F$, por el Lema \ref{Lema3.4.1}, existe $m\in M$ tal que $p\leq m$, es decir, $m\sqsubseteq p$ y por lo tanto $p\in Q$.
\end{proof}

La versión corta del Teorema de Hoffman-Mislove que estaremos usando a lo largo de este trabajo es la que se enuncia a continuación.

\begin{thm}[Hoffmann-Mislove]\label{TeoremaHM}

Sea $A$ un marco con $S=\pt(A)$ su espacio de puntos, entonces existe una biyección entre:

\begin{enumerate}[i)]

\item $A^{\wedge}=$ filtros abiertos en $A$.

\item $\mathcal{Q}S=$ conjuntos compactos saturados.

\end{enumerate}
\end{thm}

La prueba de este resultado son las demostraciones del Corolario \ref{HM1} y el Lema \ref{HM2}.


