\chapter{Marcos arreglados vs Axiomas tipo Hausdorff}\label{Relación MA y AH}

Notemos que la Definición \ref{Definición8.2.1} identifica a un tipo especifico de marcos. El Teorema \ref{Teorema8.4.4} caracteriza a los marcos arreglados por medio de una condición especial, de manera especifica con el axioma de separación $T_2$.
Con esto en mente, resulta natural el preguntarse: \emph{¿se pueden enunciar caracterizaciones para los marcos arreglados que sean meramente de marcos?}, es decir, que no se haga mención a la topología de un espacio. Un ejemplo no tan exitos de lo que 
que acabamos de preguntar lo tenemos en el siguiente resultado.

\begin{cor}\label{TadyHausdorff}
Para $A$ un marco espacial se cumple lo siguiente: $\mathcal{O}S$ es un marco Hausdorff si y solo si $A$ es $1$-arreglado. 
\end{cor}
La prueba se sigue del hecho de que la propiedad $\mathbf{(H)}$ es conservativa.\\

De esta manera, hemos relacionado la condición de arreglo con uno de los axiomas de separación. Veamos cual es la relación de esta con las otras propiedades.\\

\section{Arreglado y su relación con los propiedades de separación en $\Frm$}

El Lema \ref{Lema8.4.1} nos dice que arreglado implica $T_1$. ¿Qué pasa con las propiedades más fuertes que $T_1$?

\begin{lem}\label{TadyFHausdorff}
    Todo marco fuertemente Hausdorff es arreglado.
\end{lem}

\begin{proof}
    Consideremos $A\in \Frm$ fuertemente Hausdorff. Si $A$ cumple $\mathbf{(fH)}$, entonces todo sublocal compacto es cerrado. Por teoría de marcos, para $j\in NA$ arbitrario, $A_j$ es compacto si y solo si $\nabla(j)\in A^\wedge$. De aquí que, al ser compacto y por $\mathbf{(fH)}$ $A_j=A_{u_d}$, para algún $d\in A$, es decir, $j=u_d$ y $\nabla(j)=\nabla(u_d)$ para algún $d\in A$, en particular, por H-M, para todo $F\in A^\wedge$, $v_F\in NA$. Así $\nabla(v_F)=\nabla(u_d)$, es decir, para $x\in F$ se cumple que $u_d(x)=1=d\vee x$. Por lo tanto $A$ es arreglado.
\end{proof}

Con esto tenemos las implicaciones
\[
\mathbf{(fH)} \Rightarrow \mbox{ Arreglado }\Rightarrow T_1
\]

Resulta natural el pensar que la propiedad $\mathbf{(H)}$ este metida entre ellas. La realidad es que hasta el dia de hoy no hemos logrado demostrar que 
\[
\mbox{\emph{Todo marco Hausdorff es arreglado.}}
\]

Aunque no lo menciona, Sexton prueba en su tesis doctoral que si un marco es $T_3$, entonces este es parche trivial. Además, tenemos que parche trivial si y solo si arreglado. Por lo tanto, tenemos que
\[
T_3\Rightarrow \mbox{ Arreglado.}
\] 
Por otro la
do, Simmons en su artículo (citar regular and fitness), prueba que 
\[
\mathbf{(aju)}\Rightarrow \mbox{ Arreglado.}
\]

Las pruebas que ellos realizan nos llevan a pensar que existe cierto comportamiento en algo que nosotros llamamos \emph{intervalos de admisibilidad}. Este tema los trataremos más adelante.\\

Por el momento, solo podemos suponer que alguna relación debe existir entre la propiedad $\mathbf{(H)}$ y la condición de arreglo. Hasta que no tengamos una prueba exitosa de este hecho (porque vaya que lo hemos intentado) o exista un ejemplo que exhiba información relevante,
lo anterior seguirá siendo una duda.\\

Lo que sigue ahora es analizar el comportamiento del marco $PA$ (el marco de parches de un marco $A$), con los axiomas de separación

\subsection{¿Qué propiedades de separación cumple el marco de parches?}\label{Parchesyseparación}

Aunque al final de esta pequeña subsección pueda sentirse que es innecesaria, este análisis nos hace pensar que el marco $P^2A$ es en el que debemos enfocar en mayor parte nuestro interés. Recordemos que el marco de parches viene inspirado por la definición de espacio de parches.
La noción de que un marco sea parche trivial (o equivalentemente arreglado), se basa en imitar lo que se conoce como espacio empaquetado. En el Capítulo \ref{Parches} se prueba que la construcción del espacio de parches se estaviliza en el tercer paso, es decir,
\[
^{pp}S=\,^{ppp}S,
\]
de esta manera, la sospecha que tenemos es que $P^2A$ debe actuar de manera similar. Veamos que ocurre que el parche antecesor.\\

\begin{itemize}
\item Observemos que si un marco es regular, entonces es arreglado. Si el marco es arreglado, entonces este es parche trivial. Por definición, un marco $A$ es parche trivial si $A\simeq PA$. Por lo tanto, si $PA$ fuera regular, siempre se cumpliria que $A$ es parche trivial, lo cual no es cierto siempre.
\item Similar al punto anterior, no puede cumplirse que $PA$ cumpla $\mathbf{(H)}$, pues de ser así, también implicaría que todos el marco $A$ es parche trivial y en general, esto no es cierto.
\item Si $PA$ es $T_1$, entonces para todo $j\in \pt PA$ se cumple que $j$ es máximo. Además, si $j\in \pt PA$, entonces $j$ es un punto salvaje o $j$ es un punto ordinario, es decir,
\[
j=w_p\quad \mbox{ o }\quad j\neq w_p
\]
para algún $p\in \pt A$. Si $j$ es salvaje, entonces para $a\in A$ se cumple que 
\[
u_a\leq j\leq w_a
\]
donde $a=j(0)$. Lo anterior nos dice que $j\in \pt PA$ no siempre es máximo y con ello no necesariamente será un marco $T_1$.
\end{itemize}

Como mencionamos al principio, puede resultar decepcionante el darse cuenta que $PA$ no cumple ninguna de las propiedades que teniamos en mente. Por tal motivo habrá que investigar que ocurre con $P^2A$.

\section{Intervalos de admisibilidad}

Recordemos que, por el Lema \ref{Lema5.5.4}, si $F\in A^\wedge$, entonces $F$ es admisible, es decir, $F=\nabla(j)$ para algún $j\in NA$. Además, el filtro abierto $F$ está asociado al núcleo ajustado $v_F$ y al núcleo $w_F$. De tal manera que obtenemos un intervalo de elementos en $NA$. Al intervalo $[v_F, w_F]$, cuando $F\in A^\wedge$, lo llamamos intervalo de admisibilidad.\\

Notemos que la construcción anterior está hecha para un marco $A$ arbitrario. De tal manera que podemos hacer lo mismo para el marco $\mathcal{O}S$, es decir, considerar un filtro abierto $\nabla=\nabla(Q)\in \mathcal{O}S^\wedge$, donde $Q\in \mathcal{Q}S$ es el correspondiente conjunto compacto saturado al filtro abierto $\nabla$. Entonces, el intervalo de admisibilidad asociado a $\nabla$ es $[v_Q, w_Q]$. 
Así, obtenemos un intervalo en $N\mathcal{O}S$ y el subíndice $Q$ solo indica que este es determinado por $Q$. Veamos cual es la relación que existe entre ambos intervalos.

\begin{prop}
    Para $F\in A^\wedge$ y $Q\in \mathcal{Q}S$, si $j\in [V_Q, W_Q]$, entonces 
    \[
    \nabla(U_* j U^*)=F.
    \]
    En otras palabras, $U_*jU^*\in [V_F,W_F]$, donde $U^*$ es la reflexión espacial y $U_*$ es su adjunto derecho.
\end{prop}

\begin{proof}
    Debemos verificar que $F\subseteq \nabla(U_*jU^*)$ y $\nabla(U_*jU^*)\subseteq F$.\\

    Primero, sabemos que $N$ es un funtor, así para $U^*\colon A\to \mathcal{O}S$ obtenemos la siguiente asignación
   \[\begin{tikzcd}
	A & NA \\
	\\
	{\mathcal{O}S} & {N\mathcal{O}S}
	\arrow[""{name=0, anchor=center, inner sep=0}, "U"', from=1-1, to=3-1]
	\arrow[""{name=1, anchor=center, inner sep=0}, "{N(U)}", from=1-2, to=3-2]
	\arrow["{N(\_)}", shorten <=7pt, shorten >=7pt, maps to, from=0, to=1]
\end{tikzcd}\]
Además, $N(U)_*$ es el adjunto derecho de $N(U)$.\\

Por teoría general de marcos tenemos lo siguiente:
\begin{enumerate}
    \item $N(U)(j)\leq k\Leftrightarrow j\leq N(U)_*(k)$.
    \item Si $k\in N\mathcal{O}S$ se cumple que 
    \[
    N(U)(j)\leq k\Leftrightarrow Uj\leq kU
    \]
    \item $N(U)_*(k)=U_*kU^*$, de hecho $UN(U)_*(k)=kU$ (C-Assembly, Corolario 6.7).
    \item Por la correspondencia biyectiva entre filtros abiertos y compactos saturados sabemos que 
    \[
    x\in F\Leftrightarrow Q\subseteq U(x)\Leftrightarrow U(x)\in \nabla(Q),
    \]
    donde $Q=S\setminus F$.
\end{enumerate}
De esta manera, para $k=j$, por 3), tenemos que 
\[
N(U)_*(j)=U_*jU^* \quad\mbox{ y }\quad UN(U)_*j=jU.
\] 
Así, para $x\in F$
\[\begin{tikzcd}
	x\in A & {\mathcal{O}S} & {\mathcal{O}S} & A
	\arrow["{U^*}", from=1-1, to=1-2]
	\arrow["j", from=1-2, to=1-3]
	\arrow["{U_*}", from=1-3, to=1-4]
\end{tikzcd}\]
y $U_*(j(U(x))=\bigwedge(S\setminus j(U(x)))$. Notemos que $U_*(j(U^*(x)))\subseteq \pt A$. Luego,
\[
x\in F \Leftrightarrow U(x)\in \nabla(j)=\nabla(Q)\Leftrightarrow S\setminus j(U(x))=\emptyset
\]
Además, evaluando $j(U^*(x))$ en $U_*$ tenemos que 
\[
(U_*jU^*)(x)=\bigwedge (S\setminus j(U(x)))=\bigwedge\emptyset=1
\]
es decir, $x\in \nabla(U_*jU^*)$. Por lo tanto $F=\nabla(U_*jU^*)$.
\end{proof}

Lo anterior define una función 
\[
\mho\colon [V_Q, W_Q]\to [V_F, W_F].
\]
Por lo visto en la Subsección \ref{Estructura de bloques}, para $Q\in \mathcal{Q}S$, el intervalo $[v_Q, w_Q]$ siempre tiene un elemento intermedio
\[
v_Q\leq [Q']\leq w_Q=[M']
\]
donde $M=\{m\in \mathcal{O}S\setminus \nabla(Q)\mid a\leq m\neq S\}$ y $a\in \mathcal{O}S\setminus \nabla(Q)$. De está manera, al evaluar a $[Q']$ en la función $\mho$, obtenemos
un elemento en $NA$, en este caso la pregunta es, ¿cómo es este elemento?, es decir, ¿cúal es el comportamiendo de $U_*[Q']U^*$ dentro del intervalo $[V_F, W_F]$?\\

Notemos que si $U_*[Q']U^*$ es un $u$-núcleo, recuperamos una condición similar a la que proporciona la Definición \ref{Definición8.2.1}. Además, si esto ocurre, ¿bajo que circunstancias 
sucede?. La tarea ahora será responder esto.\\

Si empezamos con un espacio Huasdorff, entonces tenemos que... (falta escribir bien está parte).

\section{El $Q$-cuadrado}\label{Qcuadrado}

Con lo visto hasta este momento, sabemos que para los marcos $A$ y $\mathcal{O}S$, si consideremos filtros $F\in A^\wedge$ y $\nabla\in \mathcal{O}S^\wedge$, estos están en correspondencia con su respectivo núcleo ajustado
$v_F$ y $v_\nabla$. Estos producen los cocientes $A_{v_F}$ y $\mathcal{O}S_{v_\nabla}$ de los marcos principales $A$ y $\mathcal{O}S$, respectivamente. Para simplificar la notación, denotaremos por $A_F$ al cociente $A_{v_F}$ 
y por $\mathcal{O}S_\nabla$ al cociente $\mathcal{O}S_{v_\nabla}$. Con todo lo anterior tenemos el siguiente diagrama
\[
\begin{tikzcd}
	A && {A_F} \\
	&&& {\mathcal{O}Q} \\
	{\mathcal{O}S} && {\mathcal{O}S_\nabla}
	\arrow["{v_F}", from=1-1, to=1-3]
	\arrow["{U_A}"', from=1-1, to=3-1]
	\arrow[from=1-3, to=2-4]
	\arrow["g", from=1-3, to=3-3]
	\arrow["{v_\nabla}"', from=3-1, to=3-3]
	\arrow[from=3-3, to=2-4]
\end{tikzcd}
\]
El diagrama anterior es presentado por Simmons (citar Vietoris). Nosotros lo llamamos el \emph{$Q$-cuadrado}.\\

Recordemos que los núcleos $v_F$ y $v_\nabla$ son las cerraduras idempotentes de sus prenúcleos correspondientes (ver Definición \ref{Definición8.2.1}). Así, $v_F$ y $v_\nabla$ son mapeos de $A$ en $A$ y de $\mathcal{O}S$ en $\mathcal{O}S$, respectivamente.\\

De esta manera, tenemos el siguiente cuadrado, que en (citar el Vietoris y los resultados) prueban que el diagrama
\[\begin{tikzcd}
	A & A \\
	{\mathcal{O}S} & {\mathcal{O}S}
	\arrow["{f^\infty}", from=1-1, to=1-2]
	\arrow["{U_A}"', from=1-1, to=2-1]
	\arrow["{U_A}", from=1-2, to=2-2]
	\arrow["{F^\infty}"', from=2-1, to=2-2]
\end{tikzcd}\]
conmuta laxamente, es decir, $U_A\circ f^\infty\leq F^\infty \circ U_A$.\\

En este diagrama $U_A$ es el morfismo reflexión espacial, $f^\infty$ y $F^\infty$ representan los núcleos asociados a los filtros $F\in A^\wedge$ y $\nabla\in \mathcal{O}S^\wedge$.\\

Lo que probaremos aquí es más general, pues consideramos el cuadrado 
\[\begin{tikzcd}
	A & A \\
	{A_j} & {A_j}
	\arrow["{\hat{f}^\infty}", from=1-1, to=1-2]
	\arrow["j"', from=1-1, to=2-1]
	\arrow["j", from=1-2, to=2-2]
	\arrow["{f^\infty}"', from=2-1, to=2-2]
\end{tikzcd}\]

\begin{lem}
    Para $j$, $f$ y $\hat{f}$ como antes, se cumple que $j\circ \hat{f}\leq f\circ j$.
\end{lem}

\begin{proof}\label{f1f}
    Por la Proposición \ref{F=jF} se cumple que
    \[
    \hat{f}=\dot{\bigvee}\{v_y\mid y\in \hat{F}=j_*F\}\quad  \mbox{ y} \quad f=\dot{\bigvee}\{v_{j(y)}\mid j(y)\in F\}. 
    \]
Luego, para $a\in A$ se cumple que
\[
v_y(a)=(y\succ a)\leq \hat{f}(a)\leq j(\hat{f}(a))\leq (j(y)\succ j(a))=v_{j(y)}(j(a))\leq f(j(a)).
\]
Por lo tanto $j\circ \hat{f}\leq f\circ j$.
\end{proof}

Para que $\hat{f}$ y $f$ sean núcleos, necesitamos sus cerraduras idempotentes.

\begin{cor}\label{finftyf}
    Para $j$, $f$ y $\hat{f}$ como antes, se cumple que $j\circ \hat{f}^\infty\leq f^\infty\circ j$
\end{cor}

\begin{proof}
    Para $\alpha$ un ordinal, verificaremos que $j\circ \hat{f}^\alpha\leq f^\alpha\circ j$. Para ello, lo haremos por inducción.\\

    Si $\alpha=0$, el resultado es trivial.\\

    Para el paso de inducción, supongamos que para $\alpha$ es cierto. Luego
    \[
    j\circ \hat{f}^{\alpha+1}=j\circ \hat{f}\circ \hat{f}^{\alpha}\leq  f\circ j\circ \hat{f}^\alpha\leq f\circ f^\alpha\circ j=f^{\alpha+1}\circ j,
    \]
    donde la primera desigualdad es el Lema \ref{f1f} y la segunda se obtiene por la hipótesis de inducción.\\

    Si $\lambda$ es un ordinal límite, también es sencillo verificar que $j\circ \hat{f}^\lambda\leq f^\lambda\circ j$ (HAZLO)
\end{proof}