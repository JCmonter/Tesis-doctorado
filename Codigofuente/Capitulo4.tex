\chapter{Marcos arreglados vs Axiomas tipo Hausdorff}\label{Relación MA y AH}

Notemos que la Definición \ref{Definición8.2.1} identifica a un tipo especifico de marcos. El Teorema \ref{Teorema8.4.4} caracteriza a los marcos arreglados por medio de una condición especial, de manera especifica con el axioma de separación $T_2$.
Con esto en mente, resulta natural el preguntarse: \emph{¿se pueden enunciar caracterizaciones para los marcos arreglados que sean meramente de marcos?}, es decir, que no se haga mención a la topología de un espacio. Un ejemplo no tan exitos de lo que 
que acabamos de preguntar lo tenemos en el siguiente resultado.

\begin{cor}\label{TadyHausdorff}
Para $A$ un marco espacial se cumple lo siguiente: $\mathcal{O}S$ es un marco Hausdorff si y solo si $A$ es $1$-arreglado. 
\end{cor}
La prueba se sigue del hecho de que la propiedad $\mathbf{(H)}$ es conservativa.\\

De esta manera, hemos relacionado la condición de arreglo con uno de los axiomas de separación. Veamos cual es la relación de esta con las otras propiedades.\\

\section{Arreglado y su relación con los propiedades de separación en $\Frm$}

El Lema \ref{Lema8.4.1} nos dice que arreglado implica $T_1$. ¿Qué pasa con las propiedades más fuertes que $T_1$?

\begin{lem}\label{TadyFHausdorff}
    Todo marco fuertemente Hausdorff es arreglado.
\end{lem}

\begin{proof}
    Consideremos $A\in \Frm$ fuertemente Hausdorff. Si $A$ cumple $\mathbf{(fH)}$, entonces todo sublocal compacto es cerrado. Por teoría de marcos, para $j\in NA$ arbitrario, $A_j$ es compacto si y solo si $\nabla(j)\in A^\wedge$. De aquí que, al ser compacto y por $\mathbf{(fH)}$ $A_j=A_{u_d}$, para algún $d\in A$, es decir, $j=u_d$ y $\nabla(j)=\nabla(u_d)$ para algún $d\in A$, en particular, por H-M, para todo $F\in A^\wedge$, $v_F\in NA$. Así $\nabla(v_F)=\nabla(u_d)$, es decir, para $x\in F$ se cumple que $u_d(x)=1=d\vee x$. Por lo tanto $A$ es arreglado.
\end{proof}

Con esto tenemos las implicaciones
\[
\mathbf{(fH)} \Rightarrow \mbox{ Arreglado }\Rightarrow T_1
\]

Resulta natural el pensar que la propiedad $\mathbf{(H)}$ este metida entre ellas. La realidad es que hasta el dia de hoy no hemos logrado demostrar que 
\[
\mbox{\emph{Todo marco Hausdorff es arreglado.}}
\]

Aunque no lo menciona, Sexton prueba en su tesis doctoral que si un marco es $T_3$, entonces este es parche trivial. Además, tenemos que parche trivial si y solo si arreglado. Por lo tanto, tenemos que
\[
T_3\Rightarrow \mbox{ Arreglado.}
\] 
Por otro la
do, Simmons en su artículo (citar regular and fitness), prueba que 
\[
\mathbf{(aju)}\Rightarrow \mbox{ Arreglado.}
\]

Las pruebas que ellos realizan nos llevan a pensar que existe cierto comportamiento en algo que nosotros llamamos \emph{intervalos de admisibilidad}. Este tema los trataremos más adelante.\\

Por el momento, solo podemos suponer que alguna relación debe existir entre la propiedad $\mathbf{(H)}$ y la condición de arreglo. Hasta que no tengamos una prueba exitosa de este hecho (porque vaya que lo hemos intentado) o exista un ejemplo que exhiba información relevante,
lo anterior seguirá siendo una duda.\\

Lo que sigue ahora es analizar el comportamiento del marco $PA$ (el marco de parches de un marco $A$), con los axiomas de separación

\subsection{¿Qué propiedades de separación cumple el marco de parches?}\label{Parchesyseparación}

Aunque al final de esta pequeña subsección pueda sentirse que es innecesaria, este análisis nos hace pensar que el marco $P^2A$ es en el que debemos enfocar en mayor parte nuestro interés. Recordemos que el marco de parches viene inspirado por la definición de espacio de parches.
La noción de que un marco sea parche trivial (o equivalentemente arreglado), se basa en imitar lo que se conoce como espacio empaquetado. En el Capítulo \ref{Parches} se prueba que la construcción del espacio de parches se estaviliza en el tercer paso, es decir,
\[
^{pp}S=\,^{ppp}S,
\]
de esta manera, la sospecha que tenemos es que $P^2A$ debe actuar de manera similar. Veamos que ocurre que el parche antecesor.\\

\begin{itemize}
\item Observemos que si un marco es regular, entonces es arreglado. Si el marco es arreglado, entonces este es parche trivial. Por definición, un marco $A$ es parche trivial si $A\simeq PA$. Por lo tanto, si $PA$ fuera regular, siempre se cumpliria que $A$ es parche trivial, lo cual no es cierto siempre.
\item Similar al punto anterior, no puede cumplirse que $PA$ cumpla $\mathbf{(H)}$, pues de ser así, también implicaría que todos el marco $A$ es parche trivial y en general, esto no es cierto.
\item Si $PA$ es $T_1$, entonces para todo $j\in \pt PA$ se cumple que $j$ es máximo. Además, si $j\in \pt PA$, entonces $j$ es un punto salvaje o $j$ es un punto ordinario, es decir,
\[
j=w_p\quad \mbox{ o }\quad j\neq w_p
\]
para algún $p\in \pt A$. Si $j$ es salvaje, entonces para $a\in A$ se cumple que 
\[
u_a\leq j\leq w_a
\]
donde $a=j(0)$. Lo anterior nos dice que $j\in \pt PA$ no siempre es máximo y con ello no necesariamente será un marco $T_1$.
\end{itemize}

Como mencionamos al principio, puede resultar desepcionante el darse cuenta que $PA$ no cumple ninguna de las propiedades que teniamos en mente. Por tal motivo habrá que investigar que ocurre con $P^2A$.

\section{Intervalos de admisibilidad}