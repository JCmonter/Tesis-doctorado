\documentclass[12pt]{article}

% --------------------
% PAQUETES BÁSICOS
% --------------------
\usepackage[spanish,es-noshorthands]{babel}
\usepackage[utf8]{inputenc}      % Si usas pdflatex
\usepackage[T1]{fontenc}
\usepackage{lmodern}

\usepackage{amsmath, amssymb, amsthm}
\usepackage{geometry}
\usepackage{setspace}
\usepackage{graphicx}
\usepackage{tikz}
\usepackage{tikz-cd}
\usepackage[
  colorlinks=true,
  linkcolor=blue,
  citecolor=blue,
  urlcolor=red
]{hyperref}

\newtheorem{thm}{Teorema}[section]
\newtheorem{dfn}[thm]{Definición}
\newtheorem{lem}[thm]{Lema}
\newtheorem{ej}[thm]{Ejemplo}
\newtheorem{cor}[thm]{Corolario}
\newtheorem{obs}[thm]{Observación}
\newtheorem{con}[thm]{Conjetura}
\newtheorem{prop}[thm]{Proposición}

\DeclareMathOperator{\Frm}{Frm}
\DeclareMathOperator{\Ord}{Ord}
\DeclareMathOperator{\Top}{Top}
\DeclareMathOperator{\pt}{pt}
\DeclareMathOperator{\id}{id}
\DeclareMathOperator{\tp}{tp}
\DeclareMathOperator{\fd}{fd}
% --------------------
% CONFIGURACIÓN
% --------------------
\geometry{margin=2.5cm}
\onehalfspacing

% --------------------
% DATOS DEL DOCUMENTO
% --------------------
\title{Reporte mensual de actividades}
\date{13-enero-2026 a 31-enero-2026}
\author{Juan Carlos Monter Cortés}

% --------------------
% INICIO DEL DOCUMENTO
% --------------------
\begin{document}

\maketitle

Si consideramos $A\in \Frm$, entonces $S=\pt A$ denota el espacio de puntos de $A$. Además, si $x\in A$
\[
\mathcal{U}_A(x)=\{p\in S\mid x\nleq p\}
\]
es un abierto en $\mathcal{O}S$. Lo anterior proporciona el morfismo de marcos $\mathcal{U}_A:A\to \mathcal{O}S$ definido por $x\mapsto \mathcal{U}_A(x)$ el cual es suprayectivo y se le conoce como la reflexión espacial.
De manera adicional, $\mathcal{U}_A$ es un isomorfismo si y sólo si $A$ es espacial, es decir, $A\simeq \mathcal{O}S$ para algún $S\in \Top$. Además, podemos omitir el subíndice de la reflexión espacial si es claro cual
es el marco con el que se está trabajando.\\

\section{Núcleos y derivadas}
Sabemos que $j\colon A\to A$ es un núcleo en $A$ si es una función que infla, monótona, idempotente y que respeta ínfimos finitos. Si la anterior no cumple con la idempotencia, entonces al operador se le conoce como \emph{prenúcleo}.\\

Para $a, x\in A$ podemos definir los núcleos conocidos
\[
u_a(x)=a\vee x, \quad v_a(x)=(a\succ x), \quad w_a(x)=((x\succ a)\succ a).
\]

Si $A=\mathcal{O}S$ y $E\subseteq S$ definimos la función
\[
[E](U)=(E\cup U)^{\circ}
\]
donde $U\in \mathcal{O}S$. La anterior resulta ser un núcleo en $\mathcal{O}S$ y se le conoce como \emph{núcleo espacialmente inducido}. De manera particular, recordado que la implicación en $\Top$ se calcula por $(W\succ U)=(W'\cup U)^\circ$ para cada $U, W\in \mathcal{O}S$ tenemos que 
\[
u_W=[W]\quad y \quad v_W=[W'].
\] 

Si a un núcleo espacialmente inducido le calculamos su complemento dual, obtenemos un operador sobre $\mathcal{C}S$. Dicho operador resulta ser una \emph{derivada idempotente}.

\begin{dfn}\label{derivadaidempotente}
Sea $S\in \Top$ y $\mathcal{C}S$ la colección de subconjuntos cerrados de $S$. Un operador $\partial\colon \mathcal{C}S\to \mathcal{C}S$ es una derivada si satisface:
\begin{enumerate}
\item Si $X\subseteq Y$ entonces $\partial(X)\subseteq \partial(Y)$,
\item $\partial(X)\subseteq X$,
\item $\partial(X\cup Y)=\partial(X)\cup \partial(Y)$
\end{enumerate}
para todo $X,Y\in \mathcal{C}S$. De manera adicional, decimos que la derivada es idempotente si $\partial^2=\partial$.
\end{dfn}

Tanto para núcleos como para derivadas, el orden se define puntualmente y considerando $\id$ el operador identidad, $\tp$ el operador constante $x\mapsto 1$ y $\fd$ el operador constante $x\mapsto 0$ obtenemos
\[
\id\leq j\leq \tp\quad \text{y}\quad \id\geq \partial\geq \fd
\]
para $j$ un núcleo y $\partial$ una derivada.\\

Como mencionamos antes, si tomamos el núcleo $v_W=[W']$ y calculamos su complemento dual obtenemos
\begin{equation}\label{eq:derivada}
(v_{W}(U))'=([W'](U))'=((W'\cup U)^{\circ})'=(((W\cap U')')^\circ)'=\overline{(W\cap U')}=\partial_{W}(U').
\end{equation}
En este caso, la última proporciona el operador $\partial_{W}$ que es una derivada idempotente.\\

Decimos que $F\subseteq A$ es un filtro abierto si satisface:
\begin{enumerate}
\item $1\in F$,
\item Si $x, y\in F$ entonces $x\wedge y\in F$,
\item Si $x\in F$ y $x\leq y$ entonces $y\in F$. 
\item Si $X\subseteq A$ es un conjunto dirigido y $\bigvee X\in F$ entonces existe $x\in X$ tal que $x\in F$.
\end{enumerate}

Denotamos por $A^\wedge$ al conjunto de todos los filtros abiertos en $A$. Decimos que $f$ es un núcleo ajustado si este tiene la forma
\[
f=\bigvee\{v_a\mid a\in F\}
\]
y es un núcleo en $A$. Tomando el supremo puntual obtenemos
\begin{equation}\label{eq:prenucleo}
f_F=\dot{\bigvee}\{v_a\mid a\in F\}
\end{equation}
el cual resulta ser un prenúcleo en $A$.\\

Considerando $F\in A^\wedge$ e iterando sobre los ordinales construimos la cadena creciente de prenúcleos
\[
f_F^0=\id, \quad f_F^{\alpha+1}=f_F\circ f_F^{\alpha}, \quad f_F^{\lambda}=\bigvee\{f_F^{\alpha}\mid \alpha<\lambda\}
\]
para cada ordinal $\alpha$ y cada ordinal límite $\lambda$. De manera usual, denotamos por $\infty$ al ordinal bajo el cual la sucesión se estabiliza, es decir, $f_F^{\infty+1}=f_F^\infty$. A este se le conoce como la cerradura idempotente y haciendo uso de el definimos
\[
v_F=f_F^{\infty}.
\]
donde $v_F$ es el núcleo ajustado asociado a $F$.\\

Si $A=\mathcal{O}S$, tenemos que el prenúcleo (\ref{eq:prenucleo}) tiene la forma
\[
f_F=\dot{\bigvee}\{[U']\mid U\in F\}
\]
y por el Teorema de Hofmann-Mislove obtenemos $f_F=\dot{\bigvee}\{[U']\mid Q\subseteq U\}$ donde $Q\in \mathcal{Q}S$ es el conjunto compacto saturado asociado a $F$.\\ 

Calculando su complemento dual obtenemos la derivada $\partial_F(X)=\bigcap\{\overline{(X\cap U)}\mid Q\subseteq U\}$
donde $X\in \mathcal{C}S$. Por lo tanto, tenemos que 
\begin{equation}\label{eq:complementodual}
\partial_F(X)=f_F(X')'.
\end{equation} 
Al operador $\partial_F$ se le conoce como la $Q$-derivada (ver Lema 6.1 de \cite{simmons2004vietoris}).\\

Similarmente a como construimos la sucesión creciente de prenúcleos $f_F^\alpha$ tenemos la siguiente sucesión decreciente de $Q$-derivadas
\[
\partial_F^0=\id, \quad \partial^{\alpha+1}_F=\partial_F(\partial_F^\alpha), \quad \partial_F^\lambda=\bigcap\{\partial_F^\alpha\mid \alpha<\lambda\}.
\]

La manera de relacionar ambas sucesiones se muestra en el siguiente lema.

\begin{lem}\label{inducciontransfinita}
Para todo $\alpha\in \Ord$ y $X\in CS$, se cumple
\begin{equation}\label{eq:induccion}
\partial_F^\alpha(X)=\big(f_F^\alpha(X')\big)'.
\end{equation}
\end{lem}

\begin{proof}
Procedemos por inducción transfinita sobre $\alpha\in \Ord$.

\smallskip
\noindent\textbf{Caso base $\alpha=0$.}
Por definición, $\partial_F^0=\id$, es decir, $\partial_F^0(X)=X$ y $f_F^0=\id$, luego
\[
\partial_F^0(X)=X=(X'')=\big(f_F^0(X')\big)'.
\]

\smallskip
\noindent\textbf{Caso sucesor.}
Supongamos que la afirmación es cierta para $\alpha$ y probemos para $\alpha+1$.
Por definición de iteración,
\[
\partial_F^{\alpha+1}(X)=\partial_F\big(\partial_F^\alpha(X)\big).
\]
Además, por (\ref{eq:complementodual})
\[
\partial_F^{\alpha+1}(X)
=(f_F((\partial_F^\alpha(X))'))'.
\]
Por hipótesis inducción, $\partial_F^\alpha(X)=\big(f_F^\alpha(X')\big)'$ y sustituyendo obtenemos
\[
\partial_F^{\alpha+1}(X)=(f_F(f_F^\alpha(X')))'=\big(f_F^{\alpha+1}(X')\big)'
\]
que es lo que queríamos.

\smallskip
\noindent\textbf{Caso límite.}
Sea $\lambda$ un ordinal límite y supongamos que lo anterior se cumple para todo $\alpha<\lambda$.
Por definición de la iteración,
\[
\partial_F^\lambda(X)=\bigcap_{\alpha<\lambda}\partial_F^\alpha(X).
\]
Por De Morgan,
\[
\big(\partial_F^\lambda(X)\big)'
=\Big(\bigcap_{\alpha<\lambda}\partial_F^\alpha(X)\Big)'
=\bigcup_{\alpha<\lambda}\big(\partial_F^\alpha(X)\big)'.
\]
Usando la hipótesis de inducción, para todo $\alpha<\lambda$,
\[
\big(\partial_F^\alpha(X)\big)'=f_F^\alpha(X'),
\]
Luego,
\[
\big(\partial_F^\lambda(X)\big)'=\bigcup_{\alpha<\lambda} f_F^\alpha(X')=f_F^\lambda(X').
\]
Por lo tanto,
\[
\big(\partial_F^\lambda(X)\big)'=f_F^\lambda(X'),
\]
es decir,
\[
\partial_F^\lambda(X)=\big(f_F^\lambda(X')\big)'.
\]
y obtenemos lo que queríamos.
\end{proof}

Un caso particular del lema anterior es cuando $\alpha=\infty$.
\begin{cor}\label{cor:infinito}
\begin{equation}\label{eq:infinito}
\partial_F^\infty(X)=(v_F(X'))'.
\end{equation}
\end{cor}

\section{Nociones de apilamiento}

Los operadores definidos hasta ahora se relacionan por medio de las nociones de apilado y fuertemente apilado.

\begin{dfn}\label{def:apilado}
Consideremos $S\in\Top$.
\begin{enumerate}
  \item Un conjunto $Q\in \mathcal{Q}S$ es apilado si $\overline{Q}=\partial_F^\infty(S)$ y es fuertemente apilado si $v_F=[Q']$.
  \item El espacio $S$ es apilado o fuertemente apilado si cada $Q\in \mathcal{Q}S$ es apilado o fuertemente apilado, respectivamente.
\end{enumerate}
\end{dfn}

La anterior es la noción sensible a puntos introducida por Sexton y Simmons en \cite{sexton2006ordinal} (Definición 6.1) y cumplen que fuertemente apilado implica apilado. Nosotros damos las nociones libres de puntos, es decir, las definiciones en $\Frm$.

\begin{dfn}\label{def:apilado-marcos}
Siguiendo la notación anterior, consideremos $A\in \Frm$. Decimos que $A$ es:
\begin{enumerate}
  \item apilado si $(\mathcal{U}_*([Q'])\mathcal{U})(0)=v_F(0)$.
  \item fuertemente apilado si $\mathcal{U}_*[Q']\mathcal{U}=v_F$.
\end{enumerate}
\end{dfn}

Trivialmente tenemos que fuertemente apilado implica apilado. Nuestro objetivo es demostrar que las propiedades anteriores son conservativas, es decir, que un espacio $S$ es apilado (fuertemente apilado) si y sólo si su marco de abiertos $\mathcal{O}S$ es apilado (fuertemente apilado). 
Antes de hacer esto necesitamos el siguiente lema auxiliar.

\begin{lem}\label{lem:auxiliar}
Si $A=\mathcal{O}S$ y $E\subseteq S$ se cumple que
\[
\mathcal{U}_*[E]\mathcal{U}=[E].
\]
\end{lem}

\begin{proof}
Como el marco $A$ es espacial se cumple que $\mathcal{U}$ es un isomorfismo. De aquí que si $U\in \mathcal{O}\pt A$ existe $V\in \pt A$ tal que $U=\mathcal{U}(V)$ y como $U\in \pt A$, se cumple que $U=V$. Luego
\[
\begin{split}
(\mathcal{U}_*[E]\mathcal{U})(U)&=\mathcal{U}_*[E](\mathcal{U}(U))=\mathcal{U}_*([E](U))\\
&=\mathcal{U}_*(E\cup U)^{\circ}=\bigcup\{W\in \mathcal{O}S\mid W\subseteq (E\cup U)^{\circ}\}\\
&=(E\cup U)^{\circ}=[E](U).
\end{split}
\]
Por lo tanto $\mathcal{U}_*[E]\mathcal{U}=[E]$.
\end{proof}

\begin{prop}\label{apiladoconservativa}
Las nociones de fuertemente apilado y apilado son conservativas.
\end{prop}

\begin{proof}
Consideremos $S\in \Top$, $A=\mathcal{O}S$, $F\in A^\wedge$ y $Q\in \mathcal{Q}S$.\\

Primero, supongamos que $S$ es fuertemente apilado. De aquí que $v_F=[Q']$ y por el Lema \ref{lem:auxiliar} tenemos que $\mathcal{U}_*[Q']\mathcal{U}=[Q']=v_F$. Por lo tanto $\mathcal{O}S$ es fuertemente apilado.\\

Supongamos ahora que $S$ es apilado. Por lo anterior se cumple que 
\[
\overline{Q}=\partial_F^\infty(S)\iff \overline{Q}'=(\partial_F^{\infty}(S))'.
\]

Además, 
\[
\begin{split}
(\mathcal{U}_*[Q']\mathcal{U})(\emptyset)&=\mathcal{U}_*([Q'](\{V\in \pt\mathcal{O}S\mid \emptyset \not\subseteq V\}))\\
&=\mathcal{U}_*([Q'](\emptyset))=\mathcal{U}_*(Q')^{\circ}\\
&=\bigcup\{W\in \mathcal{O}S\mid W\subseteq (Q')^\circ\}\\
&=(Q')^{\circ}=\overline{Q}'.
\end{split}
\]
Por el Corolario \ref{cor:infinito} tenemos que $v_F(\emptyset)=(\partial_F^\infty(\emptyset'))'$. Por lo tanto 
\[
v_F(\emptyset)=(\mathcal{U}_*[Q']\mathcal{U})(\emptyset),
\]
es decir, $\mathcal{O}S$ es apilado.\\
\end{proof}

Nuestro siguiente paso será el verificar cual es el comportamiento de las nociones de apilamiento en conjunto con la eficiencia. 

\section{El marco $[0,1]$}

Siguiendo la idea mostrada en \cite{simmons2006regularity} (o de manera alternativa \cite{simmons2004vietoris}), consideremos el marco linealmente ordenado $A=[0,1]$. 
En este caso particular podemos hacer las siguientes observaciones:
\begin{enumerate}
\item Para cualesquiera $a,b\in A$ la implicación $(a\succ b)$ se calcula por
\[
(a\succ b)=\begin{cases}
1 & \text{si } a\leq b,\\
b & \text{si } a>b.
\end{cases}
\]
\item Los núcleos distinguidos $u_a, v_a, w_a$ para $a\in A$ se calculan como
\[
u_a(x)=\begin{cases}
  x & \text{si }a\leq x,\\
  a & \text{si }x< a,
\end{cases}, \quad v_a(x)=\begin{cases}
1 & \text{si } x\geq a,\\
a & \text{si } x<a,
\end{cases} \quad\mbox{ y} \quad w_a(x)=\begin{cases}
1 & \text{si } x>a,\\
a & \text{si } x\leq a.
\end{cases}
\]
\item Si $F\subseteq A$ es un filtro, entonces este siempre es admisible y tiene la forma
\[
F=\nabla(w_a)=(a, 1]\quad \mbox{ o }\quad F=\nabla(v_a)=[a, 1]
\]
para algún $a\in A$. De manera particular, $F\in A^\wedge$ si y solo si $F=(a, 1]$.
\item Si $F\in A^\wedge$, el núcleo ajustado $v_F$ se calcula como
\begin{equation}\label{eq:lm}
v_F(x)=l_m(x)=\begin{cases}
1 & \text{si } x> m,\\
x & \text{si } x\leq m.
\end{cases}
\end{equation}
donde $l_m=v_m\wedge w_m$ y $m=\bigwedge F$.
\item Para $S=\pt A$ se cumple que $S=[0,1)$ y los abiertos en $\mathcal{O}S$ son de la forma
\[
\mathcal{U}_A(x)=\{p\in S\mid p<x\}=[0,x).
\]
\item El marco $A$ no es $T_1$ pues ningún $p\in S$ es máximo. Por lo tanto, $A$ no satisface alguna otra propiedad de separación en $\Frm$.
\item Para $F\in A^\wedge$ el conjunto $Q\in \mathcal{Q}S$ asociado a  $F$ es de la forma $Q=[0,m]$ donde $m=\bigwedge F$.
\item Se cumple que $[Q']=v_F$ y por lo tanto $S$ es fuertemente apilado.
\item Ya que $[0, x)\in \mathcal{O}S$, entonces $[x, 1]\in \mathcal{C}S$. Como $Q=[0, m]\in \mathcal{Q}S$, tenemos que los subconjuntos compactos no son cerrados,
es decir, $S$ no es empaquetado.
\item $A$ no es eficiente.

\end{enumerate}

Todo lo anterior es un resumen de las principales características del marco $[0,1]$. Lo que 
sigue es realizar el análisis de las construcciones de parches tanto de 
$A$ como de $S$.

\subsection{Análisis de las construcciones de parches}
Recordemos que para $A\in \Frm$ y $S\in \Top$ tenemos 
\[
\text{Pbase}=\{u_a\wedge v_F\mid a\in A, F\in A^\wedge\}\quad \text{y}\quad \text{pbase}=\{U\cap Q'\mid U\in \mathcal{O}S, Q\in \mathcal{Q}S\}
\]
donde Pbase genera al marco de parches de $A$ (denotado por $PA$) y pbase genera la topología del espacio de parches de $S$ (denotado por $^pS$). Por (\ref{eq:lm}) tenemos que 
\[
\text{Pbase}=\{u_a\wedge l_m\mid a\in [0,1], m\in [0,1)\}.
\]
Por lo tanto, si queremos conocer el comportamiento del marco $PA$ necesitamos saber la relación que existe entre $a$ y $m$.\\

Para $x\in A$ tenemos que $(u_a\wedge l_m)(x)=u_a(x)\wedge l_m(x)$. Primero veamos la relación entre $x$ y $m$.
\begin{description}
\item[Caso 1:] Si $x<m$, entonces $(u_a\wedge l_m)(x)=u_a(x)\wedge x=x$.
\item[Caso 2:] Si $x=m$, entonces $(u_a\wedge l_m)(x)=u_a(m)\wedge m=x$.
\item[Caso 3:] Si $x>m$, entonces $(u_a\wedge l_m)(x)=u_a(x)\wedge 1=u_a(x)$.
\end{description}
Notemos que en los primeros 2 casos el valor de $a$ no afecta el resultado de la evaluación. Además $u_a\wedge l_m=\id$, es decir, los generadores básicos de la 
Pbase son distinto a $\id$ cuando $m<x$. Ahora veamos la relación entre $a$ y $m$ en el tercer caso.
\begin{description}
\item[Caso 3.1:] Si $m<x<a$, entonces $(u_a\wedge l_m)(x)=a$.
\item[Caso 3.2:] Si $m<a\leq x$, entonces $(u_a\wedge l_m)(x)=x$.
\end{description}
De esta manera, tenemos que la Pbase está conformada por
\[
\text{Pbase}=\begin{cases}
  \{u_a\wedge l_m\} & \text{si } m<a,\\
  \{\id\} & \text{en cualquier otro caso}.
\end{cases}
\]
es decir, el marco $PA$ está codificado por los intervalos de la forma $(m, a)$, donde $m\in \pt A$ y $a\in A$.\\

Queremos identificar si en nuestro ejemplo, $PA$ satisface alguna propiedad de separación. Para hacerlo, ocuparemos recordar información respecto a los parches.\\

Si $A\in\Frm$ es un marco arbitrario y $S$ su espacio de puntos, podemos construir el siguiente diagrama conmutativo que relaciona ambas construcciones de parches.
\[\begin{tikzcd}
	A & PA & NA \\
	{\mathcal{O}S} & {P\mathcal{O}S} & {N\mathcal{O}S} \\
	& {\mathcal{O}^pS} & {\mathcal{O}^fS}
	\arrow[from=1-1, to=1-2]
	\arrow["{U_A}"', from=1-1, to=2-1]
	\arrow[hook, from=1-2, to=1-3]
	\arrow["{P(U_A)}", from=1-2, to=2-2]
	\arrow["{N(U_A)}", from=1-3, to=2-3]
	\arrow[from=2-1, to=2-2]
	\arrow[hook, from=2-1, to=3-2]
	\arrow[hook, from=2-2, to=2-3]
	\arrow["\pi", from=2-2, to=3-2]
	\arrow["\sigma", from=2-3, to=3-3]
	\arrow[hook, from=3-2, to=3-3]
\end{tikzcd}\]
donde $\mathcal{O}^fS$ es la topología de Skula (o topología frontal) del espacio $S$. En \cite{sexton2006point} lo llaman el 
\emph{diagrama del ensamble de parches} y más detalles sobre este también pueden ser encontrados
en \cite{sexton2003point}.\\

En nuestro caso, $U_A$ es un isomorfismo. Además, bajo $U_A$, la construcción de parches resulta ser funtorial y, por lo tanto,
$PA\cong P\mathcal{O}S$. De manera similar, $NA\cong N\mathcal{O}S$.
Para obtener más información sobre $P\mathcal{O}S$ y $N\mathcal{O}S$ podemos hacer uso de los siguientes resultados, el cual puede consultarse en \cite{AvilaBezhaniMorandiZaldivar2020}.

\begin{prop}\label{Ensamble espacial}
Sea $S$ un conjunto parcialmente ordenado.
\begin{enumerate}
  \item Si $S$ no tiene anticadenas infinitas, entonces $N\mathcal{O}S$ es espacial.
  \item Si $S$ es totalmente ordenado, entonces $N\mathcal{O}S$ es espacial.
\end{enumerate}
\end{prop}

Así, por la Proposición \ref{Ensamble espacial} tenemos que $N\mathcal{O}S$ es espacial y por lo tanto $N\mathcal{O}S\cong \mathcal{O}^fS$. Juntando toda esta información 
el diagrama del ensamble de parches en nuestro caso particular queda como sigue.

\[\begin{tikzcd}
	A & PA & NA \\
	{\mathcal{O}S} & {\mathcal{O}^pS} & {\mathcal{O}^fS}
	\arrow[from=1-1, to=1-2]
	\arrow["\cong"', from=1-1, to=2-1]
	\arrow[from=1-2, to=1-3]
	\arrow["\cong", from=1-2, to=2-2]
	\arrow["\cong", from=1-3, to=2-3]
	\arrow[from=2-1, to=2-2]
	\arrow[from=2-2, to=2-3]
\end{tikzcd}\]
Por lo tanto, si queremos estudiar al marco $PA$ podemos estudiar a la topología de parches $\mathcal{O}^pS$. Recordemos que 
\[
\text{pbase}=\{U\cap Q'\mid U\in \mathcal{O}S, Q\in \mathcal{Q}S\}.
\]
y, en nuestro caso,
\[
\text{pbase}=\{[0, a)\cap [m, 1)\mid a\in [0,1], m\in [0,1)\}=\{(m,a)\}.
\]
Notemos que la anterior es la base para la topología usual en $[0,1)$  y al ser $^pS$ un espacio Hausdorff, $\mathcal{O}^pS\cong PA$ satisface $\mathbf{(H)}$. De manera adicional, $PA$ es un marco $KC$.\\

\section{Actividades realizadas}
Durante el presente periodo, se llevaron a cabo las siguientes actividades:
\begin{enumerate}
  \item Definimos las nociones libres de puntos de apilado y fuertemente apilado.
  \item Verificamos que las nociones anteriores son conservativas.
  \item Culminamos el análisis del marco $[0,1]$ y su construcción de parches.
\end{enumerate}

\bibliographystyle{amsalpha}
\bibliography{research2}

\end{document}