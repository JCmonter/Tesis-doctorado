\documentclass[12pt]{article}

% --------------------
% PAQUETES BÁSICOS
% --------------------
\usepackage[spanish,es-noshorthands]{babel}
\usepackage[utf8]{inputenc}      % Si usas pdflatex
\usepackage[T1]{fontenc}
\usepackage{lmodern}

\usepackage{amsmath, amssymb, amsthm}
\usepackage{geometry}
\usepackage{setspace}
\usepackage[
  colorlinks=true,
  linkcolor=blue,
  citecolor=blue,
  urlcolor=red
]{hyperref}

\newtheorem{thm}{Teorema}[section]
\newtheorem{dfn}[thm]{Definición}
\newtheorem{lem}[thm]{Lema}
\newtheorem{ej}[thm]{Ejemplo}
\newtheorem{cor}[thm]{Corolario}
\newtheorem{obs}[thm]{Observación}
\newtheorem{con}[thm]{Conjetura}
\newtheorem{prop}[thm]{Proposición}

\DeclareMathOperator{\Frm}{Frm}
\DeclareMathOperator{\Ord}{Ord}
\DeclareMathOperator{\Top}{Top}
\DeclareMathOperator{\pt}{pt}
\DeclareMathOperator{\id}{id}
\DeclareMathOperator{\tp}{tp}
% --------------------
% CONFIGURACIÓN
% --------------------
\geometry{margin=2.5cm}
\onehalfspacing

% --------------------
% DATOS DEL DOCUMENTO
% --------------------
\title{Reporte mensual de actividades}
\date{13-enero-2026 a 31-enero-2026}
\author{Juan Carlos Monter Cortés}

% --------------------
% INICIO DEL DOCUMENTO
% --------------------
\begin{document}

\maketitle

Si consideramos $A\in \Frm$, entonces $S=\pt A$ denota el espacio de $A$. Además, si $x\in A$
\[
\mathcal{U}_A(x)=\{p\in S\mid x\nleq p\}
\]
es un abierto en $\mathcal{O}S$. Lo anterior proporciona el morfismo de marcos $\mathcal{U}_A:A\to \mathcal{O}S$ definido por $x\mapsto \mathcal{U}_A(x)$ el cual es suprayectivo y se le conoce como la reflexión espacial.
De manera adicional, $\mathcal{U}_A$ es un isomorfismo si y sólo si $A$ es espacial, es decir, $A\simeq \mathcal{O}S$ para algún $S\in \Top$. Además, podemos omitir el subíndice de la reflexión espacial si es claro cual
es el marco con el que se está trabajando.\\

Sabemos que $j\colon A\to A$ es un núcleo en $A$ si es una función que infla, monótona, idempotente y que respeta ínfimos finitos. Para $a, x\in A$ podemos definir los núcleos conocidos
\[
u_a(x)=a\vee x, \quad v_a(x)=(a\succ x), \quad w_a(x)=((x\succ a)\succ a).
\]

Si $A=\mathcal{O}S$ y $E\subseteq S$ definimos la función
\[
[E](U)=(E\cup U)^{\circ}
\]
donde $U\in \mathcal{O}S$. La anterior resulta ser un núcleo en $\mathcal{O}S$ y se le conoce como \emph{núcleo espacialmente inducido}. De manera particular, recordado que la implicación en $\Top$ se calcula por $(W\succ U)=(W'\cup U)^\circ$ para cada $U, W\in \mathcal{O}S$ tenemos que 
\[
u_W=[W]\quad y \quad v_W=[W'].
\] 

Si a un núcleo espacialmente inducido le calculamos el complemento, obtenemos un operador sobre $\mathcal{C}S$, 
las diferencias que existen entre estos radican en que los últimos desinflan y respetan 
supremos finitos. Por ejemplo
\begin{equation}\label{eq:derivada}
(v_{W'}(U))'=([W](U))'=((W\cup U)^{\circ})'=\overline{(W'\cap U')}=\partial_{W'}(U')
\end{equation}
donde al operador $\partial_{E'}$ se le conoce como derivada sobre $\mathcal{C}S$.\\

Decimos que $F\subseteq A$ es un filtro abierto si satisface:
\begin{enumerate}
\item $1\in F$,
\item Si $x, y\in F$ entonces $x\wedge y\in F$,
\item Si $x\in F$ y $x\leq y$ entonces $y\in F$. 
\item Si $X\subseteq A$ es un conjunto dirigido y $\bigvee X\in F$ entonces existe $x\in X$ tal que $x\in F$.
\end{enumerate}

Denotamos por $A^\wedge$ al conjunto de todos los filtros abiertos en $A$. Decimos que $f_F$ es un núcleos ajustado si este tiene la forma
\[
f=\bigvee\{v_a\mid a\in F\}
\]
el cual es un núcleo en $A$. Tomando el supremo puntual obtenemos el prenúcleo
\begin{equation}\label{eq:prenucleo}
f_F=\dot{\bigvee}\{v_a\mid a\in F\}.
\end{equation}

Considerando $F\in A^\wedge$ e iterando sobre los ordinales construimos la cadena creciente de prenúcleos
\[
f_F^0=\id, \quad f_F^{\alpha+1}=f_F\circ f_F^{\alpha}, \quad f_F^{\lambda}=\bigvee\{f_F^{\beta}\mid \alpha<\lambda\}
\]
para cada ordinal $\alpha$ y cada ordinal límite $\lambda$. 

En el caso espacial, tenemos que el prenúcleo (\ref{eq:prenucleo}) tiene la forma
\[
f_F=\dot{\bigvee}\{[U']\mid U\in F\}.
\]

Finalmente, definimos 
\[
v_F=f_F^\infty
\]
como el núcleo ajustado asociado a $F$.\\

Por el Teorema de Hofmann-Mislove-Johnstone tenemos una correspondencia biyectiva entre 
\[
Q\longleftrightarrow F\longleftrightarrow v_F
\]
donde $Q\in \mathcal{Q}S$, $F\in A^\wedge$ y $v_F$ el núcleo ajustado asociado a $F$.\\

Si $A=\mathcal{O}S$, tenemos la siguiente relación entre filtros abiertos y compactos saturados 
\[
U\in F\iff Q\subseteq U
\]
y así el prenúcleo (\ref{eq:prenucleo}) tiene la forma $f_F=\dot{\bigvee}\{[U']\mid Q\subseteq U\}$. Calculando su complemento obtenemos el operador
\[
\partial_F(X)=\bigcap\{\overline{(X\cap U)}\mid Q\subseteq U\}
\]
donde $X\in \mathcal{C}S$ (ver Lema 6.1 de \cite{simmons2004vietoris}). Al anterior se le conoce como la $Q$-derivada.\\

Similarmente a como construimos la sucesión creciente de prenúcleos, tenemos la siguiente sucesión decreciente de $Q$-derivadas
\[
\partial_F^0=\tp, \quad \partial^{\alpha+1}_F=\partial_F(\partial_F^\alpha(S)), \quad \partial_F^\nabla=\bigcap\{\partial_F^\alpha\mid \alpha<\lambda\}.
\]

De esta manera tenemos que 
\begin{equation}\label{eq:complemento}
  \partial_F^\alpha(X)=\partial_F(\partial_F^{\alpha-1}(X))=(f_F(\partial_F^{\alpha-1}(X))')'=(f_F(f_F^{\alpha-1}(X')))'=(f_F^\alpha(X'))'
\end{equation}
donde $X\in \mathcal{C}S$ y para todo $\alpha\in \Ord$. De manera particular,
\[
\partial_F^\infty(X)=(v_F(X'))'.
\]

\section{Nociones de apilamiento}

Los operadores definidos hasta ahora se relacionan por medio de las nociones de apilado y fuertemente apilado.

\begin{dfn}\label{def:apilado}
Consideremos $S\in\Top$.
\begin{enumerate}
  \item Un conjunto $Q\in \mathcal{Q}S$ es apilado si $\overline{Q}=\partial_F^\infty(S)$ y es fuertemente apilado si $v_F=[Q']$.
  \item El espacio $S$ es apilado o fuertemente apilado si cada $Q\in \mathcal{Q}S$ es apilado o fuertemente apilado, respectivamente.
\end{enumerate}
\end{dfn}

La anterior es la noción sensible a puntos introducida por Sexton y Simmons en \cite{sexton2006ordinal} (Definición 6.1) y cumplen que fuertemente apilado implica apilado. Nosotros damos las nociones libres de puntos, es decir, las definiciones en $\Frm$.

\begin{dfn}\label{def:apilado-marcos}
Siguiendo la notación anterior, consideremos $A\in \Frm$. Decimos que $A$ es:
\begin{enumerate}
  \item apilado si $(\mathcal{U}_*([Q'])\mathcal{U})(0)=v_F(0)$.
  \item fuertemente apilado si $\mathcal{U}_*[Q']\mathcal{U}=v_F$.
\end{enumerate}
\end{dfn}

Trivialmente tenemos que fuertemente apilado implica apilado. Nuestro objetivo es demostrar que las propiedades anteriores son conservativas, es decir, que un espacio $S$ es apilado (fuertemente apilado) si y sólo si su marco de abiertos $\mathcal{O}S$ es apilado (fuertemente apilado). 
Antes de hacer esto necesitamos el siguiente lema auxiliar.

\begin{lem}\label{lem:auxiliar}
Si $A=\mathcal{O}S$ y $E\subseteq S$ se cumple que
\[
\mathcal{U}_*[E]\mathcal{U}=[E].
\]
\end{lem}

\begin{proof}
Como el marco $A$ es espacial se cumple que $\mathcal{U}$ es un isomorfismo. De aquí que si $U\in \mathcal{O}\pt A$ existe $V\in \pt A$ tal que $U=\mathcal{U}(V)$ y como $U\in \pt A$, se cumple que $U=V$. Luego
\[
\begin{split}
(\mathcal{U}_*[E]\mathcal{U})(U)&=\mathcal{U}_*[E](\mathcal{U}(U))=\mathcal{U}_*([E](U))\\
&=\mathcal{U}_*(E\cup U)^{\circ}=\bigcup\{W\in \mathcal{O}S\mid W\subseteq (E\cup U)^{\circ}\}\\
&=(E\cup U)^{\circ}=[E](U).
\end{split}
\]
Por lo tanto $\mathcal{U}_*[E]\mathcal{U}=[E]$.
\end{proof}

\begin{prop}\label{apiladoconservativa}
Las nociones de fuertemente apilado y apilado son conservativas.
\end{prop}

\begin{proof}
Consideremos $S\in \Top$, $A=\mathcal{O}S$, $F\in A^\wedge$ y $Q\in \mathcal{Q}S$.\\

Primero, supongamos que $S$ es fuertemente apilado. De aquí que $v_F=[Q']$ y por el Lema \ref{lem:auxiliar} tenemos que $\mathcal{U}_*[Q']\mathcal{U}=[Q']=v_F$. Por lo tanto $\mathcal{O}S$ es fuertemente apilado.\\

Supongamos ahora que $S$ es apilado. Por lo anterior se cumple que 
\[
\overline{Q}=Q(\infty)(S)\Leftrightarrow \overline{Q}'=(Q(\infty)(S))'.
\]

Además, 
\[
\begin{split}
(\mathcal{U}_*[Q']\mathcal{U})(\emptyset)&=\mathcal{U}_*([Q'](\{V\in \pt\mathcal{O}S\mid \emptyset \not\subseteq V\}))\\
&=\mathcal{U}_*([Q'](\emptyset))=\mathcal{U}_*(Q')^{\circ}\\
&=\bigcup\{W\in \mathcal{O}S\mid W\subseteq (Q')^\circ\}\\
&=(Q')^{\circ}=\overline{Q}'.
\end{split}
\]
Por (\ref{eq:complemento}) tenemos que $v_F(\emptyset)=(Q(\infty)(\emptyset'))'$. Por lo tanto 
\[
v_F(\emptyset)=(\mathcal{U}_*[Q']\mathcal{U})(\emptyset),
\]
es decir, $\mathcal{O}S$ es apilado.\\
\end{proof}

\section{El marco $[0,1]$}

\section{Actividades realizadas}
Durante el presente periodo, se llevaron a cabo las siguientes actividades:
\begin{enumerate}
  \item Definimos las nociones libres de puntos de apilado y fuertemente apilado.
  \item Verificamos que las nociones anteriores son conservativas.
  \item 
\end{enumerate}

\bibliographystyle{amsalpha}
\bibliography{research2}

\end{document}