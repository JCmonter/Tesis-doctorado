\documentclass{protocol_template}


\title{Modificaciones de parches y algunos axiomas de separación en la topología sin puntos}
\author{Juan Carlos Monter Cortés}
\advisor{Dr. Luis Ángel Zaldívar Corichi} 

%% usar solo si hay codirector. Si es más de un codirector, agregarlos separados por coma
%\coadvisor{Grado y nombre completo del codirector, y del segundo codirector}  
\date{ enero de 2026}

\begin{document}

\makefrontpage

%\begin{center}
%{\Large (5 cuartillas máximo)}
%\end{center}

\section{Introducción}

Se incluyen en esta parte antecedentes del tema. Contextualizar el área de estudio y mostrar el ``Estado del Arte".

\section{Planteamiento del Problema}

Describe qué se quiere saber o resolver.

\section{Hipótesis}

Propone una posible respuesta al planteamiento del problema.

\section{Objetivos}

%Los objetivos constituyen el propósito del trabajo a realizar y esto se puede dividir en dos partes: Objetivos Generales y Objetivos Particulares.

\section{Metodología}

%En esta parte se describen las tareas que se van a realizar para llevar a cabo los objetivos planteados anteriormente, enfatizando en los procedimientos y en las herramientas que se van a utilizar.
El plan de trabajo para la realización de este proyecto de investigación es el que se describe a continuación.\\
\begin{table}[htbp]
    \caption{Cronograma}
    \centering
    \begin{tabular}{ccccccccc}
        \toprule
        Semestre & 1 & 2 & 3 & 4 & 5 & 6 & 7 & 8 \\
        \midrule
 Revisión de bibliografía  & \checkmark & \checkmark & \checkmark & \checkmark & \checkmark & \checkmark &  &  \\\hline
Lectura de artículos  & \checkmark & \checkmark & \checkmark & \checkmark & \checkmark & \checkmark &  &  \\\hline
Proponer conjeturas &  & \checkmark & \checkmark & \checkmark & \checkmark & \checkmark & & \\\hline
Probar resultados &  &  &  \checkmark & \checkmark & \checkmark & \checkmark & \checkmark & \\\hline
Validación y rechazo de conjeturas &  &  &  \checkmark & \checkmark & \checkmark & \checkmark & \checkmark & \\\hline
Redacción de artículos y otros documentos &  & \checkmark & \checkmark  & \checkmark  & \checkmark & \checkmark  & \checkmark & \checkmark \\\hline
Desarrollar conclusiones &  &  &  & & & \checkmark & \checkmark & \checkmark \\\hline
Sustentación &  &  &  & & & & & \checkmark \\
        \bottomrule
    \end{tabular}

\end{table}

\nocite{*}
\bibliography{research2}
\vspace{24pt} 

%\textbf{Consideraciones a tener sobre las referencias.}
%\begin{itemize}
%    \item Se permitirán referencias clásicas fundamentales, aún cuando sean antiguas.
%    \item La mayoría de las referencias deberán ser recientes. Preferentemente, publicadas en los últimos 5 años y que reflejen el estado actual del problema de investigación.
%    \item Las referencias deberán presentarse en un formato estándar reconocido (APA, AMS o similar).
%    \item El apartado de referencias deberá incluir un mínimo de diez fuentes bibliográficas.
%\end{itemize}


\end{document}
