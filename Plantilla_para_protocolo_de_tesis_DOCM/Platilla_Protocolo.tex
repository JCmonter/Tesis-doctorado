\documentclass{protocol_template}

\newtheorem{thm}{Teorema}[section]
\newtheorem{dfn}[thm]{Definición}
\newtheorem{lem}[thm]{Lema}
\newtheorem{ej}[thm]{Ejemplo}
\newtheorem{cor}[thm]{Corolario}
\newtheorem{obs}[thm]{Observación}
\newtheorem{con}[thm]{Conjetura}
\newtheorem{prop}[thm]{Proposición}

\DeclareMathOperator{\Frm}{Frm}
\DeclareMathOperator{\CRing}{CRing}
\DeclareMathOperator{\Ord}{Ord}
\DeclareMathOperator{\Top}{Top}
\DeclareMathOperator{\Spec}{Spec}
\DeclareMathOperator{\pt}{pt}
\DeclareMathOperator{\id}{id}
\DeclareMathOperator{\op}{op}
\DeclareMathOperator{\tp}{tp}
\DeclareMathOperator{\fd}{fd}

\title{Modificaciones de parches y algunos axiomas de separación en la topología sin puntos}
\author{Juan Carlos Monter Cortés}
\advisor{Dr. Luis Ángel Zaldívar Corichi} 

%% usar solo si hay codirector. Si es más de un codirector, agregarlos separados por coma
%\coadvisor{Grado y nombre completo del codirector, y del segundo codirector}  
\date{\today}

\begin{document}

\makefrontpage

%\begin{center}
%{\Large (5 cuartillas máximo)}
%\end{center}

\section{Introducción}
La primera modificación de parches fue introducida por Hochster en \cite{hochster1969prime}. Su idea está motivada por el funtor 
\[
\Spec\colon \CRing^{\op}\to \Top
\]
el cual a cada anillo conmutativo con unidad $A$ le asigna el espacio topológico $S=\Spec(A)$. Debido a que esta asignación no siempre produce espacios $T_2$,
Hochster propone una ``reparación'' del espacio $S$ mediante el funtor 
\[
^p\colon \Top\to \Top
\]
donde el espacio resultante $^pS$ es un espacio Hausdorff. Al espacio $^pS$ se le conoce como el \emph{espacio de parches} y cuando $S\in \Top$ es un espacio arbitrario, su espacio de parches 
no necesariamente es Hausdorff.\\

De manera análoga, la categoría $\Frm$ se relacionada con la categoría $\Top$ por medio de los funtores contravariantes
\[
\mathcal{O}\colon \Top\to \Frm\quad \text{y}\quad \pt\colon \Frm\to \Top
\]
los cuales forman una adjunción entre ambas categorías. Esta adjunción constituye una herramienta fundamental en la teoría de marcos para trasladar nociones
topológicas clásicas a sus variantes ``point-free''. Un ejemplo de ello son las modificaciones de parches introducidas por Escardó y Sexton (véase \cite{escardo2001regular,sexton2006point}).\\

En particula, para cada $A\in \Frm$ se consideran los funtores
\[
M\colon \Frm\to \Frm\quad \text{y}\quad P\colon \Frm\to \Frm 
\]
donde $MA$ es el \emph{parche continuo} y $PA$ es el \emph{marco de parches}. Ambas construcciones buscan capturar, dentro de $\Frm$, ciertos comportamiento característicos 
del espacio de parches.\\

Por ejemplo, para $S\in \Top$, si todo subconjunto compacto es cerrado, decimos que el espacio $S$ es \emph{empaquetado}, y un espacio se dice empaquetado si y solo si $^pS=S$.
Esta noción se traduce al lenguaje de marcos diciendo que un marco $A$ es \emph{parche trivial} si
\[
A\cong PA.
\]
Los autores de \cite{sexton2006point,sexton2006ordinal,sexton2003point} caracterizan esta propiedad mediante lo que denominan \emph{marcos eficientes},
mostrando además que da lugar a una jerarquía de propiedades indexada por ordinales, es decir, para $\alpha\in \Ord$
se cumple que 
\[
1\text{-eficiente}\Rightarrow 2\text{-eficiente}\Rightarrow \dots \Rightarrow \alpha\text{-eficiente}\Rightarrow \alpha +1\text{-eficiente}\Rightarrow \dots
\]

Parte de las propiedades que aparecen en la literatura relacionan a los marcos eficientes con los axiomas clásicos de separación. Por ejemplo,

\begin{thm}\label{Teorema1}
    Un espacio $S$ que es $T_0$ tiene topología $1$-eficiente si y solo si $S$ es $T_2$. 
\end{thm}

Asimismo, en teoría de marcos existen jerarquías de propiedades de separación del tipo
\[
\mathbf{(reg)}\Rightarrow \mathbf{(fH)}\Rightarrow \mathbf{(H)}\Rightarrow T_1
\]
(véase \cite{picado2021separation}). \\

Adicionalmente, se cuenta con la caracterización:
\[
A\text{ es parche tivial}\iff A\text{ es eficiente}\iff S \text{ es empaquetado y apilado}.
\]

En \cite{sexton2006ordinal} se muestra que la noción de apilado se relaciona con la construcción del hiperespacio de Vietoris, mientras que en \cite{simmons2004vietoris} se presenta
su versión en marcos, denominada $V$-modificación.\\

De manera complementaria, existen trabajos que analizan variantes y extensiones de las construcciones de parches en contextos particulares dentro de $\Frm$.
Por ejemplo, \cite{klinke2013yet} estudia una construcción alternativa de parches para marcos continuos, mientras que Balmer y Gallauer \cite{BalmerGallauer2025PatchDensity}
introducen nociones de densidad de parche en geometría tensor-triangular. Aunque estos desarrollos no están formulados directamente en términos de eficiencia,
sugieren que las modificaciones de parches aparecen de manera natural en distintos escenarios.\\

En este proyecto se propone profundizar en el estudio de los marcos eficientes a través de la construcción del marco de parches, analizando su relación con distintos axiomas 
de separación en $\Frm$ y explorando su interacción con otras modificaciones de parches descritas en la literatura.

\section{Planteamiento del Problema}
%Describe qué se quiere saber o resolver.
La noción de marco eficiente surge como una propiedad definida en la categoría $\Frm$ que, originalmente,
es caracterizada mediante condiciones topológicas. En particular, la eficiencia se introduce como una condición necesaria y suficiente 
para que ocurra la trivialidad del parche y se organiza en una jerarquía ordinal que exhibe una estrecha relación con axiomas clásicos 
de separación.\\

Sin embargo, gran parte de los resultados existentes describen esta noción a partir de traducciones desde el contexto de espacios 
topológicos, mientras que un análisis interno, puramente formulado dentro de la categoría $\Frm$, permanece incompleto. Esto motiva
al estudio sistemático de los marcos eficientes como una clase de objetos definida en $\Frm$, independiente de referencias externas 
a espacios.\\

Por otro lado, aunque se conocen diversas jerarquías de axiomas de separación en teoría de marcos,
no está completamente claro cuál es la posición exacta que ocupa la jerarquía de eficiencia dentro de estas cadenas de
implicaciones, ni en que medida puede interpretarse por si solo como una familia de axiomas de separación.\\

En consecuencia, el problema central    que se aborda en este proyecto es determinar como la construcción del marco de parches y
la jerarquía de marcos eficientes se relacionan con los axiomas de separación en $\Frm$, así como identificar caracterizaciones internas
de eficiencia que permitan comprender su naturaleza y alcance.

\section{Hipótesis}
%Propone una posible respuesta al planteamiento del problema.
\textbf{Hipótesis general:}

Es posible caracterizar la eficiencia de manera interna dentro de la categoría $\Frm$, y la jerarquía que esta clase 
de marcos produce puede interpretarse como una familia de propiedades análogas a axiomas de separación.\\

\textbf{Hipótesis especificas:}
 \begin{enumerate}
\item Existen caracterizaciones internas de los marcos eficientes formuladas únicamente en términos de la 
estructura del marco principal y del marco de parches asociado.

\item La jerarquía de marcos eficientes se puede insertar, total o parcialmente, dentro de las 
jerarquías conocidas de axiomas de separación en $\Frm$.

\item Las modificaciones de parches y otras construcciones relacionadas, como la $V$-modificación, inducen
condiciones suficientes o necesarias para la eficiencia en marcos.
\end{enumerate}

\section{Objetivos}
\textbf{Objetivo general:}

Analizar la construcción del marco de parches y su relación con los marcos eficientes, con el fin de comprender su interacción con distintos axiomas de separación 
en la categoría $\Frm$.\\

%Los objetivos constituyen el propósito del trabajo a realizar y esto se puede dividir en dos partes: Objetivos Generales y Objetivos Particulares.

\textbf{Objetivos específicos:}
\begin{enumerate}
\item Estudiar las modificaciones de parches y los axiomas de separación desde el enfoque de $\Frm$.
\item Obtener caracterizaciones internas de marcos eficientes en términos de núcleos, sublocales o condiciones asociadas al marco de parches.
\item Investigar la posición de la jerarquía de marcos eficientes dentro de las jerarquías conocidas de axiomas de separación en teoría de marcos.
\item Analizar la relación entre la eficiencia y otras modificaciones de parches y la $V$-modificación.
\item Desarrollar ejemplos y contraejemplos que ilustren las relaciones encontradas. 
\end{enumerate}
\section{Metodología}

%En esta parte se describen las tareas que se van a realizar para llevar a cabo los objetivos planteados anteriormente, enfatizando en los procedimientos y en las herramientas que se van a utilizar.

La investigación se desarrollará dentro del contexto de la matemática pura, específicamente en teoría de marcos y topología sin puntos. El trabajo será principalmente de carácter 
analítico y deductivo, apoyado en el estudio crítico de literatura especializada y en el desarrollo de nuevas demostraciones.\\

En una primera etapa se realizará una revisión sistemática de bibliografía relacionada con teoría general de marcos, axiomas de separación, el marco de parches y marcos eficientes, con el propósito de identificar 
resultados fundamentales, técnicas recurrentes y problemas abiertos. De manera especifica se consultará \cite{PicadoPultr,Johnstone82, picado2021separation,sexton2003point,Zaldivar2023, simmonscollection, simmonsbasics}.\\ 

Posteriormente, se estudiará con detalles de la construcción del marco de parches y del parche continuo, así como su formulación en términos de filtros, núcleos y sublocales, por ejemplo \cite{escardo2006compactly, escardo2001regular, sexton2006point, sexton2006ordinal, gierz2012compendium}. 
A partir de este análisis se buscarán caracterizaciones internas de la eficiencia y de las diferentes nociones que se logren definir.\\

En una etapa posterior se investigará la relación entre la jerarquía de marcos eficientes y los axiomas de separación en teoría de marcos, con el objetivo de determinar 
posibles implicaciones, equivalencias o independencias.\\

De manera complementaria, se analizará la interacción de la eficiencia y el marco de parches con otras modificaciones de parches, como la V-modificación,  se estudiarán ejemplos y contraejemplos que permitan ilustrar los fenómenos observados.
Es esta parte, consultaremos \cite{klinke2013yet, simmons2004vietoris, johnstone2020vietoris, simmons2006fundamental, avila2020frame, simmons2006regularity}\\

Finalmente, los resultados obtenidos se organizarán en un artículo y capítulos de tesis, y se someterán a discusión con especialistas del área mediante seminarios y estancias de investigación.\\

A manera de resumen, el plan de trabajo para la realización de este proyecto de investigación es el que se describe a continuación.\\
\begin{table}[htbp]
\caption{Cronograma}
\centering
\begin{tabular}{lcccccccc}
\toprule
Actividad / Semestre & 1 & 2 & 3 & 4 & 5 & 6 & 7 & 8 \\
\midrule
Revisión de bibliografía & \checkmark & \checkmark & \checkmark & \checkmark & \checkmark & \checkmark &  &  \\
Lectura de artículos & \checkmark & \checkmark & \checkmark & \checkmark & \checkmark & \checkmark &  &  \\
Proponer conjeturas &  & \checkmark & \checkmark & \checkmark & \checkmark & \checkmark &  &  \\
Probar resultados &  &  & \checkmark & \checkmark & \checkmark & \checkmark & \checkmark &  \\
Validación y rechazo de conjeturas &  &  & \checkmark & \checkmark & \checkmark & \checkmark & \checkmark &  \\
Redacción de artículo y otros documentos &  & \checkmark & \checkmark & \checkmark & \checkmark & \checkmark & \checkmark & \checkmark \\
Desarrollar conclusiones &  &  &  &  &  & \checkmark & \checkmark & \checkmark \\
Sustentación &  &  &  &  &  &  &  & \checkmark \\
\bottomrule
\end{tabular}
\end{table}


\nocite{*}
\bibliographystyle{alpha}
\bibliography{references}
%\bibliography{research2}
%\textbf{Consideraciones a tener sobre las referencias.}
%\begin{itemize}
%    \item Se permitirán referencias clásicas fundamentales, aún cuando sean antiguas.
%    \item La mayoría de las referencias deberán ser recientes. Preferentemente, publicadas en los últimos 5 años y que reflejen el estado actual del problema de investigación.
%    \item Las referencias deberán presentarse en un formato estándar reconocido (APA, AMS o similar).
%    \item El apartado de referencias deberá incluir un mínimo de diez fuentes bibliográficas.
%\end{itemize}


\end{document}
