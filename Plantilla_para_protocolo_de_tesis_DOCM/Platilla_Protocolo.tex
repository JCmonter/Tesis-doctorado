\documentclass{protocol_template}

\newtheorem{thm}{Teorema}[section]
\newtheorem{dfn}[thm]{Definición}
\newtheorem{lem}[thm]{Lema}
\newtheorem{ej}[thm]{Ejemplo}
\newtheorem{cor}[thm]{Corolario}
\newtheorem{obs}[thm]{Observación}
\newtheorem{con}[thm]{Conjetura}
\newtheorem{prop}[thm]{Proposición}

\DeclareMathOperator{\Frm}{Frm}
\DeclareMathOperator{\Ord}{Ord}
\DeclareMathOperator{\Top}{Top}
\DeclareMathOperator{\pt}{pt}
\DeclareMathOperator{\id}{id}
\DeclareMathOperator{\tp}{tp}
\DeclareMathOperator{\fd}{fd}

\title{Modificaciones de parches y algunos axiomas de separación en la topología sin puntos}
\author{Juan Carlos Monter Cortés}
\advisor{Dr. Luis Ángel Zaldívar Corichi} 

%% usar solo si hay codirector. Si es más de un codirector, agregarlos separados por coma
%\coadvisor{Grado y nombre completo del codirector, y del segundo codirector}  
\date{ enero de 2026}

\begin{document}

\makefrontpage

%\begin{center}
%{\Large (5 cuartillas máximo)}
%\end{center}

\section{Introducción}

La construcción de parches introducida por Sexton en \cite{sexton2006point} busca dar una variante libre de puntos que se comporte de manera similar a lo que topológicamente se conoce como \emph{espacio de parches}. 
De manera especifica, se busca imitar la propiedad de que un espacio topológico $S$ sea empaquetado. Lo anterior se cumple si y solo si  
\[
^pS=S,
\]
donde $^pS$ es el espacio de parches de $S$. Esto es traducido al lenguaje de marcos como la propiedad de que un marco $A$ sea \emph{parche trivial}. En otras palabras, un marco $A$ es parche trivial si
\[
A\simeq PA,
\]
donde $PA$ es conocido como el \emph{marco de parches}. Los autores de \cite{sexton2006point}, \cite{sexton2006ordinal} y \cite{sexton2003point} se encargan de caracterizar esta propiedad por medio de lo que ellos 
definen como \emph{marcos eficientes}. Ellos prueban que dicha propiedad genera un jerarquía estratificada sobre los ordinales, es decir, para $\alpha\in \Ord$
se cumple que 
\[
1\text{-eficiente}\Rightarrow 2\text{-eficiente}\Rightarrow \dots \Rightarrow \alpha\text{-eficiente}\Rightarrow \alpha +1\text{-eficiente}\Rightarrow \dots
\]

Parte de las propiedades que aparecen en la literatura relacionan a los marcos eficientes con los axiomas clásicos de separación. 
El siguiente teorema proporciona un ejemplo de ello.

\begin{thm}\label{Teorema1}
    Un espacio $S$ que es $T_0$ tiene topología $1$-eficiente si y solo si $S$ es $T_2$. 
\end{thm}

En la actualidad, el estudio de los axiomas de separación es un área de interés dentro de los grupos de investigación sobre teoría de marcos. De manera similar a los axiomas para espacios topológicos, 
los axiomas dentro de la categoría de marcos se relacionan de manera jerárquica, en este caso
\[
\mathbf{(reg)}\Rightarrow \mathbf{(fH)}\Rightarrow \mathbf{(H)}\Rightarrow T_1
\]
donde $\mathbf{(reg)}$ es la propiedad de ser \emph{regular}, $\mathbf{(fH)}$ es \emph{fuertemente Hausdorff} y $\mathbf{(H)}$ la de ser \emph{Hausdorff} (ver \cite{picado2021separation}). De esta manera resulta natural preguntarnos lo siguiente: 
\emph{¿se puede incluir la propiedad de eficiencia en alguna parte de esta jerarquía?}.\\

De manera adicional, se tiene la siguiente caracterización
\[
A\text{ es parche tivial}\iff A\text{ es eficiente}\iff S \text{ es empaquetado y apilado}.
\]

Con ello, resulta de nuestro interés estudiar a los marcos eficientes y conocer nociones en la teoría 
de marcos que se relaciones con la propiedad de eficiencia, por ejemplo dar los análogos en marcos para 
apilado.\\

De la misma manera en la que Sexton introduce su noción de parches, existen distintos autores que han establecidos otras construcciones de parches. Todas 
ellas motivadas por la idea original de Hochster \cite{hochster1969prime} y su construcción del espacio de parches. Algunos ejemplos de estas construcciones 
pueden consultarse en \cite{escardo2001regular} y \cite{klinke2013yet}. Bajo ciertas condiciones, la construcción presentada por Escardo en \cite{escardo2001regular}
termina siendo isomorfa a la de Sexton. La construcción de Klinke en \cite{klinke2013yet} puede considerarse como un caso más general con respecto a $PA$.\\

Por lo tanto, debido a la relación que existe entre los marcos eficientes y los marcos parche trivial,
el explorar la relación que existe entre la eficiencia y las diferentes construcciones podría proporcionar información 
adicional en el estudio de dichos marcos.
 
\section{Planteamiento del Problema}
%Describe qué se quiere saber o resolver.
Dado que la noción de eficiencia es introducida como una propiedad en marcos que es caracterizada a través de 
propiedades topológicas, resultaría más general estudiar esta clase de marcos dentro de la misma categoría $\Frm$.\\

De igual manera verificar si esta clase de marcos satisface las condiciones necesarias y suficientes como para ser
considerados como una especie de axioma de separación.



\section{Hipótesis}
%Propone una posible respuesta al planteamiento del problema.


\section{Objetivos}

%Los objetivos constituyen el propósito del trabajo a realizar y esto se puede dividir en dos partes: Objetivos Generales y Objetivos Particulares.

\section{Metodología}

%En esta parte se describen las tareas que se van a realizar para llevar a cabo los objetivos planteados anteriormente, enfatizando en los procedimientos y en las herramientas que se van a utilizar.
El plan de trabajo para la realización de este proyecto de investigación es el que se describe a continuación.\\
\begin{table}[htbp]
    \caption{Cronograma}
    \centering
    \begin{tabular}{ccccccccc}
        \toprule
        Semestre & 1 & 2 & 3 & 4 & 5 & 6 & 7 & 8 \\
        \midrule
 Revisión de bibliografía  & \checkmark & \checkmark & \checkmark & \checkmark & \checkmark & \checkmark &  &  \\\hline
Lectura de artículos  & \checkmark & \checkmark & \checkmark & \checkmark & \checkmark & \checkmark &  &  \\\hline
Proponer conjeturas &  & \checkmark & \checkmark & \checkmark & \checkmark & \checkmark & & \\\hline
Probar resultados &  &  &  \checkmark & \checkmark & \checkmark & \checkmark & \checkmark & \\\hline
Validación y rechazo de conjeturas &  &  &  \checkmark & \checkmark & \checkmark & \checkmark & \checkmark & \\\hline
Redacción de artículos y otros documentos &  & \checkmark & \checkmark  & \checkmark  & \checkmark & \checkmark  & \checkmark & \checkmark \\\hline
Desarrollar conclusiones &  &  &  & & & \checkmark & \checkmark & \checkmark \\\hline
Sustentación &  &  &  & & & & & \checkmark \\
        \bottomrule
    \end{tabular}

\end{table}

%\nocite{*}
%\bibliographystyle{amsalpha}
\bibliography{references}
%\bibliography{research2}
\vspace{24pt} 

%\textbf{Consideraciones a tener sobre las referencias.}
%\begin{itemize}
%    \item Se permitirán referencias clásicas fundamentales, aún cuando sean antiguas.
%    \item La mayoría de las referencias deberán ser recientes. Preferentemente, publicadas en los últimos 5 años y que reflejen el estado actual del problema de investigación.
%    \item Las referencias deberán presentarse en un formato estándar reconocido (APA, AMS o similar).
%    \item El apartado de referencias deberá incluir un mínimo de diez fuentes bibliográficas.
%\end{itemize}


\end{document}
