\documentclass{protocol_template}

\newtheorem{thm}{Teorema}[section]
\newtheorem{dfn}[thm]{Definición}
\newtheorem{lem}[thm]{Lema}
\newtheorem{ej}[thm]{Ejemplo}
\newtheorem{cor}[thm]{Corolario}
\newtheorem{obs}[thm]{Observación}
\newtheorem{con}[thm]{Conjetura}
\newtheorem{prop}[thm]{Proposición}

\DeclareMathOperator{\Frm}{Frm}
\DeclareMathOperator{\Ord}{Ord}
\DeclareMathOperator{\Top}{Top}
\DeclareMathOperator{\pt}{pt}
\DeclareMathOperator{\id}{id}
\DeclareMathOperator{\tp}{tp}
\DeclareMathOperator{\fd}{fd}

\title{Modificaciones de parches y algunos axiomas de separación en la topología sin puntos}
\author{Juan Carlos Monter Cortés}
\advisor{Dr. Luis Ángel Zaldívar Corichi} 

%% usar solo si hay codirector. Si es más de un codirector, agregarlos separados por coma
%\coadvisor{Grado y nombre completo del codirector, y del segundo codirector}  
\date{ enero de 2026}

\begin{document}

\makefrontpage

%\begin{center}
%{\Large (5 cuartillas máximo)}
%\end{center}

\section{Introducción}

La construcción de parches introducida por Sexton en \cite{sexton2006point} busca dar una variante libre de puntos que se comporte de manera similar a lo que topológicamente se conoce como \emph{espacio de parches}. 
De manera especifica, se busca imitar la propiedad de que un espacio topológico $S$ sea empaquetado. Lo anterior se cumple si y solo si  
\[
^pS=S,
\]
donde $^pS$ es el espacio de parches. Esto es traducido al lenguaje de marcos como la propiedad de que un marco $A$ sea \emph{parche trivial}. En otras palabras, un marco $A$ es parche trivial si
\[
A\simeq PA,
\]
donde $PA$ es conocido como el \emph{marco de parches}. Los autores de \cite{sexton2006point}, \cite{sexton2006ordinal} y \cite{sexton2003point} se encargan de caracterizar esta propiedad por medio de lo que ellos 
definen como \emph{marcos eficientes}. Parte de las propiedades demostradas relacionan a los marcos eficientes con los axiomas clásicos de separación. 
El siguiente teorema de un ejemplo de ello.

\begin{thm}\label{Teorema1}
    Un espacio $S$ que es $T_0$ tiene topología $1$-eficiente si y solo si $S$ es $T_2$. 
\end{thm}

En la actualidad, el estudio de los axiomas de separación es un área de interés dentro de los grupos de investigación sobre teoría de marcos. De manera similar a los axiomas para espacios topológicos, 
los axiomas dentro de la categoría de marcos se relacionan de manera jerárquica, en este caso
\[
\mathbf{(reg)}\Rightarrow \mathbf{(fH)}\Rightarrow \mathbf{(H)}\Rightarrow T_1
\]
donde $\mathbf{(reg)}$ es la propiedad de ser \emph{regular}, $\mathbf{(fH)}$ es \emph{fuertemente Hausdorff} y $\mathbf{(H)}$ la de ser \emph{Hausdorff} (ver \cite{picado2021separation}). De esta manera resulta natural preguntarnos lo siguiente: 
\emph{¿se puede incluir la propiedad de eficiencia en alguna parte de esta jerarquía?}.\\


\section{Planteamiento del Problema}

Describe qué se quiere saber o resolver.

\section{Hipótesis}

Propone una posible respuesta al planteamiento del problema.

\section{Objetivos}

%Los objetivos constituyen el propósito del trabajo a realizar y esto se puede dividir en dos partes: Objetivos Generales y Objetivos Particulares.

\section{Metodología}

%En esta parte se describen las tareas que se van a realizar para llevar a cabo los objetivos planteados anteriormente, enfatizando en los procedimientos y en las herramientas que se van a utilizar.
El plan de trabajo para la realización de este proyecto de investigación es el que se describe a continuación.\\
\begin{table}[htbp]
    \caption{Cronograma}
    \centering
    \begin{tabular}{ccccccccc}
        \toprule
        Semestre & 1 & 2 & 3 & 4 & 5 & 6 & 7 & 8 \\
        \midrule
 Revisión de bibliografía  & \checkmark & \checkmark & \checkmark & \checkmark & \checkmark & \checkmark &  &  \\\hline
Lectura de artículos  & \checkmark & \checkmark & \checkmark & \checkmark & \checkmark & \checkmark &  &  \\\hline
Proponer conjeturas &  & \checkmark & \checkmark & \checkmark & \checkmark & \checkmark & & \\\hline
Probar resultados &  &  &  \checkmark & \checkmark & \checkmark & \checkmark & \checkmark & \\\hline
Validación y rechazo de conjeturas &  &  &  \checkmark & \checkmark & \checkmark & \checkmark & \checkmark & \\\hline
Redacción de artículos y otros documentos &  & \checkmark & \checkmark  & \checkmark  & \checkmark & \checkmark  & \checkmark & \checkmark \\\hline
Desarrollar conclusiones &  &  &  & & & \checkmark & \checkmark & \checkmark \\\hline
Sustentación &  &  &  & & & & & \checkmark \\
        \bottomrule
    \end{tabular}

\end{table}

\nocite{*}
\bibliography{references}
%\bibliography{research2}
\vspace{24pt} 

%\textbf{Consideraciones a tener sobre las referencias.}
%\begin{itemize}
%    \item Se permitirán referencias clásicas fundamentales, aún cuando sean antiguas.
%    \item La mayoría de las referencias deberán ser recientes. Preferentemente, publicadas en los últimos 5 años y que reflejen el estado actual del problema de investigación.
%    \item Las referencias deberán presentarse en un formato estándar reconocido (APA, AMS o similar).
%    \item El apartado de referencias deberá incluir un mínimo de diez fuentes bibliográficas.
%\end{itemize}


\end{document}
